\begin{surferPage}[A1+- Singularity]{$A^{+-}_1$ Singularity (Double Cone)}
	The double cone (also called ordinary double point or singularity of type $A_1^{+-}$) is the simplest singularity: when deforming the equation
	\[x^2+y^2-z^2=0\]
	slightly, it becomes smooth.

	The pictures below show surfaces corresponding to the equation
	\[x^2+y^2-z^2=a\]
	for the values $a=-\frac12$, $a=0$, $a=\frac12$:
	\begin{Centering*}%
		\includegraphics[width=1.2cm]{../../common/images/A1pm_0}\quad%
		\includegraphics[width=1.2cm]{../../common/images/A1pm_1}\quad%
		\includegraphics[width=1.2cm]{../../common/images/A1pm_2}%
	\end{Centering*}
	For negative $a$ the deformation is a hyperboloid of two sheets and for positive $a$ it is a connected hyperboloid of one sheet.

	We will see in the following examples that it is possible to deform more
	complicated singularities in a way such that several double cones develop. Each of them can then be deformed into a one- or two-sheeted hyperboloid.
\end{surferPage}
