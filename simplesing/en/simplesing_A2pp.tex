\begin{surferPage}[A2++ Cusp]{$A_2^{++}$ Singularity (Another Cusp)}
	This singularity is similar to the previous cusp-singularity. Its equation is exactly the same except that a sign has changed:
	\[
		x^3+y^2+z^2=0.
	\]
	This change of sign causes the surface to be rotation-symmetric, as is its deformation with equation $(1-a)x^3-ax^2+y^2+z^2=0$, $a>0$, into a singularity of type $A_1^{+-}$:
	\begin{Centering*}%
		\includegraphics[width=1.2cm]{../../common/images/A2pp_0}\quad%
		\includegraphics[width=1.2cm]{../../common/images/A2pp_1}\quad%
		\includegraphics[width=1.2cm]{../../common/images/A2pp_2}%
	\end{Centering*}
	The rotation-symmetry in the $yz$-plane can be seen easily:
	Writing the equation in the form $y^2+z^2=-(x^3)$, we obtain a circle for each fixed value $x<0$, because the equation $y^2+z^2=r^2$ describes a circle with radius $r$ by the Pythagorean Theorem.

	Cutting the above deformations of the cusp with a plane yields plane curves -- a plane cusp and two loops:
	\begin{Centering*}%
		\includegraphics[width=1.2cm]{../../common/images/cuspe_def_cut_1}\qquad%
		\includegraphics[width=1.2cm]{../../common/images/cuspe_def_cut_2}\qquad%
		\includegraphics[width=1.2cm]{../../common/images/cuspe_def_cut_3}%
	\end{Centering*}
\end{surferPage}
