\begin{surferPage}[A2+- Cusp]{$A_2^{+-}$ Singularity (A Cusp)}
	The following equation corresponds to a so-called cusp-singularity:
	\[
		x^3+y^2-z^2=0.
	\]
	This is the surface variant of a plane cusp which we obtain by cutting the surface with an appropriate plane. E.g., $z=0$ yields the plane curve with equation $x^3+y^2=0$:
	\begin{Centering*}%
		\includegraphics[width=1.2cm]{../../common/images/A2pm}\enspace%
		\includegraphics[width=1.2cm]{../../common/images/cuspe_cut}\enspace%
		\includegraphics[width=1.2cm]{../../common/images/cuspe_rot}\enspace%
		\raisebox{1mm}{\includegraphics[width=1.4cm]{../../common/images/kuspe_detail_gross_heller}}%
	\end{Centering*}%
	A cusp appears naturally when light is reflected in a cup as you can see in the rightmost picture!

	One can deform a singularity of type $A_k^{+-}$ in a way such that $\lfloor\frac{k+1}{2}\rfloor$ double cone singularities develop without increasing the degree of the surface. Here, we show the case $k=2$. The deformation is as follows:
	\[
		(1-a)x^3+ax^2+y^2-z^2=0.
	\]
	For $a=0$, we get a cusp, while small values of $a\neq 0$ yield a single double cone singularity since $\lfloor\frac{2+1}{2}\rfloor=1$:
	\begin{Centering*}%
		\includegraphics[width=1.2cm]{../../common/images/A2pm_0}\quad%
		\includegraphics[width=1.2cm]{../../common/images/A2pm_1}\quad%
		\includegraphics[width=1.2cm]{../../common/images/A2pm_2}%
	\end{Centering*}%
\end{surferPage}
