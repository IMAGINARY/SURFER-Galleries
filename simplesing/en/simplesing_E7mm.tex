\begin{surferPage}[E7-- Singularity]{$E_7^{--}$ Singularity}
	A singularity of type $E_7^{--}$ is given by the equation
	\[
		x^3-xy^3-z^2=0.
	\]
	It may be deformed into a surface with four ordinary double points by choosing $a$ and $b$ correctly in the following family of equations:
	\[
		x\cdot\bigl((x-b)\cdot x-y\cdot(y-a)\cdot (y+a)\bigr)-z^2=0.
	\]
	Keeping $b=0$ and varying $a$, we get the image on the left with three ordinary double points as singularities. When varying $b$ as well, as in the rightmost image, a configuration with four ordinary double points can occur:
	\vspace{-1.5ex}
	\begin{center}
		\begin{tabular}{c@{\quad}c@{\quad}c}
			\includegraphics[width=1.1cm]{../../common/images/E7mm_1} &
			\includegraphics[width=1.1cm]{../../common/images/E7mm_0} &
			\includegraphics[width=1.1cm]{../../common/images/E7mm_2}\\
			$a=0.31$ &
			$a=0$ &
			$a=0.31$\\
			$b=0$ &
			$b=0$ &
			$b=0.21$
		\end{tabular}
	\end{center}
	Actually, one can compute that the surface has four singularities whenever $2^{6}a^{6}=3^3b^4$, e.g. for $a=0.25$, $b\approx 0.15510$ or $a=0.5$, $b\approx 0.438691$.
\end{surferPage}
