\begin{surferPage}[E7-- Singularity]{$E_7^{--}$ Singularity}
The following equation corresponds to the so-called $E_7^{--}$-singularity:
    \vspace*{-0.4em}
    \begin{center}
      $x^3-xy^3-z^2=0.$
    \end{center}

  \vspace*{-0.7em}
    \begin{center}
      \begin{tabular}{c}
        \begin{tabular}{@{}c@{}}
          \includegraphics[width=1.2cm]{../../common/images/E7mm_0}
        \end{tabular}
        \end{tabular}
   \end{center}
    \vspace*{-0.4em}

An $E_7$ singularity may be deformed into a surface with four ordinary
double points by choosing $a$ and $b$ correctly in the following
family of equations: 
\[x\cdot\bigl((x-b)\cdot x-y\cdot(y-a)\cdot (y+a)\bigr)-z^2=0\]

Keeping $b=0$ and varying $a$, we get the image at the left with three
ordinary double points as singularities.
When varying $b$, too, you may even manage to realize a situation with 
four ordinary double points (rightmost image):
    \begin{center}
      \begin{tabular}{@{}c@{\quad}c@{\quad}c@{}}
        \begin{tabular}{@{}c@{}}
          \includegraphics[width=1.1cm]{../../common/images/E7mm_1}
        \end{tabular}
        &
        \begin{tabular}{@{}c@{}}
          \includegraphics[width=1.1cm]{../../common/images/E7mm_0}
        \end{tabular}
        &
        \begin{tabular}{@{}c@{}}
          \includegraphics[width=1.1cm]{../../common/images/E7mm_2}
        \end{tabular}
\\
$a=0.31$, $b=0$
& 
$a=0$, $b=0$
&
$a=0.31$, $b=0.21$
      \end{tabular}
    \end{center}

Actually, one can compute that the case of four singularities happens whenever
$2^{6}a^{6}=3^3b^4$.
E.g.: $a=0.25$, $b\approx 0.15510$ or 
$a=0.5$, $b\approx 0.438691$.

\end{surferPage}
