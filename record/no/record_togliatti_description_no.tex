\documentclass[no]{./../../common/SurferDesc}%%%%%%%%%%%%%%%%%%%%%%%%%%%%%%%%%%%%%%%%%%%%%%%%%%%%%%%%%%%%%%%%%%%%%%%
%
% The document starts here:
%
\begin{document}
\footnotesize
% Weltrekordfl�chen

%%% 1.Tafel

%%%%%%%%%%%%%%%%%%%%%%%%%%%%%
\begin{surferPage}
  \begin{surferTitle}Togliatti-flaten av femte grad\end{surferTitle}  \\

Eugenio Giuseppe Togliatti beviste i 1937 at det finnes en flate av femte grad med akkurat $31$ singulariteter --- en verdensrekord p� den tiden. 

I 1980 brukte Arneau Beauville en interessant forbindelse til kodeteori for � vise at flater av femte grad ikke kan ha flere singulariteter. Det betyr at verdensrekorden til Togliatti aldri kan forbedres! 
	
Det finnes ikke noe platonsk legeme som har symmetriplan man kan bruke til � konstruere en flate av femte grad, i tillegg til Kummer --- flaten og Barth ---flaten. Togliatti --- flaten p� bildet har derfor med sine $31$ singulariteter mindre symmetri, nemlig symmetrien til et plant pentagon.  

Ligningen vi bruker her, ble funnet av Wolf Barth (1990), og vi bruker den fordi Togliattis originale flate ikke er s� lett � visualisere.




  \begin{surferText}
     \end{surferText}
\end{surferPage}


\end{document}
%
% end of the document.
%
%%%%%%%%%%%%%%%%%%%%%%%%%%%%%%%%%%%%%%%%%%%%%%%%%%%%%%%%%%%%%%%%%%%%%%%
