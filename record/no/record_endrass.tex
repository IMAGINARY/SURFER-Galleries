\begin{surferPage}[Endraß-flaten]{Endraß-flaten av åttende grad}
	I 1995 konstruerte Stephan Endraß denne flaten av åttende grad som hovedresultatet i hans 
	vitenskapelige avhandling ved Universitetet i Erlangen. Flaten har til sammen $168$ singulariteter, 
	som fortsatt er den gjeldende verdensrekorden. 
  	
	Gjennom et generelt resultat funnet av Varchenko, vet vi at et flater av åttende 
	rad ikke kan ha mer enn $174$ singulariteter. Det vil si: $168 \le \mu(8) \le 174$. Det nøyaktige antallet er ukjent.
	
	Det var ikke lett å finne flaten. Endraß måtte lete etter den i en $5$-dimensjonal familie med flater 
	av åttende grad, hvor det gjennomsnittlige familiemedlemmet bare hadde $112$ singulariteter.

	I det interaktive bildet ser vi symmetrien til konstruksjonen: I tillegg til symmetrien til 
	et vanlig oktogon, er flaten symmetrisk med hensyn på $xy$-planet.

	Uten å bruke slike symmetrier, ville søkerommet ikke bare være av femte grad, men av enda høyere dimensjoner.
\end{surferPage}
