%
% jDM08_expl_engl_galIntros.tex
%
% 
% Einleitungen für die Gallerien in surfer.
%
% Um aus dem pdf automatisiert ein png-Bild zu erhalten, 
% können folgende Kommandos verwendet
% werden: 
%
% pdflatex jDM08_expl_engl.tex ; pdftoppm -f 1 -l 1 -aa yes jDM08_expl_engl.pdf /tmp/jDM08_expl_engl ; convert -geometry 300x600 /tmp/jDM08_expl_engl-000001.ppm jDM08_expl_engl.png
%
%
% English version: March 2009
%
% Oliver Labs
% www.OliverLabs.net
% www.AlgebraicSurface.net
%         
%

\documentclass[sans]{amsart}

%
% Die folgenden Zeilen setzen Höhe und Breite der Gallerie-Einleitungen fest:
%
\newlength{\galIntroHeight}
\newlength{\galIntroWidth}
\setlength{\galIntroHeight}{12cm}
\setlength{\galIntroWidth}{10.34cm}

%
% Die folgenden Zeilen setzen Höhe und Breite des Erklärungs-Textes fest:
%
\newlength{\explHeight}
\newlength{\explWidth}
\setlength{\explHeight}{12cm}
\setlength{\explWidth}{6cm}

%
% Die folgenden Zeilen setzen einen (leeren, d.h.\ weißen) 
% Rahmen von 0.1 cm um den Text:
%
\usepackage[
paperheight=\galIntroHeight
,paperwidth=\galIntroWidth
%,height=9.8cm
%,width=5.8cm
,left=0.1cm
,right=0.1cm
,bottom=0.1cm
,top=0.05cm
]{geometry}

%%%%%
% it seems to me that the papersize has to be specified here and further down, too.
%
\special{papersize=\galIntroHeight,\galIntroWidth}

%
% some latex packages we might need:
%
\usepackage{showidx}
\usepackage[utf8x]{inputenc}
\usepackage[T1]{fontenc}
\usepackage{latexsym}
\usepackage{amsbsy,amscd,amsmath,amstext,amsthm}
\usepackage{amsfonts}
\usepackage{amssymb}
\usepackage{color}
\usepackage{dsfont}
\usepackage{varioref}
\usepackage{epsfig}
\usepackage{eso-pic}
\usepackage{graphicx}
\usepackage[active]{srcltx}
\usepackage{hyperref}
\usepackage[utf8x]{inputenc}
\usepackage[norsk]{babel}
\usepackage[T2A,T1]{autofe}
\usepackage{ifthen}
\usepackage{setspace}
\onehalfspacing

%\usepackage[scaled=0.92]{helvet}
\usepackage{multicol}
\usepackage{pstricks,pst-grad}
\usepackage{rotating}

%%%%%%%%%%%%%%%%%%%%%%%%%%%%%%%%%%%%%%%%%%%%%%
%
% Layout related stuff
%
%%%%%%%%%%%%%%%%%%%%%%%%%%%%%%%%%%%%%%%%%%%%%%

%
% for no indenting for a new paragraph:
%
\setlength{\parindent}{0pt} 
\setlength{\parskip}{1ex plus 0.5ex minus 0.2ex}

%
% don't use any roman fonts:
%
\def\rmdefault{cmss}        % no roman
\def\sfdefault{cmss}
\def\ttdefault{cmtt}
\def\itdefault{sl}
\def\sldefault{sl}
\def\bfdefault{bx}

\usepackage{sfmath} % use sans serif fonts in math mode


%
% some colors we'll use further down
%
\newrgbcolor{lightgrey}{.8 .8 .8}
\newrgbcolor{lightocher}{0.9 0.9 0.7}
\newrgbcolor{ocher}{0.7 0.7 0.5}
%\newrgbcolor{lightgrey}{.8 .8 .8}
%\newrgbcolor{lightocher}{0.9 0.9 0.65}
%\newrgbcolor{ocher}{0.7 0.7 0.35}
%\newrgbcolor{lightgrey}{.8 .8 .8}
%\newrgbcolor{lightocher}{0.8 0.8 0.8}
%\newrgbcolor{ocher}{0.7 0.7 0.7}

%
%
%
\definecolor{titleHgColor}{rgb}{0.0,0.0,0.0}
\definecolor{titleFgColor}{rgb}{0.9,0.9,0.7}
\definecolor{titleFgColorSnd}{rgb}{1.0,1.0,0.8}

\definecolor{picsHgColor}{rgb}{0.0,0.0,0.0}
\definecolor{picsFgColor}{rgb}{0.9,0.9,0.7}
\definecolor{picsFgColorSnd}{rgb}{0.95,0.95,0.75}

\definecolor{textHgColor}{rgb}{1.0,1.0,1.0}
%\definecolor{textHgColor}{rgb}{0.9,0.9,0.9}
\definecolor{textFgColor}{rgb}{0.15,0.15,0.1}
\definecolor{textFgColorSnd}{rgb}{0.4,0.4,0.2}
\definecolor{textFgColorDark}{rgb}{0.1,0.1,0.07}

%
% some commands we often use:
%

%%%%%%%%%%%%%%%%%%%%%%%%%%%%
%%%The black board font
%%%%%%%%%%%%%%%%%%%%%%%%%%%
%\newcommand{\bK}{{\bf K}}
\newcommand{\bK}{{\Bbbk}}
%\newcommand{\bK}{{\mathbb k}}
\newcommand{\AAA}{{\mathbb A}}
\newcommand{\BB}{{\mathbb B}}
\newcommand{\CC}{{\mathbb C}}
\newcommand{\DD}{{\mathbb D}}
\newcommand{\EE}{{\mathbb E}}
\newcommand{\FF}{{\mathbb F}}
\newcommand{\GG}{{\mathbb G}}
%\newcommand{\HH}{{\mathbb H}}
\newcommand{\II}{{\mathbb I}}
\newcommand{\JJ}{{\mathbb J}}
\newcommand{\KK}{{\mathbb K}}
%\newcommand{\LL}{{\mathbb L}}
\newcommand{\MM}{{\mathbb M}}
\newcommand{\NN}{{\mathbb N}}
%\newcommand{{\mathbb O}}
\newcommand{\PP}{{\mathbb P}}
\newcommand{\QQ}{{\mathbb Q}}
\newcommand{\RR}{{\mathbb R}}
\newcommand{\SSS}{{\mathbb S}}
\newcommand{\TT}{{\mathbb T}}
\newcommand{\UU}{{\mathbb U}}
\newcommand{\VV}{{\mathbb V}}
\newcommand{\WW}{{\mathbb W}}
\newcommand{\XX}{{\mathbb X}}
\newcommand{\YY}{{\mathbb Y}}
\newcommand{\ZZ}{{\mathbb Z}}


\newcommand{\dontshow}[1]{}

\newcommand{\smcdot}{{\textup{$\cdot$}}}

%%%%%%%%%%%%%%%%%%%%%%%%%%%%%%%%%%%%%%%%%%%%%%%%%%%%%%%%%%%%%%%%%%%%%%%
%
% Begin: surfer stuff
% (Some environments for the text pages for the surfer software)
%
%%%%%%%%%%%%%%%%%%%%%%%%%%%%%%%%%%%%%%%%%%%%%%%%%%%%%%%%%%%%%%%%%%%%%%%

\newenvironment{surferGalIntroPage}{%
  % start a new page, if necessary
  \newpage
 % \twocolumn
  % it seems to me that the papersize has to be specified here and further up, too
  \special{papersize=\galIntroHeight,\galIntroWidth}
  % the page background color:
  \pagecolor{textHgColor}
  % the page font color:
  \color{textFgColor}
  % no page headers or footers:
  \thispagestyle{empty}
  %%%%% 
  \begin{flushleft}}%
  {\end{flushleft}}


\newenvironment{surferPage}{%%%%% 
  % start a new page, if necessary
  \newpage
  % it seems to me that the papersize has to be specified here and further up, too
  \special{papersize=\explHeight,\explWidth}
  % the page background color:
  \pagecolor{textHgColor}
  % the page font color:
  \color{textFgColor}
  % no page headers or footers:
  \thispagestyle{empty}
  %%%%% 
  \begin{flushleft}}%
{\end{flushleft}}

\newenvironment{surferTitle}{\bf}{

}

\newenvironment{galTitle}{\bf}{

}

\newenvironment{surferText}{}{}

%%%%%%%%%%%%%%%%%%%%%%%%%%%%%%%%%%%%%%%%%%%%%%%%%%%%%%%%%%%%%%%%%%%%%%%
%
% End of: surfer stuff
%
%%%%%%%%%%%%%%%%%%%%%%%%%%%%%%%%%%%%%%%%%%%%%%%%%%%%%%%%%%%%%%%%%%%%%%%


%%%%%%%%%%%%%%%%%%%%%%%%%%%%%%%%%%%%%%%%%%%%%%%%%%%%%%%%%%%%%%%%%%%%%%%
%
% The document starts here:
%
\begin{document}
\footnotesize
\begin{surferGalIntroPage}

\begin{surferTitle}Flater med verdensrekord\end{surferTitle}
  % 
  \begin{surferText}
    	
	En flate kalles glatt dersom den ikke har noen toppunkter (slike punkter kalles singulariteter). 
	Eksempler på glatte flater er en kule eller en torus, se de to første bildene under. 
	Velger man seg en tilfeldig flate, er den nesten alltid glatt.
 \begin{center}
      \vspace{-0.2cm}
      \begin{tabular}{@{}c@{}c@{}c@{\quad}c@{}c@{}c@{}c@{}}
        \begin{tabular}{@{}c@{}}
          smooth:
        \end{tabular}
        &
        \begin{tabular}{@{}c@{}}
          \includegraphics[width=1.1cm]{./../../common/images/kugel}
        \end{tabular}
        &
        \begin{tabular}{@{}c@{}}
          \includegraphics[width=1.1cm]{./../../common/images/torus}
        \end{tabular}
        &
        \begin{tabular}{@{}c@{}}
          many\\
          singularities:
        \end{tabular}
        &
        \begin{tabular}{c@{}@{}}
          \includegraphics[width=1.1cm]{./../../common/images/kummer}
        \end{tabular}
        &
        \begin{tabular}{c@{}@{}}
          \includegraphics[width=1.1cm]{./../../common/images/togliatti}
        \end{tabular}
        &
        \begin{tabular}{c@{}@{}}
          \includegraphics[width=1.1cm]{./../../common/images/barth_sextic}
        \end{tabular}
      \end{tabular}
    \end{center}
    \vspace{-0.2cm}
	
	Bare spesielle flater har singulariteter. Det gjør singularitetene til flatenes mest 
  interessante punkter. Flatene i programmet SURFER er definert av polynomer. Den høyeste 
  eksponenten til et polynom kalles graden til polynomet. Matematikere spør seg hvor mange
  singulariteter en flate av en bestemt grad kan ha. Vi kaller dette antallet for $\mu(d)$. 
  
  Det viser seg at antallet $\mu(d)$ er veldig vanskelig å beregne. Siden 1800-tallet har $\mu(d)$ 
  vært kjent for $d=1,2,3,4$, men ikke før i 1980 fant man det for $d=5$, og i 1996 for $d=6$.
  For $d\ge 7$, $\mu(d)$ fortsatt ukjent. 

  Enhver ny verdensrekord for $\mu(d)$ er derfor viktig. Det ser ut til å ta tid å finne en komplett løsning av $\mu(d)$ for en 
  vilkårlig $d$.\\ Her følger noen kjente resultater:
  
    
   \begin{center}
      \begin{tabular}{r|cccccccc|c}
        $d$ & $1$ & $2$ & $3$ & $4$ & $5$ & $6$ & $7$ & $8$ & $d$\\
        \hline
        \hline
        \rule{0pt}{1.2em}$\mu(d)\ge$ & $0$ & $1$ & $4$ & $16$ & $31$ & $65$ &
        $99$ & $168$ & 
        $\approx \frac{5}{12}d^3$\\[0.3em]
        \hline
        \rule{0pt}{1.2em}$\mu(d)\le$ & $0$ & $1$ & $4$ & $16$ & $31$ & $65$ &
        $104$ & $174$ & $\approx \frac{4}{9}d^3$
      \end{tabular}
    \end{center}
\end{surferText}
  \end{surferGalIntroPage}
\end{document}
%
% end of the document.
%
%%%%%%%%%%%%%%%%%%%%%%%%%%%%%%%%%%%%%%%%%%%%%%%%%%%%%%%%%%%%%%%%%%%%%%%
