\documentclass[no]{./../../common/SurferDesc}%%%%%%%%%%%%%%%%%%%%%%%%%%%%%%%%%%%%%%%%%%%%%%%%%%%%%%%%%%%%%%%%%%%%%%%
%
% The document starts here:
%
\begin{document}
\footnotesize
% Weltrekordfl�chen

%%% 1.Tafel

%%%%%%%%%%%%%%%%%%%%%%%%%%%%%


\begin{surferPage}

  \begin{surferTitle}Barths flate av sjette grad\end{surferTitle}  \\
Denne flaten av sjette grad ble konstruert av Wolf Barth i 1996.
    
    Barths flate har til sammen $65$ singulariteter. 
%   En flate av sjette grad kan ikke ha flere, noe som Jaffe og Ruberman viste kort tid etter at Barth konstruerte denne flaten. Barths verdensrekord er usl�elig!


Barths konstruksjon var en stor overraskelse fordi man lenge hadde trodd at flater av sjette grad bare kunne ha $64$ singulariteter.

En p�fallende egenskap ved flaten, er at den har symmetrien til et ikosaeder, som er en flate best�ende av 20 likesidede trekanter. Figuren viser et ikosaeder og dets symmetriplan:    
    % 
  \begin{center}
      \vspace*{-0.1cm}
      \begin{tabular}{@{}c@{\ \ }c@{\,}c@{}}
        \begin{tabular}{@{}c}
          \includegraphics[width=1.4cm]{./../../common/images/icosah}
        \end{tabular}
        &
        \begin{tabular}{@{}c}
          \includegraphics[width=1.4cm]{./../../common/images/barth_sextic_planes}
        \end{tabular}
        &
        \begin{tabular}{c@{}}
          \includegraphics[width=1.4cm]{./../../common/images/barth_sextic_and_planes}
        \end{tabular}
      \end{tabular}
    \end{center}
    \vspace*{-0.1cm}

   Barths flate av sjette grad tilfredsstiller ligningen  
    $P_6 - \alpha K^2=0,$ hvor $P_6$
    er de seks symmetriplanene, $K=x^2+y^2+z^2-1$ er enhetskula (kule med radius 1) og  
    $\alpha=\frac{1}{4}(2+\sqrt{5})$.  


  \begin{surferText}
     \end{surferText}
\end{surferPage}

%%%%%%%%%%%%%%%%%%%%%%%%%%%%%%%%



\end{document}
%
% end of the document.
%
%%%%%%%%%%%%%%%%%%%%%%%%%%%%%%%%%%%%%%%%%%%%%%%%%%%%%%%%%%%%%%%%%%%%%%%
