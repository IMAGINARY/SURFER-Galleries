\begin{surferPage}[Endra\ss-Octic]{La ottica di Endra\ss}
     Nel 1995, Stephan Endra\ss\ costru\`i questa superficie di grado $8$ (ottica) come risultato principale della sua tesi presso l'universit\`a di Erlangen.
    In tutto, la superficie ha $168$ singolarit\`a il che \`e tuttora il record mondiale. 
  
     Grazie a un risultato generale dovuto a Varchenko si sa che un'ottica non pu\`o avere pi\`u di $174$ punti singolari.
    Perci\`o: $168 \le \mu(8) \le 174$. 
    Il numero esatto \`e ancora ignoto.

     Non fu facile trovare la superficie: Endra\ss\ dovette cercarla in una famiglia di ottiche di dimensione $5$, nella quale la generica superficie ha solo $112$ singolarit\`a.

    Nella figura interattiva \`e evidente la simmetria della costruzione: 
    Oltre alla simmetria di un ottagono regolare la superficie \`e simmetrica rispetto al piano $xy$.

    Senza l'uso di queste simmetrie, lo spazio in cui ricercare la superficie sarebbe stato di dimensione maggiore.
\end{surferPage}
