\begin{surferPage}[Togliatti-Quintic]{Una quintica di Togliatti}
    Eugenio Giuseppe Togliatti insegn\`o all'Universit\`a di Genova e dimostr\`o nel 1937 l'esistenza di una superficie di grado $5$ (quintica) con esattamente $31$ singolarit\`a --- a quell'epoca un primato mondiale.

    Nel 1980 Arnaud Beauville us\`o un'interessante relazione con la teoria dei codici per dimostrare che non esiste una quintica con un numero maggiore di singolarit\`a. 
    Questo significa che il primato di Togliatti non pu\`o essere superato!

    Dato che non c'\`e un solido platonico di cui si possano usare i piani di simmetria per costruire una superficie di grado $5$ in modo simile alla  quartica di Kummer o alla sestica di Barth, la quintica con $31$ singolarit\`a ha meno simmetrie, e precisamente le simmetrie del pentagono piano.

 L'equazione che usiamo qui fu trovata da Wolf Barth (1990); usiamo questa perch\'e quella originale di Togliatti non \`e facile da visualizzare.
\end{surferPage}
