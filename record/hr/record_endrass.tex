\begin{surferPage}[Endraß-Octic]{The Endraß Octic}
     %In 1995, Stephan Endraß constructed this surface of degree $8$ (octic) as the
		Ovu plohu stupnja $8$ (oktiku) konstruirao je Stephan Endraß 1995. u svojoj disertaciji
    %main result of his dissertation at Erlangen University.
		na Sveu\v{c}ili\v{s}tu u Erlangenu.
    %Altogether, it has $168$ singularities which is still the current world
    %record. 
		Ploha ima ukupno $168$ singulariteta \v{s}to je jo\v{s} uvijek svjetski rekord.
  
     %Via a general result by Varchenko one knows that an octic cannot have more
		Varchenko je dokazao da oktika ne mo\v{z}e imati vi\v{s}e 
    %than $174$ singular points.
		od $174$ singularnih to\v{c}aka.
    %Thus: $168 \le \mu(8) \le 174$. 
		Dakle: $168 \le \mu(8) \le 174$.
    %The exact number is not known.
		Nije poznat to\v{c}an broj.

    % Finding the surface was not easy: Endraß had to search for it in a
		Nije bilo lako na\'{c}i ovu plohu: Endraß ju je morao tra\v{z}iti u 
    %$5$-dimensional family of octics, where the general member of the family
		$5$-dimenzionalnoj familiji oktika, gdje op\'{c}i \v{c}lan familije ima 
    %only has $112$ singularities.
		samo $112$ singulariteta.

    %In the interactive picture the symmetry of the construction is apparent: 
		Simetrija ove konstrukcije je o\v{c}ita na interaktivnoj slici:
    %In addition to the symmetry of a regular octagon the surface is symmetric
		osim simetrije pravilnog osmerokuta, ploha je simetri\v{c}na s obzirom
    %with respect to the $xy$ plane.
		na $xy$ ravninu.

    %Without using such symmetries the search space would have been of even
		Bez kori\v{s}tenja takvih simetrija, prostor pretra\v{z}ivanja bi bio
    %higher dimension.
		jo\v{s} ve\'{c}e dimenzije.
\end{surferPage}
