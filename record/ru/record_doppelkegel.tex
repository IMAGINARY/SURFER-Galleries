\begin{surferPage}{Двойной конус}
Как говорилось во введении к этой галерее, поверхность называется несингулярной, если она, образно говоря, не имеет «заострений», называемых сингулярностями, например, сфера, тор (изображение 1 или 2 слева): 
    \begin{center}
      \begin{tabular}{@{}c@{}c@{}c@{}c@{}}
        \begin{tabular}{@{}c}
          \includegraphics[width=1.4cm]{kugel}
        \end{tabular}
        &
        \begin{tabular}{@{}c}
          \includegraphics[width=1.4cm]{torus}
        \end{tabular}
        &
        \begin{tabular}{c@{}}
          \includegraphics[width=1.4cm]{kegel}
        \end{tabular}
      \end{tabular}
    \end{center}
Двойной конус (правое изображение) – самая простая сингулярность; она – единственная, которую можно описать уже уравнением второй степени: 
    \[x^2+y^2-z^2=0.\]
    Если его немного изменить, приписав с правой стороны уравнения величину $a\neq 0$, то он превращается в зависимости от знака §$a$ в одно- или двуполостный гиперболоид:
    \begin{center}
      \begin{tabular}{@{}c@{\ }c@{\ }c@{\ }c@{\ }c@{}}
        \begin{tabular}{@{}c@{}}
          \includegraphics[width=1.2cm]{A1pm_2}
        \end{tabular}
        &
        $\leftarrow$
        &
        \begin{tabular}{@{}c@{}}
          \includegraphics[width=1.2cm]{A1pm_1} 
        \end{tabular}
        &
        $\rightarrow$
        &
        \begin{tabular}{@{}c@{}}
          \includegraphics[width=1.2cm]{A1pm_0}
        \end{tabular}
      \end{tabular}
    \end{center}
    Поверхность второй степени имеет максимум одну сингулярность, т.е.\ $\mu(2)=1$.
\end{surferPage}
