\begin{surferPage}[Octica Endraß]{La Óctica de Endra}
    En 1995, Stephan Endra construyó una superficie de grado $8$ (óctica),
    como la conclusión principal de su disertación en la Universidad de Erlangen.
    En total, tiene $168$ singularidades, obteniendo así el actual récord mundial.
    
    A través de un resultado general de Varchenko, uno puede saber que una óctica
    no puede tener más de $174$ puntos singulares. Luego, $168 \le \mu(8) \le 174$. 
    
    Aún así, el número exacto es desconocido.

    Encontrar la superficie no fue para nada fácil: Endra tuvo que investigar en una
    familia de ócticas en el espacio de $5$ dimensiones, donde en general los
    miembros de esta familia sólo tenían $112$ singularidades.

    La simetría de la construcción es fácil de notar. Además de la simetría de
    octágono regular, la superficie es simétrica con respecto al plano $xy$.

    Lo notable es que si esas simetrías no existieran, se tendría que haber buscado en
    espacios con más dimensiones.
\end{surferPage}
