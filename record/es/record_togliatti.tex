\begin{surferPage}[Quíntica de Togliatti]{La Quíntica de Togliatti}
    Eugenio G. Togliatti construyó en 1937 una superficie de grado $5$ (quíntica)
    con $31$ singularidades, un récord en aquél momento.

    En 1980 Arneau Beauville usó una interesante relación con la teoría de códigos 
    para mostrar que no pueden existir quínticas con más de $31$ puntos singulares,
    de modo que $\mu(7)$=$31$. ¡El resultado de Togliatti es inmejorable!
    
    Como no hay ningún sólido platónico cuyas caras se puedan usar para construir una 
    superficie de grado $5$, como ocurre con la cuártica de Kummer o la séxtica de Barth, 
    la quíntica con $31$ singularidades tiene menos simetrías (sólo tiene las de un pentágono regular).

 La ecuación que usamos fue hallada por Wolf Barth (1990), ya que la original de Togliatti no es fácil de vizualizar.
\end{surferPage}
