\documentclass[es]{../../common/SurferDesc}%%%%%%%%%%%%%%%%%%%%%%%%%%%%%%%%%%%%%%%%%%%%%%%%%%%%%%%%%%%%%%%%%%%%%%%
%
% The document starts here:
%
\begin{document}
\footnotesize
% Weltrekordflchen

%%% 1.Tafel

%%%%%%%%%%%%%%%%%%%%%%%%%%%%%


\begin{surferPage}

  \begin{surferTitle}La Séxtica de Barth\end{surferTitle}  \\
    Esta superficie de grado $6$ (séxtica) fue construida por Wolf Barth en 1996.
    Tiene un total de $65$ singularidades, siendo el máximo posible para una séxtica,
    un hecho demostrado poco tiempo después por Jaffe y Ruberman. Es decir,
    ¡el récord mundial de Barth es insuperable!

    Esta construcción de Barth fue una gran sorpresa, ya que por mucho tiempo
    se pensó que las superficies de grado $6$ sólo podían tener hasta $64$ singularidades.
    
    Una caracterísitica particular de la construcción es su simetría icosaedral;
    la figura muestra un icosaedro y sus planos simétricos:
  \begin{center}
      \vspace*{-0.1cm}
      \begin{tabular}{@{}c@{\ \ }c@{\,}c@{}}
        \begin{tabular}{@{}c}
          \includegraphics[width=1.4cm]{../../common/images/icosah}
        \end{tabular}
        &
        \begin{tabular}{@{}c}
          \includegraphics[width=1.4cm]{../../common/images/barth_sextic_planes}
        \end{tabular}
        &
        \begin{tabular}{c@{}}
          \includegraphics[width=1.4cm]{../../common/images/barth_sextic_and_planes}
        \end{tabular}
      \end{tabular}
    \end{center}
    \vspace*{-0.1cm}

    La séxtica de Barth satisface la ecuación
    $P_6 - \alpha K^2=0,$ donde $P_6$
    representa los seis planos simétricos,
    $K=x^2+y^2+z^2-1$ es la esfera unitaria y
    $\alpha=\frac{1}{4}(2+\sqrt{5})$.  


  \begin{surferText}
     \end{surferText}
\end{surferPage}

%%%%%%%%%%%%%%%%%%%%%%%%%%%%%%%%



\end{document}
%
% end of the document.
%
%%%%%%%%%%%%%%%%%%%%%%%%%%%%%%%%%%%%%%%%%%%%%%%%%%%%%%%%%%%%%%%%%%%%%%%
