\begin{surferPage}[Endraß]{Het achtstegraadsoppervlak van Endraß}
     In 1995 construeerde Stephan Endraß dit oppervlak van de graad $8$ als voornaamste resultaat van zijn doctoraatsthesis aan Erlangen University.
    Dit oppervlak heeft $168$ singulariteiten, wat nog steeds het huidige wereldrecord is. 
  
     Dankzij een algemeen resultaat van Varchenko weten we dat een achtstegraadsoppervlak niet meer dan $174$ singuliere punten kan hebben. Dit betekent dus: $168 \leqslant \mu(8) \leqslant 174$. 
    De exacte waarde is (nog) niet gekend.

     Het was niet gemakkelijk om dit oppervlak te vinden: Endraß moest het zoeken in een vijfdimensionele verzameling van achtstegraadsoppervlakken. Deze oppervlakken bevatten in het algemeen echter slechts 112 singuliere punten.

    Op de interactieve afbeelding wordt de symmetrie van deze constructie duidelijk: 
    naast de symmetrie\"en van een regelmatige achthoek is dit oppervlak ook symmetrisch ten opzichte van het $xy$-vlak.

    Als deze grote symmetrie niet had kunnen gebruikt worden, zou de zoekruimte nog groter geweest zijn!
\end{surferPage}
