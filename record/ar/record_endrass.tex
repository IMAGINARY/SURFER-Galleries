\begin{surferPage}[Endraß-Octic]{مثمن إندراس}
    في العام 1995، قام ستيفان إندراس
    \textenglish{(Stephan Endraß)} ببناء هذا المنحنى من الدرجة $8$ (مثمن) وجعله النتيجة الرئيسية لأطروحته في جامعة إرلنغن.
     لهذا المنحنى $168$ نقطة متفردة وهذا لا يزال إلى اليوم الرقم القياسي العالمي.
  
     بفضل نتيجة عامة أثبتها فارشنكو 
     \textenglish{(Varchenko)}
      من المعروف أنه لا يمكن لمنحنى مثمن أن يملك أكثر من $174$ نقطة متفردة.
    وبالتالي: $168 \le \mu(8) \le 174$.
    يبقى هذا العدد غير معروف بالتحديد.

     لم يكن من السهل إيجاد المنحنى: فلقد بحث عنه إندراس في عائلة مثمنات خماسية الأبعاد وحيث يملك أفراد هذ العائلة على وجه العموم $112$ نقطة متفردة فقط.

    في الصورة التفاعلية للشكل، يبدو التناظر بوضوح: 
    بالإضافة إلى التناظر في ثماني أضلاع منتظم، فإن السطح متناظر بالنسبة لمستوى $xy$.

     لولا اللجوء إلى هذه التناظرات، لكان من الضروري أن يتم البحث في فضاء من بعد أكبر. 
\end{surferPage}
