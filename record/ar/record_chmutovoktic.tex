\begin{surferPage}[Chmutov-Octic]{مثمن شموتوف}
     ميزة ملفتة للنظر في مثمن شموتوف $\text{Chm}_{d}, \ d=8,$ وهي تناظره.
     هذا يمكن رؤيته أيضاً من خلال فحص المعادلة:
    \[\text{Chm}_{d}\colon T_d(x) + T_d(y) + T_d(z) + 1 = 0,\]
    حيث $T_d$ هو ما نسميه متعدد حدود شيبيشيف (الصورة على اليسار).  
     على اليمين، يمكن رؤية المنحنى  $T_8(x)+T_8(y)=0$:
    
     \begin{center}
      \begin{tabular}{c@{\quad}c}
        \begin{tabular}{c}
          \includegraphics[height=1.75cm]{./../../common/images/Tcheb_008.pdf}
        \end{tabular}    
        &
        \begin{tabular}{c}
          \includegraphics[height=1.75cm]{./../../common/images/Tcheb_2d_008.pdf}
        \end{tabular}    
      \end{tabular}
    \end{center}
    \vspace{-0.3cm}
    الإنتقال من هذه الصور إلى شكل المنحنى في الصورة التفاعلية مباشر نوعاً ما.


 هذه المعادلات نتجت عن أعمال شموتوف (S.V.\ Chmutov) في أوائل الثمانينيات.
   ولقد أوجدت آنذاك الرقم القياسي العالمي $\mu(d)$ للعدد الأقصى للمتفردات لمعظم الدرجات $d$. 
    في التسعينيات، حطم شموتوف رقمه القياسي وفي العام 2005، لائم كل من بريسك (S.~Breske) ولابس (O.~Labs) وفان ستراتن (D.~van~Straten) هذا البناء لمنح منحنيات حقيقية لها متفردات حقيقية فقط. 
\end{surferPage}