\begin{surferPage}[منحنى تولياتي المخمس]{منحنى تولياتي المخمس}
    برهن أوجينيو جيسيبي تولياتي %
    \textenglish{(Eugenio Giuseppe Togliatti)} %
   في 1937 وجود سطح من الدرجة $5$ (مخمس) له بالضبط $31$ متفرد--- رقم قياسي عالمي لوقتها.

    في 1980، استعمل ارنو بوفيل %
    \textenglish{(Arneau Beauville)} %
     علاقة مهمة مع نظرية الترميز من أجل برهنة عدم وجود منحنى مخمس يملك عدداً أكبر من المتفردات.
    هذا يعني أنه لن يكون من الممكن تحطيم رقم تولياتي القياسي!

   خلافاً لمنحنى كومر المربع ومنحنى بارت المسدس، فإن المنحنيات المخمسة لا يمكن بناؤها إنطلاقاً من مستويات تناظر مجسم أفلاطوني. ينتج عن ذلك أن المنحنى المخمس ذا $31$ متفرد لا يملك سوى تناظرات قليلة، وبالتحديد تناظرات خماسي الأضلاع.

    المعادلة التي نستعملها هنا تم وضعها من قبل وولف بارت %

    \textenglish{(Wolf Barth 1990)}%
    ؛ نستعمل هذه المعادلة لأن عرض المعادلة الأصلية لمنحنى تولياتي ليس سهلاً.
\end{surferPage}
