\begin{surferPage}[Togliatti'nin Beşgili]{Togliatti'nin Beşgili}
    Eugenio Giuseppe Togliatti 1937'de derecesi $5$ olan  yüzeylerden (beşgil)  
tam $31$ tekilliği olan bir tanesinin varlığını keşfetti --- bu o gün için bir dünya rekoruydu.

1980'de Arneau Beauville daha fazla tekilliği olan bir beşgil olmadığını, şifreleme kuramıyla ilginç bir bağlantısını kurarak gösterdi. Demek oluyor ki Togliatti'nin dünya rekoru artık geliştirilemez!

Kummer'in Dörtgili ya da Barth'ın Altıgilinde Platon cisimlerinin simetrileri kullanılıyor.
Derecesi $5$ olan bir yüzey için simetri düzlemleri kullanılabilecek bir Platon cismi var olmadığından,
 $31$ tekillikli beşgilin daha az simetrisi var: yalnızca düzgün beşgenin düzlemdeki simetrilerine sahip.

Biz burada  Wolf Barth'ın  1990'da bulduğu yüzey denklemini kullanıyoruz çünkü Togliatti'nin 1937'de bulduğu yüzeyi görselleştirmek zor.
\end{surferPage}
