\begin{surferPage}[Endrass'ın Sekizgili]{Endrass'ın Sekizgili}
     1995'de Stephan Endrass,  Erlangen Üniversitesi'ndeki doktora tezinin ana sonucu olarak derecesi $8$ olan bu yüzeyi (sekizgil) inşa etti.
Yüzeyin toplam   $168$ tekilliği var ve bu hala bir dünya rekoru!

Varchenko'nun genel bir sonucu aracılığıyla  bir sekizgilin $174$'ten çok tekilliği olamayacağını biliyoruz. Böylece $168 \le \mu(8) \le 174$ elde ediliyor. $\mu(8)$ sayısının tam kaç olacağı bilinmiyor.

Bu yüzeyi bulmak hiç kolay olmadı: Endrass, bu yüzeyi  $5$ boyutlu bir uzay oluşturan bir sekizgil  ailesi içinde aramak zorunda kaldı. Bu ailedeki genel bir yüzeyin tekillik sayısı $112$ idi.

İnşanın simetrisi resimde aşikar: bir düzgün  sekizgen simetrisinin yanı sıra yüzey $xy$ düzlemine göre de simetrik.

Bu simetriler kullanılmasaydı, o sekizgil uzayı daha da yüksek bir boyuta sahip olacaktı.
\end{surferPage}
