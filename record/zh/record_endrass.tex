\begin{surferPage}{恩德拉8次曲面}
1995年,史蒂芬恩德拉在他的学位论文(埃朗根大学)中构造了这个8次曲面。这个曲面共有168个奇异点,这仍然是当今的世界纪录。瓦尔琴科实际上证明了一个一般的结果:任何8次曲面的奇异点个数不可能有多于168。所以$168 \le \mu(8) \le 174$。而$\mu(8)$的具体值仍然是个未知数。

找到这样的曲面并非易事,恩德拉实际上是在一族5维8次曲面内不断的搜索而得到的。而这一族曲面一般情况下只有112个奇异点。
在图中可以清楚地看到这个曲面的对称性:除了普通的8边形的对称外,额外的有一个关于$xy$ 平面的对称。如果没有这个额外的对称,搜索空间将会有更高的维数。
\end{surferPage}

