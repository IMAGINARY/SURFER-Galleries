\begin{surferPage}[Ендрасова површ]{Ендрасова површ осмог степена}
    Као главни резултат своје докторске дисертације на Ерланген Универзитету, 
	Стефан Ендрас је 1995. године конструисао површ осмог степена.
    Површ има укупно $168$ сингуларитета, што је тренутни светски рекорд. 
  
    Користећи општи резултат Варченка зна се да површ осмог степена не може имати 
    више од  $174$ сингуларитета.
    Значи: $168 \le \mu(8) \le 174$. 
    али се тачан број не зна.

    Није било лако пронаћи овакву површ: Ендрас је за њом трагао у петодимензионој 
    породици површи осмог степена у којој само општи члан породице има 
    $112$ сингуларитета.

    На интерактивној слици је јасно уочљива симетрија конструкције: 
    уз симетрију правилног осмоугла, површ је симетрична у односу на $xy$ раван.

    Без коришћења оваквих симетрија простор за претрагу би био још више димензије.
\end{surferPage}
