\begin{surferPage}[Тољатијева површ]{Тољатијева површ петог реда}
    Еугенио Ђузепе Тољати је доказао 1937. године да постоји површ петог степена са тачно 
	31 сингуларитетом – тада је то био светски рекорд.


    Арно Бовиј је 1980. године искористио занимљиву везу са теоријом кодирања да би 
	показао непостојање површи петог степена са више сингуларитета. 
    То значи да се Тољатијев светски рекорд не може побољшати!

    Како не постоје платонска тела чије равни симетрија могу да се искористе за 
	конструисање површи степена 5 сличних Кумеровој површи четвртог степена или 
	Бартовој шестог, површ петог степена са 31 сингуларитетом има мањи број симетрија, 
	односно само оне које има правилни петоугао.


 Једначину коју овде користимо је открио Волф Барт (1990); њу користимо јер 
 Тољатијева оригинална површ није лака за визуелизацију.
\end{surferPage}
