\begin{surferPage}[Endraß-Octic]{The Endraß Octic}
     In 1995, Stephan Endraß constructed this surface of degree $8$ (octic) as the 
    main result of his dissertation at Erlangen University.
    Altogether, it has $168$ singularities which is still the current world
    record. 
  
     Via a general result by Varchenko one knows that an octic cannot have more
    than $174$ singular points.
    Thus: $168 \le \mu(8) \le 174$. 
    The exact number is not known.

     Finding the surface was not easy: Endraß had to search for it in a
    $5$-dimensional family of octics, where the general member of the family
    only has $112$ singularities.

    In the interactive picture the symmetry of the construction is apparent: 
    In addition to the symmetry of a regular octagon the surface is symmetric
    with respect to the $xy$ plane.

    Without using such symmetries the search space would have been of even
    higher dimension.
\end{surferPage}
