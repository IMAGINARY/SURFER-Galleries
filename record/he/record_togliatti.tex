\begin{surferPage}[Togliatti-Quintic]{משטח ממעלה חמישית של טוליאטי}
    אֶאוּגֶ'ניוֹ ג'יוּזֶפֶּה טוֹליאַטי (Eugenio Giuseppe Togliatti) הוכיח בשנת 1937 את קיומו של משטח ממעלה חמישית עם $31$ נקודות סינגולריות בדיוק – שיא עולמי בזמנו.


    בשנת 1980, היה זה ארנוֹ בּוֹביל (Arneau Beauville) אשר השתמש בקשר מעניין לתיאוריית
    הקידוד על-מנת להוכיח את אי-קיומו של משטח ממעלה חמישית עם מספר גדול יותר של
    נקודות סינגולריות. 
    פירוש הדבר שלא ניתן לשבור את השיא העולמי של טוליאטי!

    היות שאין גוף אפלטוני שניתן להשתמש במישורי הסימטריה שלו על-מנת
    לבנות משטח ממעלה חמישית הדומה למשטח ממעלה רביעית של קומר (Kummer)
    או למשטח ממעלה שישית של בארת' (Barth), הרי שלמשטח מהמעלה החמישית עם $31$ נקודות הסינגולריות יש פחות סימטריות,
    כלומר סימטריות של מישור המחומש.


 המשוואה המופיעה כאן נמצאה על-ידי וולף בארת' (Wolf Barth) בשנת 1990; אנו משתמשים בה כאן
    משום שהמשטח המקורי של טוליאטי קשה יותר להמחשה חזותית.
\end{surferPage}
