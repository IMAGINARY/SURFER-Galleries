\begin{surferPage}[Togliatti-Quintik]{Eine Togliatti-Quintik}
    Eugenio Giuseppe Togliatti bewies bereits im Jahr 1937,
    dass es eine Fläche vom Grad $5$ (Quintik) mit genau $31$ Singularitäten
    gibt --- damals Weltrekord!  

    1980 gelang es Arneau Beauville durch eine interessante Beziehung
    zur Kodierungstheorie zu zeigen, dass eine Quintik nicht
    mehr Singularitäten besitzen kann. 
    Dies heißt also, dass Togliattis Weltrekord niemals 
    mehr verbessert werden kann!

    Da es leider keinen Platonischen Körper gibt, dessen Symmetrieebenen man
    ausnutzen könnte, um eine Fläche vom Grad $5$ in Anlehnung an Kummers
    Quartik und Barths Sextik zu konstruieren, besitzt die abgebildete Quintik
    mit $31$ Singularitäten weniger Symmetrie, nämlich die Symmetrie eines
    ebenen Fünfecks.

    Auch diese Gleichung hat Wolf Barth gefunden (1990); Togliattis Fläche 
    von 1937 ist nämlich schwer zu visualisieren.
\end{surferPage}
