\begin{surferPage}[Endraß-Oktik]{Die Endraß-Oktik}
  Stephan Endraß konstruierte diese Fläche vom Grad $8$ --- daher Oktik --- 1995
    im Rahmen seiner Dissertation in Erlangen. 
    Sie hat insgesamt $168$ Singularitäten: aktueller Weltrekord! 

    Mit Hilfe eines allgemeinen Resultats von Varchenko weiß man außerdem,
    dass eine Oktik nicht mehr als $174$ singuläre Punkte haben kann.
    Dies heißt also: $168 \le \mu(8) \le 174$. 
    Die genaue Zahl ist noch nicht bekannt.

    Die Fläche zu finden war nicht einfach; immerhin musste Endraß sie in einer
    $5$-dimensionalen Familie von Oktiken suchen, die im Allgemeinen nur $112$
    Singularitäten haben.

    Man sieht im Bild recht gut die Symmetrie der Oktik:
    Zusätzlich zur Symmetrie eines regelmäßigen $8$-Ecks ist die Fläche
    spiegelsymmetrisch zur $xy$-Ebene. 

    Ohne die Ausnutzung von solch großen Symmetrien wäre der Suchraum nicht
    nur $5$-dimensional, sondern noch wesentlich größer gewesen!
\end{surferPage}
