\documentclass[de]{./../../common/SurferDesc}%%%%%%%%%%%%%%%%%%%%%%%%%%%%%%%%%%%%%%%%%%%%%%%%%%%%%%%%%%%%%%%%%%%%%%%
%
% The document starts here:
%
\begin{document}
\footnotesize
% Weltrekordfl�chen

%%% 1.Tafel

%%%%%%%%%%%%%%%%%%%%%%%%%%%%%
%%%%%%%%%%%%%%%%%%%%%%%%%%%%%%


\begin{surferPage}
  \begin{surferTitle}Die Endra�-Oktik\end{surferTitle} \\
  Stephan Endra� konstruierte diese Fl�che vom Grad $8$ --- daher Oktik --- 1995
    im Rahmen seiner Dissertation in Erlangen. 
    Sie hat insgesamt $168$ Singularit�ten: aktueller Weltrekord! 

    Mit Hilfe eines allgemeinen Resultats von Varchenko wei� man au�erdem,
    dass eine Oktik nicht mehr als $174$ singul�re Punkte haben kann.
    Dies hei�t also: $168 \le \mu(8) \le 174$. 
    Die genaue Zahl ist noch nicht bekannt.

    Die Fl�che zu finden war nicht einfach; immerhin musste Endra� sie in einer
    $5$-dimensionalen Familie von Oktiken suchen, die im Allgemeinen nur $112$
    Singularit�ten haben.

    Man sieht im Bild recht gut die Symmetrie der Oktik:
    Zus�tzlich zur Symmetrie eines regelm��igen $8$-Ecks ist die Fl�che
    spiegelsymmetrisch zur $xy$-Ebene. 

    Ohne die Ausnutzung von solch gro�en Symmetrien w�re der Suchraum nicht
    nur $5$-dimensional, sondern noch wesentlich gr��er gewesen!
  \begin{surferText}
     \end{surferText}
\end{surferPage}
%%%%%%%%%%%%%%%%%%%%%%%%%%%%%




\end{document}
%
% end of the document.
%
%%%%%%%%%%%%%%%%%%%%%%%%%%%%%%%%%%%%%%%%%%%%%%%%%%%%%%%%%%%%%%%%%%%%%%%
