\begin{surferPage}{恩德拉8次曲面}
1995年,史蒂芬·恩德拉(Stephan Endraß)在他的學位論文(愛爾朗根大學)中構造了這個$8$次曲面。這個曲面共有$168$個奇異點,這仍然是當今的世界紀錄。瓦爾琴科實際上證明了一個一般的結果:任何$8$次曲面的奇異點個數不可能有多於$168$。所以$168 \le \mu(8) \le 174$。而$\mu(8)$的具體值仍然是個未知數。

找到這樣的曲面並非易事,恩德拉實際上是在一族$5$維$8$次曲面內不斷的搜索而得到的。而這一族曲面一般情況下只有$112$個奇異點。
在圖中可以清楚地看到這個曲面的對稱性:除了普通的$8$邊形的對稱外,額外的有一個關於$xy$ 平面的對稱。如果沒有這個額外的對稱,搜索空間將會有更高的維數。
\end{surferPage}

