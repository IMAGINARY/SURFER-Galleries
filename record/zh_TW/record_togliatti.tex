\begin{surferPage}[陶里亞蒂五次曲面]{陶里亞蒂五次曲面}
1937年,歐亨尼奧·朱塞佩·陶里亞蒂(Eugenio Giuseppe Togliatti)證明了存在帶有31個奇異點的五次曲面——當時的世界記錄。
1980年,阿爾瑙·伯維爾(Arneau Beauville)利用編碼理論證明了五次曲面上的奇異點不可能更多。這意味著陶里亞蒂的結果是不可能被改進的。
不存在這樣的柏拉圖環體,它的對稱平面可以被用來構造五次曲面,正如庫默爾四次曲面或者巴斯六次曲面那樣。所以,帶有31個奇異點的五次曲面具有較少的對稱性,即只有平面五邊形的對稱性。
我們這裡用到的方程是沃爾夫·巴斯(Wolf Barth,1990)發現的。沒有採用陶里亞蒂當初的曲面,是因為他的曲面不容易被可視化。
\end{surferPage}
