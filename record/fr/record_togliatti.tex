\begin{surferPage}[Quintique de Togliatti]{Une Quintique de Togliatti}
    Eugenio Giuseppe Togliatti démontra en 1937 qu'il existe une surface de degré $5$ (quintique) ayant exactement $31$ singularités --- le record à cette époque.


    En 1980, Arneau Beauville utilisa un lien intéressant avec la théorie du codage
    pour montrer qu'il n'existe pas de quintique avec un plus grand nombre de
    singularités. 
    Cela signifie que le record de Togliatti ne peut être battu !

    Contrairement à la Quartique de Kummer ou à la Sextique de Barth, les surfaces de degré $5$ 
    ne peuvent pas être construites à partir des plans de symétrie d'un solide platonicien.
    Il en résulte que la quintique à $31$ singularités possède moins de symétries,
    précisément les symétries d'un pentagone plan.


 L'équation utilisée ici a été trouvée par Wolf Barth en 1990; nous utilisons
    celle-ci plutôt que l'originale de la surface de Togliatti qui n'est pas facile à visualiser.
\end{surferPage}
