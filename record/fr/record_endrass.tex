\begin{surferPage}[Octique d'Endraß]{L'Octique d'Endraß}
     En 1995, Stephan Endraß construisit cette surface de degré $8$ (octique) et en fit 
    le résultat principal de sa thèse de l'Université d'Erlangen.
    Au total, elle a $168$ singularités, soit l'actuel record. 
  
     Grâce à un résultat général de Varchenko, on sait qu'une octique ne peut avoir plus
    de $174$ points singuliers.
    Ainsi : $168 \le \mu(8) \le 174$. 
    On ne connaît cependant pas le nombre exact.

     La découverte de cette surface ne fut pas simple : Endraß dut la chercher
    dans une famille à $5$ dimensions d'octiques dont les membres ont de manière générale
    seulement $112$ singularités.

    Sur l'image interactive, la symétrie de la construction est évidente : 
    En plus de la symétrie d'un octogone régulier, la surface est symétrique
    par rapport au plan $xy$.

    Sans le recours à ces symétries, la famille de surfaces à explorer serait de dimension
    encore plus élevée.
\end{surferPage}
