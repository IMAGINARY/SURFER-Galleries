\begin{surferPage}[%
שטח מהמעלה השמינית של צ'מוטוב%
]{%
משטח ממעלה שמינית של אֶנדרַס%
}
     בשנת 1995, בנה סטפן אֶנדרַס
      \textenglish{(Stephan Endrass)}
       את המשטח מהמעלה השמינית המוצג כאן, כחלק
    עיקרי בעבודת התזה שלו באוניברסיטת ארלנגן.
    למשטח $168$ נקודות סינגולריות, והוא מחזיק עד היום בשיא
    העולם.

     תוצאה כללית של ורצ'נקו (Varchenko) מראה כי במשטח מהמעלה השמינית לא ייתכנו יותר
    מ-$174$ נקודות סינגולריות.
    לפיכך: $168 \le \mu(8) \le 174$.
    המספר המדויק אינו ידוע.

     גילוי המשטח לא היה משימה קלה: אנדרס היה צריך לחפש אותו
    בתוך משפחה של משטחים מהמעלה השמינית בעלי חמישה ממדים, שבה האיבר הכללי
    מכיל רק $112$ נקודות סינגולריות.

    בתצוגה האינטראקטיבית, הסימטריה שמאפיינת את המבנה ברורה לעין:
    בנוסף לסימטריה של מתומן משוכלל, המשטח סימטרי
    ביחס למישור $xy$.

    לולא נעשה שימוש בסימטריות אלה, מרחב החיפוש היה צריך להיות בעל
    מספר ממדים גדול אף יותר.
\end{surferPage}
