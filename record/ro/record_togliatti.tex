\begin{surferPage}[Cuintica-Togliatti]{Cuintica lui Togliatti}

   Eugenio Giuseppe Togliatti a demonstrat \^{i}n 1937 c\u{a} exist\u{a} o suprafa\c{t}\u{a} de grad $5$ (cuintic\u{a}) av\^{a}nd exact $31$ de singularit\u{a}\c{t}i 
    -- un record mondial la acel moment.
   
    \^{I}n 1980, Arnaud Beauville a folosit o leg\u{a}tur\u{a} interesant\u{a} cu teoria codurilor pentru a ar\u{a}ta inexisten\c{t}a unei cuintice cu mai multe singularit\u{a}\c{t}i.
    Aceasta \^{i}nseamn\u{a} c\u{a} recordul mondial al lui Togliatti nu poate fi niciodat\u{a} \^{i}mbun\u{a}t\u{a}\c{t}it!

      
    Deoarece nu exist\u{a} un solid platonic ale c\u{a}rui planuri de simetrie ar putea fi folosite 
    pentru a construi o suprafa\c{t}\u{a} de grad $5$ similar\u{a} cu cuartica lui Kummer sau
    sextica lui Barth, suprafa\c{t}a cuintic\u{a} cu $31$ de singularit\u{a}\c{t}i are mai pu\c{t}ine 
    simetrii, \c{s}i anume simetriile pentagonului planar.
    
    Ecua\c{t}ia pe care o folosim aici a fost g\u{a}sit\u{a} de Wolf Barth (1990). O folosim pe acesta
    deoarece suprafa\c{t}a original\u{a} a lui Togliatti nu este u\c{s}or de vizualizat.
    
\end{surferPage}
