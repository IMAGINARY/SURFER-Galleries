\begin{surferPage}[Octica-Endra\ss{}]{Octica lui Endra\ss{} }

    \^{I}n 1995, Stephan Endra\ss{} a construit aceast\u{a} suprafa\c{t}\u{a} de grad $8$ (Octic\u{a}),
    ca rezultat principal al tezei sale de doctorat la Universitatea Erlangen. Are $168$ de
    singularit\u{a}\c{t}i \^{i}n total, recordul mondial in acest moment.

   Datorit\u{a} unui rezultat general al lui Varchenko, \c{s}tim c\u{a} o octic\u{a} nu poate avea mai mult
   de $174$ de puncte singulare. A\c{s}adar $168 \le \mu(8) \le 174$. Numarul exact nu se cunoa\c{s}te \^{i}nc\u{a}.

    G\u{a}sirea aceastei suprafe\c{t}e nu e lucru u\c{s}or: Endra\ss{} a trebuit sa o caute intr-o familie
    $5$-dimensional\u{a} de octice, al c\u{a}rui membru general are doar $112$ singularit\u{a}\c{t}i.


    \^{I}n imaginea interactiv\u{a} simetria construc\c{t}iei este evident\u{a}.
    \^{I}n plus fa\c{t}\u{a} de simetria unui octogon regulat, suprafa\c{t}a este simetric\u{a}
    \c{s}i fa\c{t}\u{a} de planul $xy$.

    F\u{a}r\u{a} folosirea acestor simetrii, spa\c{t}iul de c\u{a}utare ar fi fost de dimensiune \c{s}i mai mare.
\end{surferPage}
