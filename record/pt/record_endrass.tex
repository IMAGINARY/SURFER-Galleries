\begin{surferPage}[Octica de Endraß]{A \'Octica de Endra\ss{}}
     Em 1995, Stephan Endra\ss{} construiu esta superf\'icie de grau $8$ (\'octica) e este seria o principal resultado da sua disserta\c c\~ao na Universidade de Erlangen.
    Ao todo, a superf\'icie tem $168$ singularidades, e este \'e ainda o atual recorde mundial.
  
     Atrav\'es de um resultado geral de Varchenko, sabe-se que uma \'octica n\~ao pode ter mais do que $174$ pontos singulares.
    Assim: $168 \le \mu(8) \le 174$. 
    O n\'umero exato n\~ao \'e conhecido.

     Encontrar esta superf\'icie n\~ao foi f\'acil: Endra\ss{} teve que procur\'a-la numa fam\'ilia de \'octicas de dimens\~ao $5$, na qual o membro geral da fam\'ilia tem apenas $112$ singularidades.

    Na imagem interativa, a simetria da constru\c c\~ao \'e aparente: 
    para al\'em da simetria do oct\'ogono regular, a superf\'icie \'e sim\'etrica em rela\c c\~ao ao plano $XY$.

    Sem usar essas simetrias, o espa\c co a pesquisar teria sido de dimens\~ao ainda maior.
\end{surferPage}
