\begin{surferPage}[Quíntica Togliatti]{Uma Qu\'intica de Togliatti}
    Eugenio Giuseppe Togliatti provou em 1937 que existe uma superf\'icie de grau $5$ (qu\'intica) com exatamente $31$ singularidades  --- um recorde mundial nessa \'epoca.


    Em 1980 foi Arneau Beauville quem utilizou uma rela\c c\~ao  interessante com teoria de c\'odigos a fim de provar a n\~ao--exist\^encia de um qu\'intica com um n\'umero de singularidades  superior a $31$.
   O que significa que o recorde mundial de Togliatti nunca poder\'a ser melhorado!

    Como n\~ao existe um s\'olido plat\'onico cujos planos de simetria pudessemos utilizar para construir uma superf\'icie de grau $5$ semelhante \`a  Qu\'artica de Kummer ou
    \`a S\^extica de Barth, a qu\'intica com $31$ singularidades tem menos simetrias,
    nomeadamente as simetrias do pent\'agono plano.


 A equa\c c\~ao que utilizamos aqui foi encontrada por Wolf Barth (1990);  utilizamos esta equa\c c\~ao porque a superf\'icie original de Togliatti n\~ao \'e f\'acil de visualizar.
 \end{surferPage}
