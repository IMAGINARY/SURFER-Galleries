\begin{surferPage}{Pasticcino}
\[(x^2+ \frac94y^2	+ z^2- 1)^3- x^2z^3	- \frac9{80}y^2z^3	= 0\]

\singlespacing
Lettera d'amore
\singlespacing
Non si pu\`o continuare ad andare avanti cos\`{\i}!\\
Quel che esiste, deve rimanere.\\
Quando ci rivedremo di nuovo,\\
qualcosa dovr\`a accadere.\\
Quel che ci spinge verso il picco.\\
In cima che cosa capiter\`a?\\
Di certo nessuna piattezza.\\
Perch\'e ci\`o che sta in cima,\\
non si consuma facilmente.\\
Su una panchina a Grunewald\\
In due, seduti, nella pioggia,\\
\`e stupido. Coraggio, ragazza! Scrivimi presto!\\
Per sempre tuo, Fritz! (Ricordati i picchi).\\
{\it Poesia di Joachim Ringelnatz, 1883-1934}
\singlespacing
La passione d'amore \`e generalmente collegata alla potenza emotiva di una ``singolarit\`a''. Questo collegamento compare in molte espressioni artistiche.
\singlespacing
Sperimenta cambiando l'ultimo cubo $z^3$ nell'equazione con un quadrato.
\end{surferPage}
