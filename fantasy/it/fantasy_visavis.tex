\begin{surferPage}{Vis à Vis}
Singolare o regulare - amico o nemico\\
\smallskip
\[x^2	- x^3+ y^2+ y^4+ z^3- z^4	=  0\]

\vspace{0.3cm}
Le punte, o {\it singolarit\`a}, sono identificate visivamente perch\'e la superficie non \`e liscia n\'e soffice, per esempio una cuspide o una piega.\\
\vspace{0.3cm}
La cuspide a sinistra della superficie Vis \`a Vis \`e una singolarit\`a; la collina liscia sulla destra \`e un punto regolare. Singolarit\`a sono interessanti perch\'e piccoli cambiamenti nell'equazione possono modificarli in modi sorprendenti. \\

\vspace{0.3cm}
Lo sai che ci sono persone che lavorano specificatamente per studiare questi punti? I buchi neri e il Big Bang sono singolarit\`a dell'equazione del modello cosmologico. E adesso guardati la punta delle dita: le singolarit\`a sui tuoi polpastrelli ti identificano!
\end{surferPage}
