\begin{surferIntroPage}{Superfici di Fantasia}{fantasy_kolibri}{Le superfici fantastiche}
Si sente spesso parlare di quanto complicata sia la matematica, ma \`e un dato di fatto che la matematica ci aiuta a analizzare la complessit\`a del mondo in cui viviamo. Per esempio, con il riconoscimento della struttura essenziale degli oggetti reali e le loro propriet\`a comuni. Raccogliere tutti gli oggetti con le stesse propriet\`a comuni in una \textit{classe}, tralasciando le propriet\`a che non interessano, si chiama classificazione. \`E uno dei metodi pi\`u importanti con cui realizzare un'analisi della quantit\`a stratosferica di oggetti e forme del mondo che ci circonda. Per fare questo \`e necessaria la matematica. Decidere che cosa sia importante o meno dipende da ci\`o che vogliamo capire. Questo pu\`o essere ad esempio la dimenzione o la forma di un oggetto.
\\

\vspace{0.4cm}

Descrivere e classificare forme \`e sempre stato un bisogno dell'uomo, come farlo non \`e per niente banale. Gli antichi greci usavano la geometria e le proporzioni. Pi\`u tardi gli arabi svilupparono l'algebra (Al Khwarizmi, 900 B.C.). Nel XVIII secolo i matematici un grande passo avanti fu ottenuto da Fermat e Descartes che introdussero i sistemi di coordinate per descrivere le relazioni geometriche, rendendo possibile l'uso congiunto di algebra e geometria.
\\
\vspace{0.4cm}
Il programma SURFER \`e un esempio diretto di questo uso congiunto dato che produce la geometria (l'immagine) a partire dall'algebra (la formula).
In questa galleria puoi sperimentare la bellezza della matematica e incominciare a creare oggetti matematici tu stesso. Scegli una delle superfici sulla destra. Il collegamento matematico tra la formula e la forma viene spiegato attraverso una serie di esempi in modo semplice.\\
L'immaginazione e l'intuizione sono solo tue \dots
\end{surferIntroPage}
