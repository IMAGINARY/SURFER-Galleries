\begin{surferPage}{Ding Dong}
Cambiare la figura cambiando l'equazione\\

\smallskip
\[x^2	+ y^2	+ z^3	= z^2\]

\singlespacing
L'equazione e la forma Ding Dong sono semplici. La figura si ottiene ruotando la lettera greca $\alpha$ intorno al suo asse. Se la guardi rovesciata, Ding Dong sembra una goccia di acqua. Si pu\`o osservare la goccia cadere.
\newline
Aggiungendo un piccolo parametro $a$ all'equazione e cambiandolo con continuit\`a, possiamo creare una serie di immagini che mostrano la formazione della goccia, l'avvicinamento al momento del distacco e il distacco. \`E come il fermo-immagine di un film:

\[x^2	+ y^2	+ z^3	-z^2+0.1\cdot a=0.\]

In ciascun istante la goccia \`e in equilibrio con la gravit\`a che compensa la tensione superficiale. Ma l'equilibrio della goccia non \`e stabile e si scuote prima di cadere. La teoria delle catastrofi inventata dal matematico Ren\'e Thom studia come piccoli cambiamenti nei parametri possono causare improvvisi cambi di equilibrio.
\end{surferPage}
