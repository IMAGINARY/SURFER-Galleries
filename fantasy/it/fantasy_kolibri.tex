\begin{surferPage}{Colibr\`{\i}}
L'equazione stabilisce quali sono i punti\\
  
  \smallskip
\[z^3+ y^2	z^2	= x^2\]

\singlespacing
In termini algebrici, il Colibr\`{\i} \`e dato da tutti i punti $(x, y, z)$ che verificano l'equazione
\smallskip
\[ x^2= y^2z^2+z^3.\]
\smallskip
Per esempio, $(0,0,0),$ $(1,0,1)$ e $(3,-2,-3)$ sono punti del Colibr\`{\i}, mentre $(0,1,1)$ no.\\
 \singlespacing
Il mondo tridimensionale in cui viviamo \`e governato da tre direzioni: avanti e indietro, sinistra e destra, su e gi\`u. Queste direzioni sono identificate da  $x$, $y$ e $z$. Ogni punto nello spazio \`e determinato dai valori di queste tre direzioni. I valori $(x,y,z)$ si chiamano le coordinate del punto.\\
\singlespacing
Inseriamo ora tutti i punti dello spazio con i loro valori nell'equazione e coloriamo soltanto quelli che soddisfano l'equazione. Tutti insieme i punti colorati formano l'immagine.
\end{surferPage}
