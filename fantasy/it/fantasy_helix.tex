\begin{surferPage}{Spirale}
Pi\`u sottile di una bolla di sapone\\
  \smallskip
\[6x^2	= 2x^4	+ y^2	z^2\]

\singlespacing
Le bolle di sapone sono delicate; sembra che scoppino soltanto a guardarle. La loro superficie ha due lati. L'esterno \`e di sapone, l'interno \`e d'acqua. Se lo strato di sapone diventa troppo sottile---capita quando la bolla diventa grande---l'acqua la fa scoppiare.\\
\vspace{0,3cm}
Le superfici algebriche sono molto pi\`u sottili di una bolla di sapone, c'\`e soltanto lo strato di punti. E visto che \`e la nostra immaginazione che crea questi punti, senza peso o densit\`a, non scoppiano, addirittura se ci sono punte o pieghe come nella Spirale.\\
\vspace{0,3cm}
Ma se vogliamo creare un modello tridimensionale reale della Spirale, dobbiamo costruire una scultura pi\`u spessa della effettiva superficie. Si pu\`o fare rinforzando la superficie su di un lato.
\end{surferPage}
