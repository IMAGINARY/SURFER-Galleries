\begin{surferPage}{Ugello}
Una parola con infinite lettere\\
\smallskip
\[y z (x^2	+ y - z)	= 0\]

\vspace{0.3cm}
Gli impressionisti dipingevano case e prati con migliaia di puntini colorati. Allo stesso modo, le superfici matematiche sono formate da migliaia di punti, punti che non hanno spessore o peso, ma che risolvono l'equazione! \\
\vspace{0.3cm}
Un modo per immaginare l'infinito \`e quello di iniziare a contare: $1, 2, 3,$ \dots\\
C'\`e sempre un altro numero e non arriveremo mai alla fine.\\
\vspace{0.3cm}
Ma non sono solo le superfici che contengono infiniti punti. Gi\`a tra $0$ e $1$ ce ne sono infiniti. Sembra impossibile? Prova a immaginare i punti infinitamente piccoli, disegnati con una matica di spessore zero. Per riempire la linea tra $0$ e $1$ devi disegnarne tantissimi, infinitamente tanti.
\end{surferPage}
