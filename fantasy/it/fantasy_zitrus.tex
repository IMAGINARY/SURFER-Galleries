\begin{surferPage}{Limone}
Questo non \`e un limone---le immagini possono ingannare\\
\smallskip
\[x^2 + z^2 = y^3 (1 - y)^3\] 


\singlespacing
Senza dubbio guardando la prima volta questa immagine pensiamo: <<\`E un limone>>. Ma, se fosse un limone, perch\'e non ha profumo o sapore? Perch\'e \`e liscio e senza macchie? Certamente non \`e un limone!
\singlespacing
Questa forma non \`e un limone, ma un suo modello matematico. Ci aiuta per migliorare la comprensione delle propriet\`a della forma del limone. In geografia c'\`e una citazione simile di $Alfred\ H.\ S.\ Korzybski$: ``La carta geografica non \`e il terreno.'' \\
\singlespacing

Le equazioni ci permettono di costruire modelli matematici che ci aiutano per studiare accuratamente la forma delle cose.
\singlespacing
Questo fa parte della poesia della matematica: possiamo generare bellissime superfici per mezzo di equazioni algebriche che ci trasportano col pensiero verso parti inattese del cervello.
\end{surferPage}
