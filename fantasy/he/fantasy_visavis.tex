\begin{surferPage}{אל מול - Vis \`a Vis}
ייחודי או רגיל – ידיד או אויב
\smallskip
\[x^2	- x^3+ y^2+ y^4+ z^3- z^4	=  0\]

\vspace{0,3cm}
ניתן לזהות באופן חזותי נקודות סינגולריות משום שהמשטח אינו חלק או רך; לדוגמה חוד או קפל.
\vspace{0,3cm}
החוד בחלק השמאלי של משטח Vis \`a Vis הוא נקודת סינגולריות; ואילו הגבעה החלקה בצד הימני היא נקודה רגילה. נקודות סינגולריות מעוררות עניין משום ששינויים קטנים במתכונת המשוואה יכולים לשנות את הופעתן בצורה מפתיעה. \\

\vspace{0,3cm}
הידעתם שישנם אנשים המקדישים את כל זמנם לחקר של נקודות אלה? חורים שחורים והמפץ הגדול מהווים נקודות סינגולריות של משוואות בקנה מידה קוסמולוגי. כעת, הביטו בקצות האצבעות שלכם; הסינגולריות של טביעות האצבעות שלנו מייחדת כל אחת ואחד מאיתנו!
\end{surferPage}
