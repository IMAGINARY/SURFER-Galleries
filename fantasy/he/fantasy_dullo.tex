\begin{surferPage}{\textenglish{Dullo}}
תופעה ייחודית בטבע
\smallskip
\[(x^2+ y^2+ z^2)^2	= x^2+ y^2\]

למתמטיקה זיקה הדוקה לענפי מדע אחרים כגון פיזיקה, כימיה או טכנולוגיה. המתמטיקה מעמידה לרשותנו כלים רבי-עוצמה המאפשרים לנו להבין טוב יותר את העולם הסובב. 

לדוגמה, תופעות רבות שבהן אנו נתקלים בעודנו חוקרים את הטבע, מובילים ליצירה של מודלים הכוללים נקודות סינגולריות.

מקרה אחד כזה הוא התפשטות של גלי הקול הנוצרים על-ידי מחיאות הכפיים של קהל האוהדים הנלהבים באצטדיון כדורגל. הצורה שאותה לובשת תופעה זו נקראת משטח דוּלוֹ 
\textenglish{(Dullo surface)}. לצורה זו נקודת סינגולריות ברורה במרכזה, ולפיכך, שופט הכדורגל נמנע מלעמוד בנקודת סינגולריות זו על המגרש כאשר הקהל מריע לאחר הבקעת שער. עוצמת הרעש בנקודה זו מגיעה לרמות העלולות לגרום נזק לאוזניו!
\end{surferPage}
