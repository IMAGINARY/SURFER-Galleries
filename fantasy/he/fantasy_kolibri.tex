\begin{surferPage}{יונק הדבש}
המשוואה היא הקובעת את מיקום הנקודות
  
  \smallskip
\[z^3+ y^2	z^2	= x^2\]

\singlespacing
במונחים אלגבריים, יונק הדבש (Hummingbird) נוצר על-ידי כל הנקודות $(x, y, z)$ המקיימות את המשוואה
\smallskip
\[ x^2= y^2z^2+z^3.\]
\smallskip
לדוגמה: $(0,0,0),$ $(1,0,1)$ ו- $(3,-2,-3)$ הן נקודות על המשטח של יונק הדבש, ואילו $(0,1,1)$ לא נמצא על המשטח.\\
 \singlespacing
 עולמנו התלת-ממדי נשלט על-ידי שלושה כיוונים, קדימה ואחורה, ימינה ושמאלה, למעלה ולמטה. כיוונים אלה מיוצגים על-ידי הצירים $x$, $y$ ו-$z$. ניתן לתאר כל נקודה בחלל באמצעות הערכים עבור שלושת הכיוונים הללו. ערכים אלה נקראים קואורדינטות $(x,y,z)$ של הנקודה.\\
\singlespacing
כעת נציב את כל הנקודות בחלל ביחד עם ערכיהם בתוך המשוואה, ונצבע רק את אלה המקיימים את המשוואה. סך כל הנקודות הצבועות מניב בעצם את התמונה.
\end{surferPage}
