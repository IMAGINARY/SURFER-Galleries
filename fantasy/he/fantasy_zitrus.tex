\begin{surferPage}{ לימון - Zitrus}
זה אינו לימון – התמונה עשויה לתעתע בכם\\
\smallskip
\[x^2 + z^2 = y^3 (1 - y)^3\] 


\singlespacing
במבט ראשון, רבים מאיתנו ודאי יאמרו לעצמם: "אה, זה לימון". אך אם זה אכן לימון, מדוע אין לו ניחוח או טעם? מדוע אין לו נקבוביות וכתמים? ברור שהצורה שלפנינו אינה לימון אמיתי! 
\singlespacing
ואמנם, הצורה שלפנינו אינה לימון, כי אם מודל מתמטי של לימון. הוא מסייע לנו להבין טוב יותר את מאפייניה של צורת הלימון. בגאוגרפיה קיימת אמרה מתאימה שאותה טבע הפילוסוף והמדען $אלפרד \ ה.\ ס.\ קוז'יבסקי (Alfred\ H.\ S.\ Korzybski)$: המפה אינה השטח" \\
\singlespacing

משוואות מאפשרות לבנות מודלים מתמטיים אשר מסייעים לנו לחקור את הצורות השונות בעולם שסביבנו. 
\singlespacing
כל אלה מהווים חלק מהפואטיקה של המתמטיקה: ביכולתנו ליצור משטחים בעלי יופי יוצא דופן באמצעות משוואות אלגבריות היוצרות גשר בין המחשבות שלנו לפינות בלתי מוכרות של מוחנו.
\end{surferPage}
