\documentclass[es]{SurferDesc}%%%%%%%%%%%%%%%%%%%%%%%%%%%%%%%%%%%%%%%%%%%%%%%%%%%%%%%%%%%%%%%%%%%%%%%
%
% The document starts here:
%
\begin{document}
\footnotesize
% Einfache Singularitäten 


\begin{surferPage}
  \begin{surferTitle}Vos y Yo\end{surferTitle}  \\
Singular versus liso - amigo o enemigo\\
\smallskip
\[x^2	- x^3+ y^2+ y^4+ z^3- z^4	=  0\]

\vspace{0.3cm}
Los puntos singulares, o singularidades, a menudo se identifican f\'acilmente de forma visual porque son puntos donde la superficie no es lisa ni suave, como por ejemplo un pico o un pliegue.\\
\vspace{0.3cm}
La superficie Vos y Yo ilustra muy bien lo que es una singularidad, el pico de la izquierda, y lo que no lo es, la colina lisa de la derecha. Las singularidades son interesantes entre otras cosas porque, al contrario de lo que ocurre con los puntos lisos que son estables, pequeños cambios en la ecuaci\'on pueden cambiar su aspecto de un modo sorprendente.\\
\vspace{0.3cm}
¿Sab\'es que hay gente que se dedica especialmente al estudio de estos puntos? Los agujeros negros y el principio del universo, Big Bang, son singularidades de las ecuaciones de los modelos cosmol\'ogicos. Sin ir m\'as lejos, ¡las singularidades de nuestras huellas dactilares nos identifican!

  \begin{surferText}
     \end{surferText}
\end{surferPage}


\end{document}
%
% end of the document.
%
%%%%%%%%%%%%%%%%%%%%%%%%%%%%%%%%%%%%%%%%%%%%%%%%%%%%%%%%%%%%%%%%%%%%%%%
