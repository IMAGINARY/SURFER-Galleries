\begin{surferPage}{Limón}
Esto no es un lim\'on - la traici\'on de las im\'agenes\\
\smallskip
\[x^2 + z^2 = y^3 (1 - y)^3\] 

\vspace{0.3cm}
Al ver esta imagen seguramente todos hemos pensado: “esto es un lim\'on”. Pero si es un lim\'on, ¿por qu\'e no tiene olor ni sabor? ¿por qu\'e no tiene ni poros ni manchas? ¡Est\'a claro que esto no puede ser un lim\'on!\\
\vspace{0.3cm}
En efecto, esta figura no es un lim\'on, sino un modelo matem\'atico de un lim\'on, que nos ayuda a entender mejor las propiedades de la forma que tiene \'este. Las ecuaciones nos permiten construir modelos matem\'aticos que se parecen a las cosas, y estudiar estos modelos matem\'aticos nos ayuda, a su vez, a entender mejor la forma de las cosas.
\vspace{0.3cm}
\begin{center}
\emph{El mapa no es el territorio}, Alfred H. S. Korzybski
\end{center}
\vspace{0.3cm}
Todo esto forma parte de la “poes\'ia” de la Matem\'atica. A partir de ecuaciones algebraicas podemos generar bellas superficies que transportan nuestros pensamientos hasta rincones insospechados de nuestra mente.
\end{surferPage}
