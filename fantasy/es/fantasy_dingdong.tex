\begin{surferPage}{Gota}
Cambia la figura cambiando la ecuación\\

\smallskip
\[x^2	+ y^2	+ z^3	= z^2\]

\singlespacing
La ecuación y la forma de Gota son simples. La figura se obtiene girando la letra griega $\alpha$ (Alfa) alrededor de su eje. Si la mirás desde arriba, Gota parece una gota de agua cayendo, de ahí su nombre.
\newline
Si le agregás un pequeño parámetro $a$ a la ecuación y lo modificás varias veces, podemos crear una serie de imágenes que muestran cómo emerge la gota, cómo se aproxima a su forma final, y luego se separa. Es como una serie de imágenes fijas de una película: 

\[x^2	+ y^2	+ z^3	-z^2+0.1\cdot a=0.\]

En todo momento la gota está en una situación de equilibrio donde la gravedad compensa la tensión de la superficie. Pero el equilibrio de la gota no es estable y tiembla antes de caerse. La Teoría de las catástrofes, del matemático Ren\'e Thom, estudia cómo pequeñas modificaciones en los parámetros pueden causar cambios inmediatos en el equilibrio.
\end{surferPage}
