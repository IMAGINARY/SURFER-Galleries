\documentclass[es]{../../common/SurferDesc}%%%%%%%%%%%%%%%%%%%%%%%%%%%%%%%%%%%%%%%%%%%%%%%%%%%%%%%%%%%%%%%%%%%%%%%
%
% The document starts here:
%
\begin{document}
\footnotesize
% Einfache Singularitäten 

 
%%%%%%%%%%%%%%%%%%%%%%%%%%%%%%%%

\begin{surferPage}
  \begin{surferTitle}Trompo\end{surferTitle}  \\
La ecuaci\'on, un nombre inequ\'ivoco\\
\smallskip
\[x^2 + y^2	= z^3	(1 - z) \]

\vspace{0.3cm}
Todas las figuras que se muestran en la exposici\'on tienen nombre. Si tuvieras que nombrarlas, ¿qu\'e nombres hubieras elegido?\\
\vspace{0.3cm}
¿Qu\'e nombres cre\'es que habr\'a puesto otra persona? ¡Preguntalo!\\
\vspace{0.3cm}
Pero, ¿podemos encontrar un modo de nombrar figuras que nunca lleve a confusi\'on? En la Matem\'atica se ha resuelto nombrarlas por su ecuaci\'on.\\
\vspace{0.3cm}
Una sola ecuaci\'on determina toda la figura, todos sus puntos, todas sus curvas, incluso todos sus agujeros, todos sus pliegues y todas sus puntas. S\'olo falta aprender a encontrarlos en la ecuaci\'on o saber dibujarla. Es como cuando te enseñan a escribir: una vez que sab\'es escribir sin faltas una palabra, todo el mundo la entiende.\\
\vspace{0.3cm}
Adem\'as, las ecuaciones se escriben y se interpretan igual en todas partes, porque el lenguaje de la Matem\'atica es universal, como las partituras musicales.

  \begin{surferText}
     \end{surferText}
\end{surferPage}



\end{document}
%
% end of the document.
%
%%%%%%%%%%%%%%%%%%%%%%%%%%%%%%%%%%%%%%%%%%%%%%%%%%%%%%%%%%%%%%%%%%%%%%%