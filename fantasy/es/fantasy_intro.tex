\begin{surferIntroPage}{Superficies de Fantasia}{fantasy_kolibri}{Las Superficies de Fantasía}
Muchas veces oímos hablar de lo complicada que es la matemática, pero no podemos negar que nos ayuda a comprender la complejidad del mundo en que vivimos. Por ejemplo, en el reconocimiento de estructuras básicas y en el conocimiento de importantes propiedades de los objetos de la realidad. Una de las formas que tenemos de construir una visión general de estas formas y objetos es clasificándolas. Es decir, reunir todos los objetos que tengan una misma propiedad en una \textit{clase}, e ignorar las de menor importancia. Algo importante podría ser, en principio, el tamaño o la forma de un objeto. Para esto, la matemática es fundamental. Decidir qué es lo importante y qué no lo es, depende de lo que querramos comprender.

Desde los primeros tiempos, describir y clasificar formas es una necesidad humana; cómo hacerlo no es para nada obvio. Los antiguos Griegos usaban fundamentalmente la Geometría y las proporciones de objetos geométricos. Luego, los árabes difundieron el uso del Álgebra (Al Khwarizmi, 900 A.C.). Pero durante el siglo XVIII, la introducción del sistema de coordenadas para describir relaciones geométricas fue un importante trabajo realizado por los matematicos Descartes y Fermat. Esto hizo posible el uso conjunto del Álgebra y la Geometría.

El programa SURFER es un ejemplo de esta relación, porque crea geometría (la imagen) a partir del álgebra (la fórmula).
En esta galería podrás experimentar la belleza de la matemática y ser vos mismo el creador. Elegí una de las superficies del lado derecho. La conexión matemática entre la ecuación y la forma está explicada por medio de una serie de ejemplos sencillos. \\
La imaginación y la intuición van por tu cuenta ...
\end{surferIntroPage}
