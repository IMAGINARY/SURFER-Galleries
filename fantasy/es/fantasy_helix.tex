\begin{surferPage}{Hélice}
Mas fino que una capa de jabón\\
  \smallskip
\[6x^2	= 2x^4	+ y^2	z^2\]

\singlespacing
Las burbujas de jabón son muy sensibles; parecen explotar con solo mirarlas. Sus superficies tienen dos caras. La cara externa es de jabón y la interna de agua. Si la capa de jabón se vuelve muy fina - es decir si la burbuja se vuelve más grande - el agua la hace explotar.\\
\vspace{0,3cm}
Las superficies algebraicas son incluso más finas que las capas de jabón, consisten únicamente de puntos. Y como usamos nuestra imaginación para crear estos puntos, no tienen masa ni densidad, no explotan, incluso pueden tener picos o torceduras como la hélice.\\
\vspace{0,3cm}
Para poder crear un modelo tridimensional de la superficie de la hélice, tenemos que construir una figura más gruesa que la hélice original, debido a que los puntos que la forman no tienen masa. Esto lo podemos hacer reforzando la superficie de algún lado.
\end{surferPage}
