\begin{surferPage}[Dullo]{La curva de Dullo}
Un fenómeno único en la naturaleza\\
\smallskip
\[(x^2+ y^2+ z^2)^2	= x^2+ y^2\]

\singlespacing
La matemática esta muy emparentada con otras ciencias como la física, la química o la computación, las cuales nos proveen de poderosas herramientas para entender el mundo que nos rodea. 
\singlespacing
Por ejemplo, muchos fenómenos con los cuales nos encontramos al estudiar la naturaleza dan surgimiento a modelos con singularidades.
\singlespacing
Este es el caso del estudio de la propagación de las ondas de sonido producido por el aplauso del público en un estadio de fútbol. Esta clase de fenómenos, toman la forma de la surpeficie de Dullo. La superficie de Dullo tiene una clara singularidad en el centro, por eso el árbitro del partido evitará estar en esta parte de la cancha cuando se celebre un gol. ¡El ruido le lastimaría los oidos!
\end{surferPage}
