\begin{surferPage}{Nepalés}
El mundo sin fin\\

\smallskip
\[(x y - z^3 -1)^2= (1 - x^2	- y^2)^3\]

\singlespacing
Imaginate que hubiese una superficie simplemente bella y te guste y la querés encerrar en una bola de cristal para decorar. ¡Pero no podés elegir cualquier superficie para poner en tu cuarto!
\\
\singlespacing
Existen superficies que son infinitas, es decir que no tienen un límite e, incluso si fueran lo suficientemente lindas, nunca serías capaz de encerrarlas en una bola de cristal por completo, sin importar el tamaño de la bola. En estos casos decimos que la superficie es \textit{no acotada}. Para poder graficar la superficie debemos ocultar algunas partes de la misma.
\\
\singlespacing
La propiedad de ser no acotada no es fácil de reconocer, ni siquiera con la ayuda del SURFER. Es como si quisiéramos saber si el universo es infinito, dado que no conocemos sus bordes o fronteras, puede tenerlas como no.
\end{surferPage}
