\documentclass[es]{./../../common/SurferDesc}%%%%%%%%%%%%%%%%%%%%%%%%%%%%%%%%%%%%%%%%%%%%%%%%%%%%%%%%%%%%%%%%%%%%%%%
%
% The document starts here:
%
\begin{document}
\footnotesize
% Einfache Singularitäten 




%%%%%%%%%%%%%%%%%%%%%%%%%%%%%

\begin{surferPage}
  \begin{surferTitle}Quaste\end{surferTitle} \\
The ABC of equations
  \smallskip
\[8z^9-24x^2z^6-24y^2z^6+36z^8+24x^4z^3-168x^2y^2z^3\]
\[+24y^4z^3-72x^2z^5-72y^2z^5+54z^7-8x^6-24x^4y^2\]
\[-24x^2y^4-8y^6 + 36x^4z^2-252x^2y^2z^2+36y^4z^2\]
\[- 54x^2z^4-108y^2z^4 + 27z^6-108x^2y^2z + 54y^4z\]
\[-54y^2z^3 + 27y^4 = 0\]\\
\vspace{0.3cm}
Did you have a close look at the equation of Quaste? It  looks very complicated.
The figure itself can be described in simple words: the upper border has the form of the Greek letter $\alpha$, the right border has the shape of a curve with a peak. Such a peak is called {\it cusp}. If you drag such a cusp along the alpha curve you obtain Quaste. Surfaces with such a property are called Cartesian products, in honour of the French mathematician  Ren\'e Descartes.\\
\vspace{0.3cm}
Monomials of degree $1$ are $x$, $y$, $z$. Monomials of degree $2$ are $x^2, xy, y^2, xz, yz, z^2$. And so on. The higher the degree, the more monomials we have, and this affords us more possibilities to create more complicated shapes. It is like an alphabet: if we have more letters at our disposal, we can write more complex words and phrases. 




  \begin{surferText}
     \end{surferText}
\end{surferPage}
%%%%%%%%%%%%%%%%%%%%%%%%%%%%%
\end{document}