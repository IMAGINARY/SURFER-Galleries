\begin{surferPage}{Pasión}
Singularidad en el coraz\'on\\
\smallskip
\[(x^2+ 9/4y^2	+ z^2- 1)^3- x^2z^3	- 9/80y^2z^3	= 0\]

\singlespacing
Fuera de ley, mi coraz\'on\\a saltos va en su desaz\'on.\\
\vspace{0.3cm}
Ya muerde ac\'a, sucumbe all\'i,\\cazando all\'a, cazando aqu\'i.\\
\vspace{0.3cm}
Donde lo intente yo dejar\\mi coraz\'on no se ha de estar.\\
\vspace{0.3cm}
Donde lo deba yo poner\\mi coraz\'on no ha de querer.\\
\vspace{0.3cm}
Cuando le diga yo que s\'i,\\dir\'a que no, contrario a m\'i.\\
\vspace{0.3cm}
Bravo le\'on, mi coraz\'on\\tiene apetitos, no raz\'on.

\begin{flushright}
{\it Fuera de ley, Alfonsina Storni}
\end{flushright}

\singlespacing 
La pasi\'on amorosa acostumbra a identificarse con la fuerza emotiva de alguna ``singularidad'' dolorosa y por ello esta asociaci\'on ha sido ampliamente utilizada en el arte, no solamente arquitect\'onico y pict\'orico, sino tambi\'en en el narrativo e interpretativo.
\end{surferPage}
