\begin{surferPage}{장식용 술}
식에서의 ABC
  \smallskip
\[8z^9-24x^2z^6-24y^2z^6+36z^8+24x^4z^3-168x^2y^2z^3\]
\[+24y^4z^3-72x^2z^5-72y^2z^5+54z^7-8x^6-24x^4y^2\]
\[-24x^2y^4-8y^6 + 36x^4z^2-252x^2y^2z^2+36y^4z^2\]
\[- 54x^2z^4-108y^2z^4 + 27z^6-108x^2y^2z + 54y^4z\]
\[-54y^2z^3 + 27y^4 = 0\]\\
\vspace{0.3cm}
이 식을 자세히 들여다 보셨나요? 매우 복잡해 보이죠? 그러나 간단하게 설명할 수 있습니다. 위쪽 경계는 그리스문자 $\alpha$ 처럼 생겼고, 오른쪽 경계는 뾰족한 점이 있는 곡선을 가지고 있습니다. 이러한 꼭지점을 {\it 뾰족점} 이라고 부릅니다. 이 뾰족점을 $\alpha$ 곡선을 따라 끌면 장식용 술 모양을 얻을 것입이다. \\
\vspace{0.3cm}
차수 $1$ 의 단항식은 $x$, $y$, $z$ 입니다. 차수 $2$ 의 단항식은 $x^2, xy, y^2, xz, yz, z^2$ 입니다. 마찬가지로 더 높은 차수의 단항식을 생각해 볼 수 있습니다. 당연히 차수가 높을 수록 더 많은 종류의 단항식을 얻을 수 있습니다. 더 높은 차수의 단항식을 이용하면 더 복잡한 모양들을 만들 수 있습니다. 알파벳처럼 사용할 수 있는 문자가 많아지면 더 복잡한 단어나 문장을 쓸 수 있듯이요.
\end{surferPage}
