\begin{surferPage}{벌새}
방정식은 좌표계 위의 점들을 결정합니다. \\
  
  \smallskip
\[x^3+ x^2	z^2	= y^2\]

\singlespacing
벌새는 다음 식을 만족하는 점 $(x, y, z)$ 로 구성되어 있습니다.
\smallskip
\[ y^2= x^3 + x^2 z^2.\]
\smallskip
예를들면, $(0,0,0),$ $(1,1,0)$ 그리고 $(-3,-2,3)$ 벌새 위의 점이지만 $(0,1,1)$ 은 아닙니다.\\
 \singlespacing
삼차원 세상은 세개의 방향에 의해 결정됩니다. 앞뒤, 좌우 그리고 위아래. 이 방향은 세 변수 $x$, $y$ 그리고 $z$를 이용해 표현됩니다. 공간의 모든 점의 위치는 이 세 변수의 값에 의해서 결정됩니다. 이것을 우리는 좌표라 부릅니다.\\
\singlespacing
주어진 식을 만족하는 점들에만 색깔을 입히면 오른쪽 그림과 같은 대수곡면이 만들어 집니다.
\end{surferPage}
