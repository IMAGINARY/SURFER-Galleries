\begin{surferPage}{벌새}
식은 좌표계 위의 점들을 결정합니다. \\
  
  \smallskip
\[z^3+ y^2	z^2	= x^2\]

\singlespacing
대수적 용어로 벌새는 다음 식을 만족하는 점 $(x, y, z)$ 들에 의해 만들어 집니다.
\smallskip
\[ x^2= y^2z^2+z^3.\]
\smallskip
예를들면, $(0,0,0),$ $(1,0,1)$ 그리고 $(3,-2,-3)$ 벌새 위의 점이지만 $(0,1,1)$ 은 아닙니다.\\
 \singlespacing
우리의 삼차원 세상은 세개의 방향에 의해 결정됩니다. 앞뒤, 좌우 그리고 위아래. 이러한 방향을 각각 $x$, $y$ 그리고 $z$ 방향이라고 합니다. 공간의 모든 점들의 위치는 이 세 방향 크기에 의해서 결정됩니다. 이것을 우리는 좌표라 부릅니다.\\
\singlespacing
주어진 식을 만족하는 점들에만 색깔을 입히면 오른쪽 그림과 같은 대수곡면이 만들어 집니다.
\end{surferPage}
