\begin{surferPage}{Nepali}
끝없는 세상 \\
\smallskip
\[(x y - z^3 -1)^2= (1 - x^2	- y^2)^3\]

\singlespacing
여러분도 아름다운 곡면을 만들어 유리공 안에 넣어서 가지고 놀고 싶을 겁니다. 그러나 모든 곡면을 다 넣을 수 있는  것은 아닙니다.
\\
\singlespacing
어떤 곡면은 무한히 커서 설령 너무나 매력적인 곡면일지라도 절대 유한한 크기의 유리공 안에 집어 넣을 수 없습니다. 이러한 곡면을 \textit{무한 곡면} 이라고 합니다. 이런 곡면에 색을 입하기 위해서는 유한한 크기를 갖도록 잘라내어야 합니다. 
\\
\singlespacing
SURFER의 도움을 받아도 어떤 곡면이 유한곡면인지 판별하는 것은 쉽지 않습니다. 이것은 마치 우주가 유한한가를 알아내는 것과 마찬가지 입니다.
\end{surferPage}
