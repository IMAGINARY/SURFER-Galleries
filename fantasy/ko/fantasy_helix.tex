\begin{surferPage}{헬리콥터}
비누막 보다 얇은 곡면\\
  \smallskip
\[6x^2	= 2x^4	+ y^2	z^2\]

\singlespacing
비누방울은 외부 충격에 매우 민감합니다. 쳐다만 봐도 비누방울이 터질 것 같지 않나요? 비누방울의 안쪽은 물로 이루어져 있고, 바깥쪽은 비누거품으로 이루어져 있습니다. 비누방울의 표면이 얇아지면, 예를들어  방울이 더 커질 때, 안쪽의 물이 방울을 터뜨려버립니다.\\
\vspace{0,3cm}
대수 곡면은 질량이나 부피가 없는 점이 모여 이루어지기 때문에 비누방울보다도 더 얇습니다. 또한 대수곡면은 우리의 상상력을 통해 만들어지기 때문에 주름 또는 뽀족점을 가지고 있더라도 터지지 않습니다.\\
\vspace{0,3cm}

하지만 여러분이 나선곡면의 형태를 가진 조각품 만들고자 한다면, 실제 나선곡면과는 달리 각 면을 두껍게 만들어야 합니다.
\end{surferPage}
