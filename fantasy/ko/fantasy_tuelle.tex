\begin{surferPage}{노즐}
무한히 많은 문자로 이루어진 단어\\
\smallskip
\[y z (x^2	+ y - z)	= 0\]

\vspace{0.3cm}
인상주의자들은 수 천개 혹은 수 만개의 색깔있는 점으로 집과 초원을 그렸습니다. 유사하게 수학적 곡면은 무수히 많은 점들로 이루어져 있습니다. 그러나 이 점들은 폭이나 질량이 없는, 그저 식을 만족시키는 좌표를 가지는 점일 뿐입니다! \\
\vspace{0.3cm}
우리는 숫자를 하나씩 세어가며 무한을 이해할 수 있습니다. $1, 2, 3,$ \dots\\
우리가 말한 수보다 항상 더 큰 숫자가 존재하고 그 결과 우리는 절대 끝까지 셀 수 없습니다. \\
\vspace{0.3cm}
그러나 곡면만이 무한히 많은 점을 포함하는 건 아닙니다. 수직선의 $0$ 과 $1$ 사이에도 무한히 많은 점이 존재합니다. 불가능해 보이나요? 두께가 0인 연필로 그린 무한히 작은 점을 상상해 보세요. $0$ 과 $1$ 을 채우기 위해서는 셀 수 없이 많은 점들이 필요하게 됩니다. 우리는 이러한 것들을 무한이라고 부릅니다.
\end{surferPage}
