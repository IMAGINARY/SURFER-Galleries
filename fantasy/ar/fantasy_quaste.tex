\begin{surferPage}[ويست]{كويست (Quaste)}
ألف باء المعادلات
  \smallskip
\[8z^9-24x^2z^6-24y^2z^6+36z^8+24x^4z^3-168x^2y^2z^3\]
\[+24y^4z^3-72x^2z^5-72y^2z^5+54z^7-8x^6-24x^4y^2\]
\[-24x^2y^4-8y^6 + 36x^4z^2-252x^2y^2z^2+36y^4z^2\]
\[- 54x^2z^4-108y^2z^4 + 27z^6-108x^2y^2z + 54y^4z\]
\[-54y^2z^3 + 27y^4 = 0\]\\
\vspace{0.3cm}
هل نظرتم عن كثب إلى معادلة الكويست؟ إنها تبدو معقدة جداً.
يمكن وصف الشكل بحد ذاته بكلمات بسيطة: الطرف الأعلى له شكل حرف ألفا اليوناني $\alpha$، الطرف اليمين له شكل منحنى مع قمة. تُسمى هذه القمة نقطة الإرتداد. عند تنقيل نقطة الإرتداد على طول المنحنى على شكل ألفا، نحصل على كويست. يُسمى الشكل الذي يملك خاصة مماثلة  بالحاصل الكارتيزي وذلك تكريماً لعالم الرياضيات الفرنسي رينيه ديكارت \textenglish{(Ren\'e Descartes)}.

\vspace{0.3cm}
أحاديات الحدود من الدرجة $1$ هي $x$ $y$  $z$. أحاديات الحدود من الدرجة $2$ هي
 $x^2, xy, y^2, xz, yz, z^2$  
. إلى آخره. كلما كبرت الدرجة، كلما زاد عدد أحاديات الحدود مما يوفر لنا المزيد من الإمكانيات لخلق أشكال أكثر تعقيداً. هذا يشبه الأبجدية: حين يتوفر لنا عدد أكبر من الأحرف، يمكننا كتابة كلمات وجمل أكثر تعقيداً.
\end{surferPage}
