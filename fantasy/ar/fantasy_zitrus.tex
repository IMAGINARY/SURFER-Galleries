\begin{surferPage}[ليمون]{ليمون }
هذا ليس ليمون حامض - خيانة الصور\\
\smallskip
\[x^2 + z^2 = y^3 (1 - y)^3\]


\singlespacing
عند رؤية هذه الصورة، لا شك أن الفكرة الأولى التي تأتي إلى الذهن هي: "هذا ليمون حامض". ولكن لماذا ليس لهذا الليمون الحامض رائحة أو طعم؟ لماذا ليس له مسام أو بقع؟ من الواضح أنه لا يعقل أن يكون ليموناً حامضاً!
\singlespacing
ليس هذا الشكل ليموناً حامضاً وإنما نموذجاً رياضياً لليمون الحامض. يساعدنا النموذج على فهم خصائص شكل الليمون الحامض بشكل أفضل. في الجغرافيا، يوجد قول مماثل من ألفرد كورزيبسكي
\textenglish{\mbox{(Alfred\ H.\ S.\ Korzybski)}}: 
 "إن الخريطة ليست الأرض." \\
\singlespacing

تسمح لنا المعادلات ببناء نماذج رياضية تساعدنا على دراسة أشكال الأجسام بشكل أفضل.
\singlespacing
هذا ما يعطي للرياضيات جانباً شاعرياً: يمكننا خلق أشكال جميلة بواسطة معادلات تنقل أفكارنا إلى آفاق غير متوقعة في أذهاننا.
\end{surferPage}
