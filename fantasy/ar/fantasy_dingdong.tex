\begin{surferPage}{دينغ دونغ (Ding Dong)}
يمكن تغيير الشكل عن طريق تغيير المعادلة\\

\smallskip
\[x^2	+ y^2	+ z^3	= z^2\]

\singlespacing
يملك الدينغ دونغ (Ding Dong) شكلاً ومعادلة في غاية البساطة.  يتم الحصول على هذا الشكل بتدوير الحرف اليوناني ألفا حول محوره. إذا نظرتم إليه رأساً على عقب، سيبدو لكم الدينغ دونغ كنقطة ماء. يمكننا مشاهدة النقطة أثناء تساقطها.
\newline
ذا أضفنا بارامتر  $a$  صغير القيمة إلى المعادلة وقمنا بتغييره بإستمرار، فيمكننا أن نخلق سلسلة من الصور التي تبين ظهور النقطة، وكيفية تقدمها من وضعها النهائي وأخيراً إنفصالها، كلقطة ثابتة مأخوذة من فيلم: 


\[x^2	+ y^2	+ z^3	-z^2+0.1\cdot a=0.\]

في كل لحظة، تكون النقطة في وضع توازن حيث تعادل قوة الجاذبية قوة التوتر السطحي، ولكن توازن النقطة ليس ثابتاً مما يجعلها ترتعش قبل سقوطها. تقوم نظرية الكوارث التي أوجدها العالم الرياضي رينيه ثوم بدراسة تغييرات التوازن الفورية الناشئة عن تغييرات بسيطة في البارامتر.
\end{surferPage} 