\begin{surferPage}[نيبالي]{نيبالي (Nepali)}
عالم بدون نهاية \\

\smallskip
\[(x y - z^3 -1)^2= (1 - x^2 - y^2)^3\]

\singlespacing
قد نرى شكلاً يعجبنا فنريد أن نحفظه داخل كرة بلورية مليئة بالثلج. ولكن من الخطأ الإعتقاد أنه بالإمكان إختيار اي شكل كان لعرضه في غرفة الجلوس!
\\
\singlespacing
هناك أشكال تمتد إلى ما لانهاية وحتى لو كانت في غاية الجمال، فسيكون من المستحيل حجزها داخل كرة بلورية، مهما كبر حجم هذه الكرة. في هذه الحال، نتكلم عن السطوح \textit{غير المقيدة}. لرسم مثل هذا السطح علينا إخفاء أجزاء منه.
\\
\singlespacing
لا يمكن التعرف بسهولة على الخصائص التي يجب تقييدها، حتى مع مساعدة برنامج \textenglish{SURFER}. فهذا كما لو أننا نريد أن نعرف ما إذا كان الكون مقيداً: بما أننا لا نعرف حدوده، فإننا لا نعرف إن كان له أو لا.
\end{surferPage}
