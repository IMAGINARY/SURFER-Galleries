\begin{surferIntroPage}{اشكال من الخيال}{fantasy_kolibri}{اشكال من الخيال}
غالباً ما نسمع كم هي معقدة الرياضيات، ولكنها حقيقة مثبة أن الرياضيات تساعدنا على فهم تعقيدات عالمنا، مثلاً، من خلال التعرف على الهياكل الأساسية والخصائص المشتركة المهمة التي تملكها الأجسام الحقيقية. جمع كافة الأجسام التي تملك نفس الخصائص المهمة في فئة واحدة مع تجاهل الخصائص الأقل أهمية هو ما يُدعى التصنيف. إنها أحد أهم الوسائل التي تسمح بالحصول على لمحة عامة عن التنوع اللامتناهي الهائل من كائنات وأشكال في عالمنا. والرياضيات أساسية لتحقيق هذا الغرض: تقرير ما هو مهم وما هو أقل أهمية يعتمد على ما نريد أن نفهمه. فقد نهتم بشكل الجسم أو بحجمه.
 \\

\vspace{0.4cm}

منذ القدم، إحتاج الإنسان إلى وصف وتصنيف الأشكال ولكن لم يكن من السهل إنجاز ذلك. قدماء اليونان، استعملوا الهندسة ونسب الأجسام الهندسية. لاحقاً، طور العرب علم الجبر (الخوارزمي، 900 قبل الميلاد). في القرن الثامن عشر، حقق كل من عالمي الرياضيات فيرما
 \textenglish{(Fermat)}
  وديكارت
 \textenglish{(Descartes)}
    إنجازاً عظيماً بتقديم النظام الإحداثي من أجل وصف العلاقات الهندسية. هذا ما سمح بإستعمال الجبر والهندسة في آن واحد.
\\
\vspace{0.4cm}
برنامج SURFER هو أحسن مثال عن هذه العلاقة إذ أنه يخلق الهندسة (الصورة) إنطلاقاً من الجبر (الصيغة).
في هذه الجاليريا، يمكنك اكتشاف جمال الرياضيات واستعمال خيالك للإبداع. يمكن إختيار إحدى السطوح على اليمين. العلاقة الرياضية بين الصيغة والشكل مشروحة بشكل بسيط من خلال سلسة من الأمثلة.\\
اتبعوا حدسكم وأطلقو العنان لمخيلتكم \dots
\end{surferIntroPage}
