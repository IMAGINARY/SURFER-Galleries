\begin{surferPage}[طائر الطنان]{طائر الطنان }
المعادلة هي التي تقرر النقاط\\
  
  \smallskip
\[z^3+ y^2	z^2	= x^2\]

\singlespacing
بلغة الجبر، طائر الطنان مُعطىً بكافة النقاط التي تحقق المعادلة
\smallskip
\[ x^2= y^2z^2+z^3.\]
\smallskip
على سبيل المثال، النقطة
 $(0,0,0)$
  والنقطة  
  $(1,0,1)$
   والنقطة
 $(3,-2,-3)$ 
     نقاط من طائر الطنان بينما نقطة
     $(0,1,1)$
      لا تنتمي إليه.\\
 \singlespacing
عالمنا الثلاثي الأبعاد تديره ثلاثة اتجاهات: إلى الأمام أو الوراء، إلى الشمال أو إلى اليمين، إلى الأعلى أو إلى الأسفل. يتم تحديد هذه الإتجاهات بواسطة الإحداثيات $x$ و$y$ و$z$. يمكن وصف كل نقطة في الفضاء بإعطاء قيمة لكل اتجاه من هذه الإتجاهات. هذه القيم هي إحداثيات
 $(x,y,z)$.
  \\
\singlespacing
نُدخل الآن كافة نقاط الفضاء في الصيغة ونلون فقط تلك التي تحقق المعادلة. تشكل كل هذه النقاط الملونة الصورة.
\end{surferPage}
