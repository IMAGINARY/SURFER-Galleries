\begin{surferPage}[بزباز]{بزباز }
 كلمة مع عدد لا متناهٍ من الأحرف \\
\smallskip
\[y z (x^2	+ y - z) = 0\]

\vspace{0.3cm}
رسم الرسامون الإنطباعيون المنازل والمروج مع آلاف النقاط الملونة. بصورة مماثلة، تتشكل السطوح الرياضية من آلاف النقاط، ولكنها نقاط بدون عرض وبدون  ثقل وإنما نقاط حلول للمعادلة! \\

\vspace{0.3cm}
إحدى الطرق لتخيل اللانهاية هي بالإبتداء بالعد:
$ 1,2,3 $ \dots
\\
مهما تقدمنا بالعد، سيكون هناك دائماً رقم أكبر ولن نتكمن أبداً من العد حتى النهاية.
\vspace{0.3cm}
بالإضافة إلى كون السطح الجبري مكوناً من عدد لا متناهٍ من النقط، فقط بين $0$ و$1$، هناك عدد لامتناهٍ. أيبدو ذلك مستحيلاً؟ عليك أن ترى النقاط على أنها صغيرة بصورة غير متناهية تم رسمها بواسطة قلم رصاص سماكته صفر. يجب رسم الكثير منها لملء الخط بين $0$ و$1$، عدد لامتناهٍ منها في الواقع.
\end{surferPage}
