\begin{surferPage}{Nepali}
Een wereld die nooit eindigt \\

\smallskip
\[(x y - z^3 -1)^2= (1 - x^2 - y^2)^3\]

\singlespacing
Zou het niet mooi zijn als je elk willekeurig oppervlak zomaar in een kristallen sneeuwbol zou kunnen vatten en in je woonkamer zetten\dots De droom van elke wiskundige!
\\
Helaas is dit gewoon onmogelijk met bepaalde oppervlakken!
\\
\singlespacing
Er bestaan oppervlakken die zich oneindig ver uitstrekken. En ook al zijn ze bijzonder mooi, je zal ze nooit in een sneeuwbol kunnen vatten, hoe groot die ook is. Als dit het geval is noemen we het oppervlak \textit{onbegrensd}. Om zo'n oppervlak te kunnen tekenen moeten we delen ervan verbergen.
\\
\singlespacing
Of een oppervlak begrensd is, is niet gemakkelijk te zien, zelfs niet met SURFER. Het is net alsof we willen weten of het universum begrensd is of niet: aangezien we de grenzen niet kennen, weten we niet of ze er al dan niet zijn.
\end{surferPage}
