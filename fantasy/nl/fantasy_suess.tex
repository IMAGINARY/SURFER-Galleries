\begin{surferPage}{Lief}
\[(x^2+ 9/4y^2	+ z^2- 1)^3- x^2z^3	- 9/80y^2z^3	= 0\]

\singlespacing
Liefdesbrief
\singlespacing
So kann es nun nicht weitergehn! \\
Das, was besteht, muß bleiben. \\
Wenn wir uns wieder wiedersehn, \\
Muß irgendetwas geschehn. \\
Was wir dann auf die Spitze treiben.\\ 
Was - was auf einer Spitze tut? \\
Gewiß nicht Plattitüden. \\
Denn was auf einer Spitze ruht, \\
Wird nicht so leicht ermüden. \\
Auf einer Bank im Grunewald \\
Zu zweit im Regen sitzen, \\
Ist blöd. Mut, Mädchen! Schreibe bald! \\
Dein Fritz! (Remember Spitzen). \\
 {\it Gedicht van Joachim Ringelnatz}
\singlespacing 
De passie van de liefde wordt meestal in verband gebracht met de emotionele kracht van een ``singulariteit''. Dit verband komt tevoorschijn in vele kunstvormen.
\singlespacing 
Probeer de laatste $z^3$ van deze vergelijking eens te vervangen door een kwadraat en bekijk wat er gebeurt.
\end{surferPage}
