\begin{surferIntroPage}{Fantasie-oppervlakken}{fantasy_kolibri}{De Fantasie-oppervlakken}
We horen vaak hoe ingewikkeld wiskunde wel is, maar het blijft een feit dat wiskunde helpt om onze complexe wereld te begrijpen. Dit gebeurt bijvoorbeeld door fundamentele structuren en belangrijke gemeenschappelijke kenmerken in re\"ele objecten te herkennen. Het verzamelen van alle objecten die bepaalde belangrijke kenmerken gemeenschappelijk hebben in \'e\'en \textit{klasse}, waarbij we minder belangrijke eigenschappen negeren, wordt \textit{classificatie} genoemd. Dit is \'e\'en van de belangrijkste methodes om een overzicht te krijgen van de schier oneindig diverse objecten en vormen van deze wereld, en hiervoor is de wiskunde onontbeerlijk. Wat belangrijk is en wat niet hangt af van wat je wil bestuderen; dit kan bijvoorbeeld de grootte of de vorm van een object zijn.
\\

\vspace{0.4cm}

Het beschrijven en classificeren van bepaalde vormen is een oud probleem, maar hoe je dit moet aanpakken is niet meteen duidelijk. De Oude Grieken gebruikten vooral meetkunde en eigenschappen van meetkundige objecten. Later ontwikkelden de Arabieren de algebra (Al Khwarizmi, 900 n.C.). Een grote doorbraak kwam er in de 18de eeuw, toen Fermat en Descartes het co\"ordinatenstelsel introduceerden, en daarmee een belangrijk verband tussen meetkunde en algebra introduceerden.
\\
\vspace{0.4cm}
Het SURFER-programma is hier een prima voorbeeld van, omdat het iets meetkundigs (de afbeelding) cre\"eert uit iets algebra\"isch (de vergelijking).
In deze galerij kan je de schoonheid van de wiskunde ontdekken en je eigen creativiteit botvieren. Kies \'e\'en van de oppervlakken rechts. Het wiskundige verband tussen de vergelijking en de vorm die je ziet wordt eenvoudig uitgelegd met een aantal voorbeelden.\\
De creatieve ingevingen moet je zelf aanbrengen \dots
\end{surferIntroPage}
