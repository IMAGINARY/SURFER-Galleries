\begin{surferPage}{Citrus}
{\it Ceci n'est pas un citron} - hoe beelden je kunnen misleiden\\
\smallskip
\[x^2 + z^2 = y^3 (1 - y)^3\] 


\singlespacing
Op het eerste zicht denk je bij deze afbeelding ongetwijfeld: ``Dat is een citroen''. Maar als het een citroen is, waarom ruik of proef je dan niets? Waarom heeft hij geen plekken? Het kan dus duidelijk geen echte citroen zijn! 
\singlespacing
Deze vorm is geen citroen, maar een wiskundig model ervan. Ze helpt ons om de eigenschappen van de citroenvorm beter te begrijpen. Er bestaat een passende quote uit de geografie van $Alfred\ H.\ S.\ Korzybski$: ``De kaart is het terrein niet.'' \\
\singlespacing

Vergelijkingen laten toe om wiskundige modellen te bouwen die ons helpen om bepaalde vormen beter te begrijpen. 
\singlespacing
Het is een prachtig voorbeeld van wiskundige po\"ezie: de mooie oppervlakken die we met algebra\"ische vergelijkingen tevoorschijn kunnen toveren, laten ons toe om onbekende kanten van onze verbeeldingen te ontdekken\dots
\end{surferPage}
