\begin{surferPage}{Hemel en hel}
We maken nieuwe vormen \\
\smallskip
\[x^2	- y^2z^2	= 0\]

\singlespacing
Om nieuwe vormen te cre\"eren moeten we weten hoe de vergelijkingen in elkaar zitten. De ingredi\"enten zijn de de zogenaamde {\it eentermen}, algebra\"ische uitdrukkingen met letters en getallen.
\singlespacing
Uit een eenterm kunnen we de volgende elementen halen:
het teken, co\"effici\"enten, variabelen, exponenten en de graad.\\
\singlespacing
Bijvoorbeeld: 
\smallskip
\[2xy^2z = +2x^1y^2z^1.\]
\\
\smallskip
De  {\it graad} van een eenterm is de som van de exponenten van haar variabelen. In het voorbeeld hierboven: $graad = 1 + 2 + 1 = 4$.  \\
\singlespacing
Om vergelijkingen te vormen gebruiken we rekenkundige bewerkingen zoals de optelling, aftrekking en vermenigvuldiging. Dit zijn bewerkingen die we al van de lagere school kennen, en waarmee we alle algebra\"ische oppervlakken weer kunnen geven.
\singlespacing
Kan je vormen maken met gaten en pieken door enkel op te tellen en te vermenigvuldigen?
\end{surferPage}
