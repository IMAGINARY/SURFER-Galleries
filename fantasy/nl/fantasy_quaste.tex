\begin{surferPage}{Quaste}
Het ABC van vergelijkingen
  \smallskip
\[8z^9-24x^2z^6-24y^2z^6+36z^8+24x^4z^3-168x^2y^2z^3\]
\[+24y^4z^3-72x^2z^5-72y^2z^5+54z^7-8x^6-24x^4y^2\]
\[-24x^2y^4-8y^6 + 36x^4z^2-252x^2y^2z^2+36y^4z^2\]
\[- 54x^2z^4-108y^2z^4 + 27z^6-108x^2y^2z + 54y^4z\]
\[-54y^2z^3 + 27y^4 = 0\]\\
\vspace{0.3cm}
Heb je al eens goed gekeken naar de vergelijking van Quaste? Die ziet er erg ingewikkeld uit.
De figuur zelf kan eenvoudig beschreven worden: de bovengrens heeft de vorm van de Griekse letter $\alpha$, de grens rechts heeft de vorm van een kromme met een piek. Zo'n piek wordt een {\it keerpunt} genoemd. Als je het keerpunt langs de alpha-kromme laat lopen krijg je Quaste. Oppervlakken met een dergelijke eigenschap worden Cartesische producten genoemd, ter ere van de Franse wiskundige Ren\'e Descartes.\\
\vspace{0.3cm}
De eentermen van graad $1$ zijn $x$, $y$, $z$. De eentermen van graad $2$ zijn $x^2, xy, y^2, xz, yz, z^2$, enzovoort. Hoe hoger de graad, hoe meer eentermen er zijn, en dit geeft ons meer mogelijkheden om ingewikkelde vormen te maken. Het is net als een alfabet: als we meer letters ter beschikking hebben, kunnen we moeilijkere woorden en zinnen schrijven.
\end{surferPage}
