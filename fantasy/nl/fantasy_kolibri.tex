\begin{surferPage}{Kolibrie}
De vergelijking bepaalt de punten\\
  
  \smallskip
\[z^3+ y^2	z^2	= x^2\]

\singlespacing
In algebra\"ische termen wordt de Kolibrie gegeven door alle punten $(x, y, z)$ die voldoen aan de vergelijking
\smallskip
\[ x^2= y^2z^2+z^3.\]
\smallskip
Zo zijn bijvoorbeeld $(0,0,0),$ $(1,0,1)$ en $(3,-2,-3)$ punten van de Kolibrie, maar $(0,1,1)$ niet.\\
 \singlespacing
Onze driedimensionale wereld wordt bepaald door drie richting: vooruit en achteruit, links en rechts, opwaarts en neerwaarts. Deze richtingen worden voorgesteld door de variabelen $x$, $y$ and $z$. Elk punt in de ruimte kan beschreven worden met een waarde voor $x$, $y$ en $z$. Deze waarden worden de co\"ordinaten $(x,y,z)$ van het punt genoemd.\\
\singlespacing
We vullen nu de waarden van alle mogelijke punten in de vergelijking in en kleuren enkel die punten waarvoor de vergelijking voldaan is. De gekleurde punten vormen dan samen de afbeelding rechts.
\end{surferPage}
