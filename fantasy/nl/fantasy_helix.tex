\begin{surferPage}{Helix}
Nog dunner dan zeepbellen\\
  \smallskip
\[6x^2	= 2x^4	+ y^2	z^2\]

\singlespacing
Zeepbellen zijn heel gevoelig; ze lijken al uiteen te spatten als je er nog maar naar kijkt. Hun oppervlak heeft twee kanten: aan de buitenkant bevindt zich de zeep en aan de binnenkant het water. Als de zeeplaag te dun wordt - wat gebeurt als de bel groter wordt - laat het water de bel uiteenspatten.\\
\vspace{0,3cm}
Algebra\"ische oppervlakken zijn nog veel dunner dan zeepbellen, want ze bestaan enkel uit laagjes punten. En omdat we enkel onze verbeelding gebruiken om deze punten te cre\"eren, zonder massa of dichtheid, barsten ze niet, zelfs als ze plooien of scherpe kanten hebben zoals de Helix.\\
\vspace{0,3cm}
Echter, als we het Helix-oppervlak als driedimensionaal model willen voorstellen moeten we iets bouwen dat dikker is dan het echte Helix-oppervlak. Dit kunnen we doen door het oppervlak langs \'e\'en kant te versterken.
\end{surferPage}
