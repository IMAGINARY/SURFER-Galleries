\begin{surferPage}{Vis-\`a-Vis}
Singulier of regulier - vriend of vijand\\
\smallskip
\[x^2	- x^3+ y^2+ y^4+ z^3- z^4	=  0\]

\vspace{0.3cm}
Singuliere punten, of singulariteiten, worden visueel duidelijk gemaakt omdat het oppervlak niet glad of zacht is, bijvoorbeeld bij een keerpunt of een plooi.\\
\vspace{0.3cm}
Het Vis-\`a-Vis-oppervlak toont het verschil tussen singuliere en reguliere punten heel duidelijk: de scherpe piek aan de ene kant is een singulariteit. De gladde heuvel aan de overkant, daarentegen, is een regulier punt. Singulariteiten zijn interessant omdat kleine veranderingen in de vergelijking het gedrag rond deze punten verrassend kunnen veranderen. \\

\vspace{0.3cm}
Wist je dat er mensen zijn die zich specifiek met de studie van deze punten bezighouden? Zwarte gaten en de Big Bang vormen singulariteiten van de vergelijkingen van het kosmologisch model. En kijk eens naar je vingertoppen; de singulariteiten van onze vingerafdrukken onderscheiden ons van elkaar!
\end{surferPage}
