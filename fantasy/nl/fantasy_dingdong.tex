\begin{surferPage}{Dingdong}
Verander de figuur door de vergelijking te veranderen\\

\smallskip
\[x^2	+ y^2	+ z^3	= z^2\]

\singlespacing
De vergelijking en de vorm van Dingdong zijn eenvoudig. De figuur verkrijgen we door de Griekse letter $\alpha$ rond de $z$-as te wentelen. Als je er ondersteboven naar kijkt, lijkt Dingdong net op een waterdruppel die op het punt staat te vallen.
\newline
Door een kleine parameter $a$ aan de vergelijking toe te voegen en ze continu te veranderen, kunnen we een reeks beelden cre\"eren die tonen hoe de druppel opduikt, hoe ze haar eindpositie nadert en zich uiteindelijk afsplitst. Het zijn net afzonderlijke frames van een film.

\[x^2	+ y^2	+ z^3	-z^2+0.1\cdot a=0.\]

Op elk moment bevindt de druppel zich in een evenwichtstoestand, waar zwaartekracht de oppervlaktespanning compenseert. Maar dit evenwicht is niet stabiel, en de druppel trilt tot hij valt. De catastrofetheorie van de wiskundige Ren\'e Thom bestudeert hoe kleine veranderingen in de parameters onmiddellijke veranderingen in het evenwicht teweegbrengen.
\end{surferPage}
