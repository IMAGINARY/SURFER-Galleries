\begin{surferPage}{Tuit}
Een woord met oneindig veel letters\\
\smallskip
\[y z (x^2	+ y - z)	= 0\]

\vspace{0.3cm}
De impressionisten schilderden hun huizen en velden met duizenden gekleurde punten. Op gelijkaardige manier worden ook wiskundige oppervlakken gevormd door duizenden punten. Deze punten hebben echter geen massa of afmetingen; ze voldoen simpelweg aan een vergelijking!\\
\vspace{0.3cm}
Een manier om oneindigheid voor te stellen is door te beginnen tellen: $1, 2, 3,$ \dots\\
Dit tellen houdt nooit op; hoe ver we ook gaan, altijd is er wel een groter getal te vinden.\\
\vspace{0.3cm}
Maar niet enkel dit oppervlak bevat oneindig veel punten. Op de getallenas liggen tussen de punten $0$ en $1$ al oneindig veel andere punten. Lijkt dit onmogelijk? Bedenk dan gewoon dat onze punten oneindig klein zijn. Ze worden als het ware getekend met een potlood van dikte nul. Om het lijnstuk tussen $0$ en $1$ te vullen heb je er al heel veel nodig, namelijk oneindig veel.
\end{surferPage}
