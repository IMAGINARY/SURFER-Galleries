\begin{surferPage}{Paradis et Enfer}
Nous créons de nouvelles formes \\
\smallskip
\[x^2	- y^2z^2	= 0\]

\singlespacing
Pour créer de nouvelles formes, il faut comprendre le mécanisme des équations. Elles sont constituées de {\it monômes}, des expressions algébriques avec des lettres et des nombres.
\singlespacing
Un monôme peut contenir les éléments suivants :
Signes, coefficients, variables, exposants et le degré.\\
\singlespacing
Par exemple : 
\smallskip
\[2xy^2z = +2x^1y^2z^1.\]
\\
\smallskip
Le {\it degré} d'un monôme est la somme des exposants de ses variables : $degré = 1+2+1=4$.  \\
\singlespacing
Pour former les équation, on utilise les opérations telles que l'addition, la soustraction et la multiplication. Nous connaissons ces opérations depuis l'école primaire. Elles sont utilisées pour former toutes les surfaces algébriques.
\singlespacing
Pouvez-vous créer des formes avec des pointes et des trous, juste en utilisant l'addition et la multiplication ?
\end{surferPage}
