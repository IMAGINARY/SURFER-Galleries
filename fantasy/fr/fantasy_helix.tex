\begin{surferPage}{Helix}
Plus fin qu'un film de savon\\
  \smallskip
\[6x^2	= 2x^4	+ y^2	z^2\]

\singlespacing
Les bulles de savon sont fragiles; elles semblent éclater juste en les regardant. Leur surface a deux faces. Le savon est sur la face extérieure et l'eau sur celle intérieure. Si la couche de savon devient trop mince - ce qui arrive si la bulle grandit trop -, l'eau fait éclater la bulle.\\
\vspace{0,3cm}
Les surfaces algébriques sont bien plus fines que les films de savon, elles sont seulement faites d'une couche de points. Et comme nous n'utilisons que notre imagination pour créer ces points n'ayant ni masse, ni densité, elles n'éclatent pas, même si elles ont des pointes et des plis comme Helix.\\
\vspace{0,3cm}
Si l'on veut toutefois créer un modèle tridimensionnel de la surface Helix, on doit en faire une sculpture plus épaisse que la vraie surface. On peut pour cela renforcer une de ses faces.
\end{surferPage}
