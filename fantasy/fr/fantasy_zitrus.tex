\begin{surferPage}{Zitrus}
Ceci n'est pas un citron - la traîtrise des images\\
\smallskip
\[x^2 + z^2 = y^3 (1 - y)^3\] 


\singlespacing
On peut sans doute penser au premier regard sur cette image : ``C'est un citron''. Mais alors, si c'est un citron, pourquoi n'a t-il ni odeur, ni goût ? Et pourquoi n'y a t-il pas de pores ou de taches ? Ce n'est certainement pas un citron !
\singlespacing
Cette forme est seulement un modèle mathématique du citron. Il nous aide à mieux saisir les propriétés de la forme du citron. En géographie, une citation appropriée est due à $Alfred\ H.\ S.\ Korzybski$: ``La carte n'est pas le territoire.'' \\
\singlespacing

Les équations nous permettent de bâtir des modèles mathématiques nous aidant à mieux appréhender la forme des objets. 
\singlespacing
Cela fait partie de la poésie des mathématiques : on peut générer de belles surfaces au moyen d'équations algébriques qui amènent notre réflexion dans des recoins insoupçonnés de notre esprit.
\end{surferPage}
