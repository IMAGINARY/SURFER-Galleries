\begin{surferPage}{Colibri}
L'équation dessine les points\\
  
  \smallskip
\[z^3+ y^2	z^2	= x^2\]

\singlespacing
En langage algébrique, le Colibri est l'ensemble des points $(x, y, z)$ qui satisfont l'équation
\smallskip
\[ x^2= y^2z^2+z^3.\]
\smallskip
Par exemple, $(0,0,0),$ $(1,0,1)$ et $(3,-2,-3)$ font partie du Colibri, au contraire du point $(0,1,1)$.\\
 \singlespacing
 Notre espace de dimension trois obéit à trois directions : de l'avant à l'arrière, de gauche à droite, de haut en bas. Ces directions sont repérées par $x$, $y$ et $z$. Tout point de l'espace peut être décrit par une valeur dans chacune de ces trois directions. Ces valeurs sont les coordonnées $(x,y,z)$ de ce point.\\
\singlespacing
On rentre maintenant tous les points de l'espace dans l'équation, et on colorie seulement ceux pour lesquels l'équation est satisfaite. Tous les points en couleur forment alors l'image de la surface.
\end{surferPage}
