\begin{surferPage}{Tique}
L'équation, pour nommer sans ambiguïté \\
\smallskip
\[x^2 + y^2	= z^3	(1 - z) \]


\singlespacing
Toutes les figures de cette galerie portent des noms. Comment les auriez-vous appelées ? Comment une autre personne les aurait-elles nommées ?\\
\vspace{0.3cm}
Existe t-il un moyen de nommer des formes sans entraîner de confusion ? Les mathématiques ont trouvé une solution : en les appelant par leur équation. Celle-ci détermine tous ses points, les courbes, les trous, les plis et les pointes. Il vous suffit de savoir où trouver ces formes dans la formule et de les dessiner.\\
\vspace{0.3cm}
Les équations sont écrites et interprétées de la même manière dans le monde entier car, à l'instar des partitions musicales, le langage mathématique est universel.
\end{surferPage}
