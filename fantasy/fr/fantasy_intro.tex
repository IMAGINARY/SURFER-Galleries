\begin{surferIntroPage}{Surfaces Fantasy}{fantasy_kolibri}{Les surfaces Fantasy}
On entend souvent dire à quel point les mathématiques sont compliquées, mais elles nous servent en fait à comprendre la complexité du monde qui nous entoure. Par exemple, par la détection de structures fondamentales et de propriétés communes importantes des objets réels. Le regroupement dans une même \textit{classe} de tous les objets ayant des propriétés importantes identiques en ignorant les moins importantes est la classification. C'est l'un des moyens essentiels d'apprécier dans son ensemble l'infinie diversité des objets et des formes environnants. Les mathématiques sont fondamentales pour cela. Ce qui est important ou non dépend de ce que l'on veut explorer. Cela peut par exemple être la taille ou la forme d'un objet.
\\

\vspace{0.4cm}

Le besoin humain de décrire et classifier les formes est ancien, le faire n'est pas évident pour autant. Les grecs anciens utilisaient essentiellement la géométrie et les proportions des objets géométriques. L'algèbre fut développée plus tard principalement par les Arabes (Al Khwarizmi, 800 ap. J.-C.). Au 17è siècle, l'introduction d'un système de coordonnées pour décrire des relations géométriques fut une grande avancée de Fermat et de Descartes. Il devint dès lors possible d'utiliser en même temps l'algèbre et la géométrie.
\\
\vspace{0.4cm}
Le programme SURFER en est une parfaite illustration puisqu'il crée de la géométrie (l'image) à partir d'algèbre (la formule).
Dans cette galerie, vous pouvez apprécier la beauté des mathématiques et devenir vous-même créatif. Choisissez une des surfaces à droite. Le lien mathématique entre les formules et les formes est expliqué de manière simple, à l'aide d'une série d'exemples.\\
\`A vous de laisser faire votre imagination et votre intuition \dots
\end{surferIntroPage}
