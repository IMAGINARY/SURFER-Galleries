\begin{surferPage}{Vis-à-vis}
Singulier ou régulier - ami ou ennemi\\
\smallskip
\[x^2	- x^3+ y^2+ y^4+ z^3- z^4	=  0\]

\vspace{0.3cm}
Les points singuliers, ou singularités, sont visuellement reconnaissables lorsque la surface n'est pas lisse, par exemple en ayant une pointe ou un pli.\\
\vspace{0.3cm}
La pointe à gauche de la surface Vis-\`a-vis est une singularité; au contraire, la bosse lisse sur la droite marque un point régulier. Les  singularités sont fascinantes car un petit changement dans l'équation peut faire varier leur apparence de manière surprenante. \\

\vspace{0.3cm}
Savez vous que des gens concentrent leurs recherches sur ces points ? Les trous noirs et le Big-Bang constituent des singularités pour les équations du modèle cosmologique. Regardez maintenant le bout de vos doigts. Les singularités de nos empreintes digitales nous identifient !
\end{surferPage}
