\begin{surferPage}{Douceur}
\[(x^2+ 9/4y^2	+ z^2- 1)^3- x^2z^3	- 9/80y^2z^3	= 0\]

\singlespacing
Love-letter
\singlespacing
It can`t go on and on like this!\\
This what we have we have to keep.\\
When we'll meet,\\
Something must be.\\
Together singularity.\\
What`s up on singularity?\\
For sure no platitudes.\\
Because what's singularity,\\
Is not wearing down so easily.\\
On a park bench in Grunewald\\
In two to face the rain,\\
Attempts in vain. Girl, take heart! Write me a card!\\
Yours, Bertie! (Erinner Singularity).\\
{\it Poem by Joachim Ringelnatz}
\singlespacing 
Le sentiment amoureux est généralement relié au pouvoir émotionnel d'une ``singularité''. Cette connexion apparaît dans l'Art sous de nombreuses formes.
\singlespacing 
Remplacez le dernier exposant 3 de l'équation par un exposant 2, et regardez ce qu'il se produit.
\end{surferPage}
