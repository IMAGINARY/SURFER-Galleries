\begin{surferPage}{Quaste}
L'ABC des équations
  \smallskip
\[8z^9-24x^2z^6-24y^2z^6+36z^8+24x^4z^3-168x^2y^2z^3\]
\[+24y^4z^3-72x^2z^5-72y^2z^5+54z^7-8x^6-24x^4y^2\]
\[-24x^2y^4-8y^6 + 36x^4z^2-252x^2y^2z^2+36y^4z^2\]
\[- 54x^2z^4-108y^2z^4 + 27z^6-108x^2y^2z + 54y^4z\]
\[-54y^2z^3 + 27y^4 = 0\]\\
\vspace{0.3cm}
Avez-vous regardé de près l'équation de Quaste ? Elle semble bien compliquée.
La figure elle-même peut être décrite en des termes simples : le bord haut a la forme de la lettre grecque $\alpha$, le bord droit celle d'une courbe avec une pointe, dite {\it cuspide}. En promenant une cuspide le long de la courbe en alpha,on obtient Quaste. Une surface ayant une telle propriété est appelée produit Cartésien, en l'honneur de René Descartes.\\
\vspace{0.3cm}
Les monômes de degré $1$ sont $x$, $y$ et $z$. Ceux de degré $2$ sont $x^2, xy, y^2, xz, yz, z^2$. Et ainsi de suite. Plus le degré est grand, plus il y a de monômes, ce qui fournit plus de possibilités pour créer des formes compliquées. Par analogie avec un alphabet : plus on a de lettres à disposition, plus on peut écrire des mots et des phrases complexes.
\end{surferPage}
