\begin{surferPage}{Bec verseur}
Un mot à une infinité de lettres\\
\smallskip
\[y z (x^2	+ y - z)	= 0\]

\vspace{0.3cm}
Les impressionnistes ont peint des maisons et des jardins avec des milliers de points de couleur. De même, les surfaces mathématiques sont formées de milliers de points. Ceux-ci n'ont ni taille ni masse mais ils sont solutions de l'équation ! \\
\vspace{0.3cm}
Une manière d'appréhender l'infini est de compter : $1, 2, 3,$ \dots\\
Aussi loin que nous comptons, il y aura toujours un nombre plus grand et nous n'arriverons jamais au bout.\\
\vspace{0.3cm}
Non-seulement la surface contient une infinité de points, mais il y en a déjà une infinité rien qu'entre $0$ et $1$. Impossible ? Pensez que les points sont infiniment petits et sont dessinés avec un crayon d'épaisseur nulle. Il faut alors en dessiner beaucoup pour remplir le segment entre $0$ et $1$, en fait une infinité.
\end{surferPage}
