\begin{surferPage}{Ding Dong}
Modifiez la figure en modifiant l'équation\\

\smallskip
\[x^2	+ y^2	+ z^3	= z^2\]

\singlespacing
L'équation et la forme de Ding Dong sont simples. La figure est obtenue en faisant tourner la lettre grecque Alpha autour de son axe. En l'observant à l'envers, Ding Dong ressemble à une goutte d'eau. On peut regarder la goutte tomber.
\newline
Si l'on ajoute un paramètre $a$, petit, dans l'équation et qu'on le fait varier continument, on peut créer une série d'images montrant l'apparition de la goutte, comment elle se rapproche de sa position finale et finit par se détacher, comme dans un arrêt sur images d'un film :

\[x^2	+ y^2	+ z^3	-z^2+0.1\cdot a=0.\]

La goutte est à chaque instant en situation d'équilibre où la gravité et la tension superficielle se compensent. Mais l'équilibre de la goutte est instable et elle se met à trembler avant de chuter. La théorie des Catastrophes, du mathématicien René Thom, étudie la façon dont une petite variation des paramètres entraîne une modification immédiate de l'équilibre.
\end{surferPage}
