\begin{surferPage}{Frente a Frente}
Singular ou regular - amigo ou inimigo\\
\smallskip
\[x^2	- x^3+ y^2+ y^4+ z^3- z^4	=  0\]

\vspace{0.3cm}
Pontos singulares, ou singularidades, s\~ao identificados visualmente porque a superf\'icie nesses pontos n\~ao \'e lisa ou suave, como,  por exemplo,  numa c\'uspide ou numa dobra.\\
\vspace{0.3cm}
A c\'uspide do lado esquerdo do Frente a Frente \'e uma singularidade; contudo, o monte suave da direita j\'a \'e um ponto regular. Singularidades s\~ao interessantes porque pequenas mudan\c cas na equa\c c\~ao podem mudar a sua apar\^encia de uma forma surpreendente. \\

\vspace{0.3cm}
Sabia que existem pessoas que se dedicam especialmente a estudar esses pontos? Os buracos negros e o Big Bang constituem singularidades das equa\c c\~oes do modelo cosmol\'ogico. E agora olhe para a ponta dos seus dedos: as singularidades das nossas impress\~oes digitais identificam-nos!
\end{surferPage}
