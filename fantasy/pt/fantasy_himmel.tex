\begin{surferPage}{Céu e Inferno}
N\'os criamos novas formas \\
\smallskip
\[x^2	- y^2z^2	= 0\]

\singlespacing
Para criarmos novas formas temos que entender como as equa\c c\~oes funcionam. Os seus elementos s\~ao os chamados  {\it mon\'omios}, express\~oes alg\'ebricas com letras e n\'umeros.
\singlespacing
Um mon\'omio pode conter os seguintes elementos:
sinais, coeficientes, vari\'aveis, expoentes e grau.\\
\singlespacing
Por exemplo: 
\smallskip
\[2xy^2z = +2x^1y^2z^1.\]
\\
\smallskip
O  {\it grau} de um mon\'omio \'e a soma dos expoentes das suas vari\'aveis: $grau = 1+2+1=4$.  \\
\singlespacing
Para formar equa\c c\~oes, utilizamos opera\c c\~oes aritm\'eticas como a adi\c c\~ao, a subtra\c c\~ao e a multiplica\c c\~ao. Estas s\~ao as opera\c c\~oes que conhecemos desde a escola prim\'aria. Elas s\~ao utilizadas para exibir todas as superf\'icies alg\'ebricas.
\singlespacing
Consegue criar formas com picos e buracos, utilizando somente a adi\c c\~ao e a multiplica\c c\~ao?
\end{surferPage}
