\begin{surferPage}{Drapeado}
O ABC das equa\c c\~oes
  \smallskip
\[8z^9-24x^2z^6-24y^2z^6+36z^8+24x^4z^3-168x^2y^2z^3\]
\[+24y^4z^3-72x^2z^5-72y^2z^5+54z^7-8x^6-24x^4y^2\]
\[-24x^2y^4-8y^6 + 36x^4z^2-252x^2y^2z^2+36y^4z^2\]
\[- 54x^2z^4-108y^2z^4 + 27z^6-108x^2y^2z + 54y^4z\]
\[-54y^2z^3 + 27y^4 = 0\]\\
\vspace{0.1cm}
Repare na equa\c c\~ao do Drapeado. Parece muito complicada. A figura em si pode ser descrita de uma forma simples:
a borda superior desta superf\'icie \'e um la\c co com a forma da letra grega $\alpha$, a borda direita tem a forma de uma curva pontiaguda. Tal pico \'e chamado {\it c\'uspide}. Deslocando
a curva pontiaguda ao longo da curva em forma de alfa, obtemos o Drapeado. As superf\'icies com esta propriedade s\~ao
designadas  ''produtos cartesianos'', em homenagem ao matem\'atico franc\^es  Ren\'e Descartes.\\
\vspace{0.3cm}
Mon\'omios de grau $1$ s\~ao $x$, $y$, $z$. Mon\'omios de grau $2$ s\~ao $x^2, xy, y^2, xz, yz, z^2$. E assim por diante. Quanto maior for o grau, mais mon\'omios temos, e isso oferece-nos mais possibilidades para criarmos formas mais complicadas. \'E como um alfabeto: se  temos mais letras \`a nossa disposi\c c\~ao, conseguimos escrever palavras e frases mais complexas.
\end{surferPage}
