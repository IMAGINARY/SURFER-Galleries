\begin{surferPage}{Beija-Flor}
A equa\c c\~ao escolhe os pontos\\
  
  \smallskip
\[z^3+ y^2	z^2	= x^2\]

\singlespacing
Em termos alg\'ebricos, a superf\'icie Beija-Flor \'e definida por todos os pontos $(x, y, z)$ que satisfazem a equa\c c\~ao:
\smallskip
\[ x^2= y^2z^2+z^3.\]
\smallskip
Por exemplo, $(0,0,0),$ $(1,0,1)$ e $(3,-2,-3)$ s\~ao pontos do Beija-Flor, enquanto que $(0,1,1)$ n\~ao faz parte dela.\\
 \singlespacing
 O nosso mundo tridimensional \'e governado por tr\^es dire\c c\~oes: para frente e para tr\'as, esquerda e direita, para cima e para baixo. Estas dire\c c\~oes s\~ao identificadas com $x$, $y$ e $z$. Cada ponto no espa\c co pode ser descrito por um valor para cada uma das suas dire\c c\~oes. Estes valores dizem-se coordenadas $(x,y,z)$ desse ponto.\\
\singlespacing
Agora colocamos todos os pontos do espa\c co com os seus valores na equa\c c\~ao e colorimos somente aqueles nos quais a equa\c c\~ao \'e satisfeita. Todos os pontos coloridos juntos formam a imagem.
\end{surferPage}
