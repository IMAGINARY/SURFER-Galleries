\begin{surferPage}{Nepal}
Um mundo sem fim \\

\smallskip
\[(x y - z^3 -1)^2= (1 - x^2	- y^2)^3\]

\singlespacing
Imagine que encontra uma superf\'icie muito bonita e pretende coloc\'a-la numa bola de cristal com neve para brincar com ela.  N\~ao pense que  pode levar qualquer superf\'icie para a sua sala de estar!
\\
\singlespacing
Existem superf\'icies que se estendem at\'e ao infinito e, mesmo que elas sejam muito bonitas, nunca iremos ser capazes de as colocar numa bola de cristal, n\~ao importa o tamanho da bola de cristal. Chamamos a essas superf\'icies \textit{n\~ao restritas}. Para representar essas superf\'icies temos que esconder parte delas.
\\
\singlespacing
N\~ao \'e f\'acil reconhecer se uma superf\'cie \'e ou n\~ao  restrita, nem mesmo com a ajuda do SURFER. 
\'E como se quis\'essemos saber se o Universo \'e limitado: uma vez que  n\~ao sabemos as suas fronteiras, ele pode ter algumas, ou n\~ao.
\end{surferPage}
