\begin{surferPage}{Gota}
Alterar a figura alterando a equa\c c\~ao\\

\smallskip
\[x^2	+ y^2	+ z^3	= z^2\]

\singlespacing
A equa\c c\~ao e a forma da Gota s\~ao simples. A figura \'e obtida ao rodar a letra grega alfa em torno do seu eixo. Se  olhamos para ela de cabe\c ca para baixo, a Gota parece-se com uma gota de \'agua. N\'os conseguimos ver a gota a cair.
\newline
Se adicionarmos um pequeno par\^ametro $a$ \`a equa\c c\~ao e se o alterarmos continuamente, podemos criar uma s\'erie de imagens que mostram o surgimento da gota, como ela se aproxima da sua posi\c c\~ao final e como, finalmente, se separa. S\~ao como imagens est\'aticas de um filme:

\[x^2	+ y^2	+ z^3	-z^2+0.1\cdot a=0.\]

Em cada momento a gota est\'a numa situa\c c\~ao de equil\'ibrio onde a gravidade compensa a tens\~ao superficial. Mas o equil\'ibrio da gota n\~ao \'e est\'avel e ela treme antes de cair. A teoria das cat\'astrofes da autoria do matem\'atico Ren\'e Thom estuda como pequenas altera\c c\~oes nos par\^ametros podem causar altera\c c\~oes imediatas no equil\'ibrio.
\end{surferPage}
