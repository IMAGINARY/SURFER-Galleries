\begin{surferPage}{Citrus}
Isto n\~ao \'e um lim\~ao - a trai\c c\~ao das imagens\\
\smallskip
\[x^2 + z^2 = y^3 (1 - y)^3\] 

\singlespacing
N\~ao h\'a d\'uvida que quando olhamos pela primeira vez para esta imagem, pensamos: ``\'E um lim\~ao``. Mas, se \'e um lim\~ao, porque  n\~ao tem cheiro nem sabor? Porque n\~ao tem poros nem manchas? \'E claro que n\~ao pode ser um lim\~ao!
\singlespacing
Esta forma n\~ao \'e um lim\~ao, mas sim um modelo matem\'atico do lim\~ao. Ajuda-nos a obter uma melhor compreens\~ao das propriedades da forma do lim\~ao. Em geografia h\'a uma cita\c c\~ao atribu\'ida a {\it Alfred\ H.\ S.\ Korzybski}: ''O mapa n\~ao \'e o territ\'orio''. \\
\singlespacing

As equa\c c\~oes permitem-nos construir modelos matem\'aticos que nos ajudam a estudar melhor as formas das coisas. 
\singlespacing
E assim surge a beleza da matem\'atica: podemos gerar superf\'icies atraentes ou simular objectos reais a partir simplesmente de equa\c c\~oes alg\'ebricas, transportando os nossos pensamentos para zonas inesperadas da nossa mente.
\end{surferPage}
