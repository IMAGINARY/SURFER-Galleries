\begin{surferPage}{Balão}
A equa\c c\~ao, um nome sem ambiguidades \\
\smallskip
\[x^2 + y^2	= z^3	(1 - z) \]


\singlespacing
Todas as figuras nesta galeria t\^em nomes. Como chamaria a cada uma delas? Como outra pessoa as chamaria?\\
\vspace{0.3cm}
Ser\'a que podemos encontrar um modo de nomear formas que nunca leve \`a confus\~ao? 
A Matem\'atica encontrou uma solu\c c\~ao: nomeando-as pela sua equa\c c\~ao. A equa\c c\~ao determina todos os seus pontos, todas as curvas, buracos, rugas e picos. S\'o temos que saber como encontrar essas formas dentro da f\'ormula e como desenh\'a-las.\\
\vspace{0.3cm}
As equa\c c\~oes s\~ao escritas e interpretadas do mesmo modo em todo o mundo, porque a linguagem matem\'atica \'e universal, do mesmo modo que  as partituras musicais.
\end{surferPage}
