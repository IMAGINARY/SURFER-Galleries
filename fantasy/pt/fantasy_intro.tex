\begin{surferIntroPage}{Superfícies e Fantasia}{fantasy_kolibri}{Superfícies e Fantasia}
Todos n\'os j\'a ouvimos dizer que a Matem\'atica \'e complicada, mas a verdade \'e que ela nos ajuda a compreender a complexidade do nosso mundo.  Por exemplo, atrav\'es do
reconhecimento das estruturas fundamentais e das propriedades comuns  importantes de
objetos reais. A recolha numa \textit{classe} de todos os objetos com as mesmas propriedades importantes,
enquanto se ignoram as propriedades menos importantes, diz-se {\it  classifica\c c\~ao}. \'E uma das formas mais importantes para se obter uma vis\~ao geral da diversidade infinita de
objetos e de formas do nosso mundo. Para isso, a Matem\'atica \'e fundamental. Decidir
sobre o que \'e importante ou n\~ao vai depender do que queremos compreender. 
Pode ser, por exemplo, o tamanho ou a forma de um objecto.
\\

\vspace{0.4cm}

Descrever e classificar as formas \'e uma velha necessidade humana,  mas n\~ao \'e \'obvio como faz\^e-lo. Os antigos Gregos utilizaram principalmente a geometria e as propor\c c\~oes dos objetos geom\'etricos. Mais tarde, a \'algebra viria a ser desenvolvida essencialmente pelos \'Arabes (Al Khwarizmi, 900
A.C.). No s\'eculo XVIII, os matem\'aticos
Fermat e Descartes realizaram uma grande conquista ao introduzirem o sistema de coordenadas para descrever as rela\c c\~oes geom\'etricas. Esta conquista tornou poss\'ivel a utiliza\c c\~ao conjunta da \'algebra e da geometria.
\\
\vspace{0.4cm}
O programa SURFER \'e um excelente exemplo para esta rela\c c\~ao, uma vez que cria
a geometria (a imagem) a partir  da \'algebra (a f\'ormula).
Nesta galeria pode
experimentar a beleza da Matem\'atica e tornar-se criativo. Escolha uma
das superf\'icies do lado direito. A liga\c c\~ao matem\'atica entre f\'ormula
e forma \'e explicada de uma modo simples por meio de uma s\'erie de exemplos.\\
A imagina\c c\~ao e a intui\c c\~ao ficam do seu lado...
\end{surferIntroPage}
