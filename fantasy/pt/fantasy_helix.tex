\begin{surferPage}{Hélice}
Mais fina  que a superf\'icie de uma bolha de sab\~ao\\
  \smallskip
\[6x^2	= 2x^4	+ y^2	z^2\]

\singlespacing
As bolhas de sab\~ao s\~ao sens\'iveis; elas parecem estourar s\'o de olharmos para elas. As suas superf\'icies t\^em dois lados. Por fora \'e sab\~ao e por dentro \'e \'agua. Se a camada de sab\~ao ficar muito fina - o que ocorre quando a bolha fica maior - a \'agua faz a bolha estourar.\\
\vspace{0,3cm}
As superf\'icies alg\'ebricas s\~ao muito mais finas do que as superf\'icies das bolhas de sab\~ao, pois elas s\~ao apenas constitu\'idas por camadas de pontos. E desde que usemos a nossa imagina\c c\~ao para criar esses pontos, sem massa ou densidade, elas n\~ao estouram, mesmo se elas tiverem picos e rugas como a H\'elice.\\
\vspace{0,3cm}
No entanto, se  quisermos criar um modelo tridimensional da superf\'icie H\'elice, temos que construir uma escultura mais espessa do que a superf\'icie H\'elice real. Tal pode ser feito  refor\c cando a superf\'icie num dos seus lados.
\end{surferPage}
