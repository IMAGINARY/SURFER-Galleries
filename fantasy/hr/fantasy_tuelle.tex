\begin{surferPage}{Nozzle (Nos)}
%Infinitely many letters in a word
Bezbroj slova u jednoj rije\v{c}i\\
\smallskip
\[y z (x^2	+ y - z)	= 0\]

\vspace{0.3cm}
%Impressionists painted houses and meadows with thousands of coloured dots. Similarly, mathematical surfaces are formed by thousands of points, but points that have neither width nor mass but which solve the equation! 
Impresionisti su slikali ku\'{c}ice i proplanke tisu\'{c}ama \v{s}arenih to\v{c}kica. Sli\v{c}no tome, matemati\v{c}ke plohe se sastoje od tisu\'{c}a to\v{c}aka koje nemaju \v{s}irinu niti masu, ali koje rje\v{s}avaju jednad\v{z}bu!\\
\vspace{0.3cm}
%A way to imagine infinity is to start counting: $1, 2, 3,$ \dots\\
\v{Z}elimo li zamisliti beskona\v{c}nost, mo\v{z}emo po\v{c}eti brojati: $1, 2, 3,$ \dots\\
%There is always a larger number and we will never manage to count to the end.
Uvijek postoji ve\'{c}i broj od onih koje smo nabrojali. Nikad ne\'{c}emo zavr\v{s}iti brojanje.\\
\vspace{0.3cm}
%But not only the surface contains infinitely many points. Only between $0$ and $1$ there are infinitely many of them. This seems impossible? Just imagine that the points are infinitely small. They are drawn with a pencil of zero thickness. You have to draw many of them to fill the line between $0$ and $1$, namely infinitely many.
Ne samo da ploha sadr\v{z}i beskona\v{c}no mnogo to\v{c}aka: samo izme\dj{}u brojeva $0$ i $1$ postoji beskona\v{c}no mnogo brojeva. Izgleda li vam ovo nemogu\'{c}e? Zamislite da su to\v{c}ke beskona\v{c}no malene, nacrtane olovkom bez debljine. Da bismo ispunili interval izme\dj{}u $0$ i $1$ potrebno nam je beskona\v{c}no mnogo to\v{c}aka.
\end{surferPage}
