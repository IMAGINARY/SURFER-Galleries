\begin{surferPage}{Nozzle (Nos)}
Bezbroj slova u jednoj rije\v{c}i\\
\smallskip
\[y z (x^2	+ y - z)	= 0\]

\vspace{0.3cm}
Impresionisti su slikali ku\'{c}ice i proplanke tisu\'{c}ama \v{s}arenih to\v{c}kica. Sli\v{c}no tome, matemati\v{c}ke plohe se sastoje od tisu\'{c}a to\v{c}aka koje nemaju \v{s}irinu niti masu, ali koje rje\v{s}avaju jednad\v{z}bu!\\
\vspace{0.3cm}
\v{Z}elimo li zamisliti beskona\v{c}nost, mo\v{z}emo po\v{c}eti brojati: $1, 2, 3,$ \dots\\
Uvijek postoji ve\'{c}i broj od onih koje smo nabrojali. Nikad ne\'{c}emo zavr\v{s}iti brojanje.\\
\vspace{0.3cm}
Ne samo da ploha sadr\v{z}i beskona\v{c}no mnogo to\v{c}aka: samo izme\dj{}u brojeva $0$ i $1$ postoji beskona\v{c}no mnogo brojeva. Izgleda li vam ovo nemogu\'{c}e? Zamislite da su to\v{c}ke beskona\v{c}no malene, nacrtane olovkom bez debljine. Da bismo ispunili interval izme\dj{}u $0$ i $1$ potrebno nam je beskona\v{c}no mnogo to\v{c}aka.
\end{surferPage}
