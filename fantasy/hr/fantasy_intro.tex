\begin{surferIntroPage}{Fantasy Surfaces}{fantasy_kolibri}{The Fantasy Surfaces}
%We often hear about how complicated mathematics is, but it is a fact that it helps us to understand the complexity of our world. For example, through recognition of fundamental structures and important common properties of real objects. Gathering all objects with the same important properties in one \textit{class}, while ignoring the less important properties, is called classification. It is one of the most important means to get an overview of the sheer infinite diversity of the objects and forms of our world. For that, mathematics is fundamental. To decide about what is important or not depends on what you want to understand. This can be for example the size or form of an object.
\v{C}esto \v{c}ujemo kako je matematika te\v{s}ka, ali \v{c}injenica je da nam upravo matematika putem prepoznavanja osnovnih struktura i va\v{z}nih zajedni\v{c}kih svojstava odre\dj{}enih objekata poma\v{z}e da shvatimo svijet oko nas. Matematika je osnova svake klasifikacije - okupljanja svih objekata s nekim va\v{z}nim svojstvom, pri \v{c}emu se ostala manje va\v{z}na svojstva zanemaruju. Va\v{z}no svojstvo mo\v{z}e biti na primjer veli\v{c}ina ili oblik nekog predmeta. Klasifikacijom objekata dobivamo pregled razli\v{c}itih predmeta i oblika koji nas okru\v{z}uju.  \\

\vspace{0.4cm}

%To describe and to classify forms is an old human need, how to do it is not obvious. The ancient Greek mainly used geometry and proportions of geometrical objects. Later algebra was essentially developed by the Arabs (Al Khwarizmi, 900 B.C.). In the 18th century it was a big achievement by the mathematicians Fermat and Descartes to introduce the coordinate system to describe geometrical relations. This made it possible to use algebra and geometry together.
Nije uvijek o\v{c}ito na koji na\v{c}in treba opisati i klasificirati neke oblike. Stari Grci su prete\v{z}no koristili geometriju i proporcije geometrijskih objekata. Kasniju algebru su ve\'{c}im dijelom razvili Arapi (Al Khwarizmi, 900 p.n.e.) Veliko otkri\'{c}e matemati\v{c}ara Fermata i Descartesa u 18. stolje\'{c}u koje je povezalo geometriju i algebru je koordinatni sustav u kojem mo\v{z}emo opisivati geometrijske odnose.   \\
\vspace{0.4cm}
%The SURFER programme is a prime example for this relation since it creates geometry (the image) out of algebra (the formula).
Program SURFER tako\dj{}er povezuje geometriju i algebru: kreiramo geometrijsku sliku iz algebarske formule.
%In this gallery you can experience the beauty of mathematics and become creative yourself. Choose one of the surfaces on the right side. The mathematical connection between formula and form is explained through a series of examples in a simple way.
U galeriji mo\v{z}ete vidjeti ljepotu matematike i postati kreativni. Izaberite jednu od ploha na desnoj strani. Matemati\v{c}ka veza izme\dj{}u formule i oblika je na jednostavan na\v{c}in obja\v{s}njena u nizu primjera.
\\
%The imagination and intuition comes from you \dots
Iskoristite svoju ma\v{s}tu i intuiciju \dots
\end{surferIntroPage}
