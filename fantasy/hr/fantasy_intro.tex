\begin{surferIntroPage}{Fantasy Surfaces}{fantasy_kolibri}{The Fantasy Surfaces}
\v{C}esto \v{c}ujemo kako je matematika te\v{s}ka, ali \v{c}injenica je da nam upravo matematika putem prepoznavanja osnovnih struktura i va\v{z}nih zajedni\v{c}kih svojstava odre\dj{}enih objekata poma\v{z}e da shvatimo svijet oko nas. Matematika je osnova svake klasifikacije - okupljanja svih objekata s nekim va\v{z}nim svojstvom, pri \v{c}emu se ostala manje va\v{z}na svojstva zanemaruju. Va\v{z}no svojstvo mo\v{z}e biti na primjer veli\v{c}ina ili oblik nekog predmeta. Klasifikacijom objekata dobivamo pregled razli\v{c}itih predmeta i oblika koji nas okru\v{z}uju.  \\

\vspace{0.4cm}
Nije uvijek o\v{c}ito na koji na\v{c}in treba opisati i klasificirati neke oblike. Stari Grci su prete\v{z}no koristili geometriju i proporcije geometrijskih objekata. Kasniju algebru su ve\'{c}im dijelom razvili Arapi (Al Khwarizmi, 900 p.n.e.) Veliko otkri\'{c}e matemati\v{c}ara Fermata i Descartesa u 18. stolje\'{c}u koje je povezalo geometriju i algebru je koordinatni sustav u kojem mo\v{z}emo opisivati geometrijske odnose.   \\
\vspace{0.4cm}
Program SURFER tako\dj{}er povezuje geometriju i algebru: kreiramo geometrijsku sliku iz algebarske formule.
U galeriji mo\v{z}ete vidjeti ljepotu matematike i postati kreativni. Izaberite jednu od ploha na desnoj strani. Matemati\v{c}ka veza izme\dj{}u formule i oblika je na jednostavan na\v{c}in obja\v{s}njena u nizu primjera.
\\
Iskoristite svoju ma\v{s}tu i intuiciju \dots
\end{surferIntroPage}
