\begin{surferPage}{Vis à Vis (Licem u lice)}
Singularan ili regularan - prijatelj ili neprijatelj\\
\smallskip
\[x^2	- x^3+ y^2+ y^4+ z^3- z^4	=  0\]

\vspace{0.3cm}
Singularne to\v{c}ke ili singularitete vizulano identificiramo budu\'{c}i da ploha nije glatka ili mekana, kao na primjer \v{s}iljak ili pregib. \\
\vspace{0.3cm}
\v{S}iljak na lijevoj strani plohe Vis \`a Vis je singularitet; me\dj{}utim, glatko uzvi\v{s}enje na desnoj strani je regularna to\v{c}ka. Singulariteti su zanimljivi jer male promjene u jednad\v{z}bi mogu uzrokovati njihovo pojavljivanje na iznena\dj{}uju\'{c}i na\v{c}in.\\
\vspace{0.3cm}
Znate li da postoje ljudi koji se bave isklju\v{c}ivo prou\v{c}avanjem ovih to\v{c}aka? Crne rupe i Veliki prasak su singulariteti jednad\v{z}bi modela svemira. A sada pogledajte svoje prste. Singulariteti otisaka prstiju nas odre\dj{}uju!
\end{surferPage}
