\begin{surferPage}{Vis à Vis (Licem u lice)}
%Singular or regular - friend or enemy
Singularan ili regularan - prijatelj ili neprijatelj\\
\smallskip
\[x^2	- x^3+ y^2+ y^4+ z^3- z^4	=  0\]

\vspace{0.3cm}
%Singular points, or singularities, are identified visually because the surface is not smooth or soft, for instance, like a cusp or a fold.
Singularne to\v{c}ke ili singularitete vizulano identificiramo budu\'{c}i da ploha nije glatka ili mekana, kao na primjer \v{s}iljak ili pregib. \\
\vspace{0.3cm}
%The cusp on the left of the Vis \`a Vis surface is a singularity; however, the smooth hill on the right is a regular point. Singularities are interesting because small changes in the equation can change their appearance in a surprising way. \\
\v{S}iljak na lijevoj strani plohe Vis \`a Vis je singularitet; me\dj{}utim, glatko uzvi\v{s}enje na desnoj strani je regularna to\v{c}ka. Singulariteti su zanimljivi jer male promjene u jednad\v{z}bi mogu uzrokovati njihovo pojavljivanje na iznena\dj{}uju\'{c}i na\v{c}in.\\
\vspace{0.3cm}
%Do you know that there are people dedicated especially to studying these points? Black holes and the Big Bang constitute singularities of cosmological model equations. And now look at your finger tips, the singularities of our fingerprints identify us!
Znate li da postoje ljudi koji se bave isklju\v{c}ivo prou\v{c}avanjem ovih to\v{c}aka? Crne rupe i Veliki prasak su singulariteti jednad\v{z}bi modela svemira. A sada pogledajte svoje prste. Singulariteti otisaka prstiju nas odre\dj{}uju!
\end{surferPage}
