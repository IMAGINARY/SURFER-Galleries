\begin{surferPage}{Helix (Pu\v{z})}
%Thinner than a soap film
Tanji od sapunice na povr\v{s}ini vode \\
  \smallskip
\[6x^2	= 2x^4	+ y^2	z^2\]

\singlespacing
%Soap bubbles are sensitive; they appear to burst by just looking at them. Their surfaces have two sides. Outside is the soap and inside water. If the soap layer becomes too thin - this happens if the bubble gets bigger - the water makes the bubble burst.
Mjehuri\'{c}i sapunice su osjetljivi, \v{c}ini nam se da \'{c}e se raspasti od samog pogleda na njih. Njihova povr\v{s}ina ima dvije strane. Izvana je sapun, a iznutra je voda. Ako sloj sapuna postane pretanak - ovo se daga\dj{}a kada se mjehuri\'{c} pove\'{c}ava - mjehuri\'{c} se rapada.\\
\vspace{0,3cm}
%Algebraic surfaces are much thinner than soap films, they are only made out of point layers. And since we use our imagination to create these points, without mass or density, they do not burst, even if they have peaks and wrinkles as Helix.
Algebarske plohe su mnogo tanje od sloja sapunice, one se sastoje od sloja to\v{c}aka. Budu\'{c}i da su to\v{c}ke bez mase i gusto\'{c}e plod na\v{s}e ma\v{s}te, one se ne raspadaju \v{c}ak ni ako \v{c}ine nabore i vrhove kao na Helixu.\\
\vspace{0,3cm}
%But, if we want to create a three-dimensional model of the Helix surface, we have to build a sculpture thicker than the real Helix surface. This can be done by reinforcing the surface on one side.
No, ukoliko \v{z}elimo kreirati trodimenzionalni model plohe Helix, moramo izgraditi skulpturu tanju od prave povr\v{s}ine pu\v{z}a. To mo\v{z}emo napraviti poja\v{c}avanju\'{c}i jednu stranu plohe.
\end{surferPage}
