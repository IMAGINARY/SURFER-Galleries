\begin{surferPage}{Helix (Pu\v{z})}
Tanji od sapunice na povr\v{s}ini vode \\
  \smallskip
\[6x^2	= 2x^4	+ y^2	z^2\]

\singlespacing
Mjehuri\'{c}i sapunice su osjetljivi, \v{c}ini nam se da \'{c}e se raspasti od samog pogleda na njih. Njihova povr\v{s}ina ima dvije strane. Izvana je sapun, a iznutra je voda. Ako sloj sapuna postane pretanak - ovo se daga\dj{}a kada se mjehuri\'{c} pove\'{c}ava - mjehuri\'{c} se rapada.\\
\vspace{0,3cm}
Algebarske plohe su mnogo tanje od sloja sapunice, one se sastoje od sloja to\v{c}aka. Budu\'{c}i da su to\v{c}ke bez mase i gusto\'{c}e plod na\v{s}e ma\v{s}te, one se ne raspadaju \v{c}ak ni ako \v{c}ine nabore i vrhove kao na Helixu.\\
\vspace{0,3cm}
No, ukoliko \v{z}elimo kreirati trodimenzionalni model plohe Helix, moramo izgraditi skulpturu tanju od prave povr\v{s}ine pu\v{z}a. To mo\v{z}emo napraviti poja\v{c}avanju\'{c}i jednu stranu plohe.
\end{surferPage}
