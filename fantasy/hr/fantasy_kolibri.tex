\begin{surferPage}{Hummingbird (Kolibri\'{c})}
Jednad\v{z}ba odre\dj{}uje to\v{c}ke \\

  \smallskip

\[z^3+ y^2	z^2	= x^2\]

\singlespacing
U algebarskom smislu, kolibri\'{c} je skup svih to\v{c}aka $(x, y, z)$ koje zadovoljavaju jednad\v{z}bu

\smallskip

\[ x^2= y^2z^2+z^3.\]

\smallskip
Na primjer, $(0,0,0),$ $(1,0,1)$ i $(3,-2,-3)$ su to\v{c}ke kolibri\'{c}a, dok to\v{c}ka $(0,1,1)$ ne pripada 
kolibri\'{c}u.\\
 \singlespacing
Na\v{s} trodimenzionalni svijet je odre\dj{}en s tri smjera: ispred ili iza, lijevo ili desno, gore ili dolje. Ozna\v{c}imo ova tri smjera s $x$, $y$ i $z$. Svaku to\v{c}ku u prostoru mo\v{z}emo predstaviti s vrijednostima za svaki od njezina tri smjera. Te vrijednosti $(x,y,z)$ zovemo koordinatama dane to\v{c}ke.  \\
\singlespacing
Sada u jednad\v{z}bu uvrstimo koordinate svih to\v{c}aka u prostoru i obojimo samo one to\v{c}ke koje zadovoljavaju jednad\v{z}bu. Obojene to\v{c}ke formiraju sliku.
\end{surferPage}
