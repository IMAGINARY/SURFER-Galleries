\begin{surferPage}{Hummingbird (Kolibri\'{c})}
%The equation decides the points
Jednad\v{z}ba odre\dj{}uje to\v{c}ke \\
  
  \smallskip
	
\[z^3+ y^2	z^2	= x^2\]

\singlespacing
%In algebraic terms, Hummingbird is given by all points $(x, y, z)$ that hold the equation
U algebarskom smislu, kolibri\'{c} je skup svih to\v{c}aka $(x, y, z)$ koje zadovoljavaju jednad\v{z}bu

\smallskip

\[ x^2= y^2z^2+z^3.\]

\smallskip

%For example, $(0,0,0),$ $(1,0,1)$ and $(3,-2,-3)$ are points of Hummingbird, while $(0,1,1)$ is not part of it.
Na primjer, $(0,0,0),$ $(1,0,1)$ i $(3,-2,-3)$ su to\v{c}ke kolibri\'{c}a, dok to\v{c}ka $(0,1,1)$ ne pripada 
kolibri\'{c}u.\\
 \singlespacing
 %Our three-dimensional world is governed by three directions: ahead and back, left and right, up and down. These directions are identified with $x$, $y$ and $z$. Every point in space can be described by a value for each of its directions. These values are called the coordinates $(x,y,z)$ of this point.
Na\v{s} trodimenzionalni svijet je odre\dj{}en s tri smjera: ispred ili iza, lijevo ili desno, gore ili dolje. Ozna\v{c}imo ova tri smjera s $x$, $y$ i $z$. Svaku to\v{c}ku u prostoru mo\v{z}emo predstaviti s vrijednostima za svaki od njezina tri smjera. Te vrijednosti $(x,y,z)$ zovemo koordinatama dane to\v{c}ke.  \\
\singlespacing
%We now place all points in space with their values in the equation and colour only those where the equations is satisfied. All coloured points together then form the image.
Sada u jednad\v{z}bu uvrstimo koordinate svih to\v{c}aka u prostoru i obojimo samo one to\v{c}ke koje zadovoljavaju jednad\v{z}bu. Obojene to\v{c}ke formiraju sliku.
\end{surferPage}
