\begin{surferPage}{Quaste}
Abeceda jednad\v{z}bi
  \smallskip
\[8z^9-24x^2z^6-24y^2z^6+36z^8+24x^4z^3-168x^2y^2z^3\]
\[+24y^4z^3-72x^2z^5-72y^2z^5+54z^7-8x^6-24x^4y^2\]
\[-24x^2y^4-8y^6 + 36x^4z^2-252x^2y^2z^2+36y^4z^2\]
\[- 54x^2z^4-108y^2z^4 + 27z^6-108x^2y^2z + 54y^4z\]
\[-54y^2z^3 + 27y^4 = 0\]\\
\vspace{0.3cm}
Jeste li dobro promotrili jednad\v{z}bu Quaste? Izgleda vrlo komplicirano.
Figuru mo\v{z}emo jednostavno opisati: gornji rub ima oblik gr\v{c}kog slova $\alpha$, desni rub ima oblik krivulje sa \v{s}iljkom. Ovakav \v{s}iljak se zove {\it vrh}. Povu\v{c}ete li vrh du\v{z} $\alpha$ krivulje, dobit \'{c}ete Quaste. Plohu s tim svojstvom zovemo Kartezijev produkt u \v{c}ast francuskog matemati\v{c}ara Ren\'ea Descartesa.  \\
\vspace{0.3cm}
Monomi 1. stupnja su $x$, $y$, $z$, monomi 2. stupnja su $x^2, xy, y^2, xz, yz, z^2$ i tako dalje. \v{S}to je stupanj vi\v{s}i, imamo sve vi\v{s}e monoma te vi\v{s}e mogu\'{c}nosti za kreiranje slo\v{z}enijih oblika. To je poput abecede: \v{s}to vi\v{s}e slova imamo na raspolaganju, mo\v{z}emo napisati vi\v{s}e rije\v{c}i i re\v{c}enica.
\end{surferPage}
