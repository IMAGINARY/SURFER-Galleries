\begin{surferPage}{Zitrus (Limun)}
%This is not a lemon - the treachery of images
Ovo nije limun, ne dajte da vas slika zavara!\\
\smallskip
\[x^2 + z^2 = y^3 (1 - y)^3\] 


\singlespacing
%No doubt when we first set eyes on this image we all think: ``It's a lemon``. But, if it is a lemon, why does it neither have a scent nor a taste? Why doesn't it have pores or spots? Clearly it can't be a lemon! 
Pogledamo li ovu sliku, bez sumnje \'{c}emo re\'{c}i da vidimo limun. Ali, ako je to limun, za\v{s}to nema okus i miris? Za\v{s}to nema pore ili mrljice? O\v{c}ito to ne mo\v{z}e biti limun!
\singlespacing
%This shape is not a lemon, but a mathematical model of it. It helps us to get a better grasp of the properties of the lemon's shape. In geography there is a matching quote by $Alfred\ H.\ S.\ Korzybski$: ''The map is not the territory.'' 
Ova slika predstavlja matemati\v{c}ki model limuna. Poma\v{z}e nam u prou\v{c}avanju svojstava limunovog oblika. U geografiji postoji citat $Alfreda\ H.\ S.\ Korzybskija$: ''Karta nije teritorij.''\\
\singlespacing

%Equations allow us to build mathematical models that help us to study the shape of things better. 
Jednad\v{z}be nam omogu\'{c}uju gradnju matemati\v{c}kih modela koji nam poma\v{z}u u prou\v{c}avanju raznih oblika.
\singlespacing
%All this is part of the poetry of mathematics: we can generate beautiful surfaces by means of algebraic equations that transport our thoughts to unexpected corners of our mind.
Sve je to dio matemati\v{c}ke poezije: mo\v{z}emo generirati lijepe plohe putem algebarskih jednad\v{z}bi koje vode na\v{s}e misli do neslu\'{c}enih dijelova na\v{s}eg uma.
\end{surferPage}
