\begin{surferPage}{Nepali}
%Never ending world 
Beskrajan svijet\\

\smallskip
\[(x y - z^3 -1)^2= (1 - x^2	- y^2)^3\]

\singlespacing
%Maybe you find a surface simply beautiful and want to place it in a crystal ball with snow to play with it. But don't think that you can choose any surface to put it into your living room!
Mo\v{z}da ste na\v{s}li plohu tako lijepu da je \v{z}elite smjestiti u kristalnu kuglu sa snijegom i igrati se s njom! No, nemojte misliti da se bilo koja ploha mo\v{z}e smjestiti u va\v{s} dnevni boravak!
\\
\singlespacing
%There are surfaces that extend until infinity and, even if they are extremely pretty, you will never be able to place them into a crystal ball, no matter of its size. In this case we call the surface \textit{not constrained}. To paint such a surface we have to hide parts of it.
Postoje plohe koje se prote\v{z}u u beskona\v{c}nost i koje, iako su vrlo lijepe, nikada ne\'{c}ete mo\'{c}i smjestiti u kristalnu kuglu, koliko god ona velika bila. Za takvu plohu ka\v{z}emo da je neograni\v{c}ena. Da bismo obojili neograni\v{c}enu plohu, moramo sakriti neke njezine dijelove.
\singlespacing
%The property to be constrained can not be recognised easily, not even with the help of SURFER. It is as if we tired to find out if the universe is constrained: since we do not know its borders, it might have some or not.
Svojstvo ograni\v{c}enosti ne mo\v{z}emo lako prepoznati \v{c}ak i uz pomo\'{c} SURFER-a. To je kao da \v{z}elimo saznati je li svemir ograni\v{c}en: budu\'{c}i da ne znamo njegove granice, mo\v{z}da je ograni\v{c}en, a mo\v{z}da i nije.
\end{surferPage}
