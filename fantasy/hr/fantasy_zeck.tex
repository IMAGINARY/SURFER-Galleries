\begin{surferPage}{Tick (Krpelj)}
%The equation, an unambiguous name 
Jednad\v{z}ba, nedvosmisleno ime\\
\smallskip
\[x^2 + y^2	= z^3	(1 - z) \]


\singlespacing
%All figures in this gallery have names. How would you have called them? How would another person name them?
Sve figure u ovoj galeriji imaju imena. Kako biste ih vi nazvali?\\
\vspace{0.3cm}
%Can we find a way of naming shapes that never leads to confusion? Mathematics has found a solution: by naming them through their equation. It determines all its points, all curves, holes, wrinkles and peaks. You just have to know how to find these forms inside the equation and how to draw them.
Mo\v{z}emo li dati imena figurama tako da ne do\dj{}e do zabune? Matematika ima rje\v{s}enje: nazovimo ih njihovim jednad\v{z}bama. Jednad\v{z}ba odre\dj{}uje sve to\v{c}ke, krivulje, rupe, nabore i \v{s}iljke plohe. Moreate samo znati kako na\'{c}i te oblike u jednad\v{z}bi i kako ih nacrtati. \\
\vspace{0.3cm}
%Equations are written and interpreted all over the world in the same way, because the language of mathematics is universal, just like musical scores.
Jednad\v{z}be se pi\v{s}u i interpretiraju na isti na\v{c}in diljem svijeta jer je jezik matematike univerzalan.
\end{surferPage}
