\begin{surferPage}{Ding Dong}
Oblik se mijenja mijenjanjem jednad\v{z}be\\

\smallskip
\[x^2	+ y^2	+ z^3	= z^2\]

\singlespacing
Jedand\v{z}ba i oblik plohe Ding Dong su jednostavni. Oblik se dobiva okretanjem simbola $\alpha$ oko svoje osi.
Gledaju\'{c}i naopa\v{c}ke, Ding Dong se doima kao kapljica vode. Mo\v{z}emo promatrati padanje te kapljice.
\newline
Dodamo li mali parametar $a$ jednad\v{z}bi, neprestalno mijenjaju\'{c}i jednad\v{z}bu mo\v{z}emo kreirati niz slika koje prikazuju pojavljivanje kapi, njezino pribli\v{z}avanje krajnjoj poziciji te njezino nestajanje:

\[x^2	+ y^2	+ z^3	-z^2+0.1\cdot a=0.\]
U svakom trenutku kapljica je u stanju ravnote\v{z}e u kojem gravitacija kompenzira napetost povr\v{s}ine. Me\dj{}utim, ravnote\v{z}a kapi nije stabilna i nestaje prije pada. Teorija katastrofe matemati\v{c}ara Ren\'ea Thoma prou\v{c}ava kako male promjene u parametrima mogu uzrokovati promjene u ravnote\v{z}i.
\end{surferPage}
