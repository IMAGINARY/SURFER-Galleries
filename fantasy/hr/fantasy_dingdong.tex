\begin{surferPage}{Ding Dong}
%Change the figure by changing the equation
Oblik se mijenja mijenjanjem jednad\v{z}be\\

\smallskip
\[x^2	+ y^2	+ z^3	= z^2\]

\singlespacing
%The equation and the form of Ding Dong are simple. The figure is obtained by turning the Greek letter Alpha around its axis. If you look at it upside down, Ding Dong looks like a drop of water. We can watch the drop falling.
Jedand\v{z}ba i oblik plohe Ding Dong su jednostavni. Oblik se dobiva okretanjem simbola $\alpha$ oko svoje osi. 
Gledaju\'{c}i naopa\v{c}ke, Ding Dong se doima kao kapljica vode. Mo\v{z}emo promatrati padanje te kapljice.
\newline
%If you add a small parameter $a$ to the equation and change it continuously, we can create a series of images that show the emergence of the drop, how it approaches its end position and finally gets separated. It is like still images of a film:
Dodamo li mali parametar $a$ jednad\v{z}bi, neprestalno mijenjaju\'{c}i jednad\v{z}bu mo\v{z}emo kreirati niz slika koje prikazuju pojavljivanje kapi, njezino pribli\v{z}avanje krajnjoj poziciji te njezino nestajanje:

\[x^2	+ y^2	+ z^3	-z^2+0.1\cdot a=0.\]

%In every moment the drop is in a situation of balance where gravity compensates the surface tension. But the balance of the drop is not stable and it shivers before falling off. Catastrophe theory by the mathematician Ren\'e Thom studies how small changes in parameters can cause immediate changes in balance.
U svakom trenutku kapljica je u stanju ravnote\v{z}e u kojem gravitacija kompenzira napetost povr\v{s}ine. Me\dj{}utim, ravnote\v{z}a kapi nije stabilna i nestaje prije pada. Teorija katastrofe matemati\v{c}ara Ren\'ea Thoma prou\v{c}ava kako male promjene u parametrima mogu uzrokovati promjene u ravnote\v{z}i.
\end{surferPage}
