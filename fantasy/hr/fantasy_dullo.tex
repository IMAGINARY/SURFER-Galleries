\begin{surferPage}{Dullo}
%Unique phenomena in nature
Jedinstveni prirodni fenomen\\
\smallskip
\[(x^2+ y^2+ z^2)^2	= x^2+ y^2\]

\singlespacing
%Mathematics is closely linked to other sciences like physics, chemistry or technology and provides powerful tools to understand the world around us.
Matematika je usko vezana za ostale znanosti poput fizike, kemije i tehnologije. Ona je mo\'{c}an alat za razumijevanje svijeta oko nas. 
\singlespacing
%For instance, many phenomena we come across when studying nature give rise to models with singularities.
Na primjer, mnoge fenomene na koje nailazimo u prirodi mo\v{z}emo promatrati kao modele sa singularitetima.
\singlespacing
%This is the case of the propagation of sound waves produced by the public's enthusiastic applause in a football stadium. This phenomenon takes the form of the Dullo surface. It has a clear singularity in its centre and, therefore, the soccer referee avoids being on this spot on the pitch when a goal is being celebrated. The noise would harm his ears!
Jedan od tih fenomena je \v{s}irenje zvuka koji proizvodi publika na nogometnom stadionu. Ovaj fenomen mo\v{z}emo modelirati plohom Dullo. Ova ploha ima singularitet u svom sredi\v{s}tu pa stoga nogometni suci izbjegavaju sredi\v{s}te stadiona u trenutku postizanja gola. Buka bi mogla na\v{s}tetiti njihovim u\v{s}ima!

\end{surferPage}
