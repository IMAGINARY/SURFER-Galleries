\begin{surferPage}{Квачица}
Једначина, недвосмислено име \\
\smallskip
\[x^2 + y^2	= z^3	(1 - z) \]


\singlespacing
Све фигуре у овој галерији имају име. Како бисте их ви назвали? Како би их неко други назвао?\\
\vspace{0.3cm}
Можемо ли да нађемо начин да именујемо облике тако да никад не дође до конфузије? Математика је пронашла начин: облик добија име по својој једначини. Једначина одређује све тачке, све кривине, празнине, наборе и врхове. Само треба знати како да се све ово пронађе у формули и како да се то нацрта.\\
\vspace{0.3cm}
Једначине се пишу и тумаче свет на исти начин, јер је језик математике универзалан, баш као и нотни текст.
\end{surferPage}
