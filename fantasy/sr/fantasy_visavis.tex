\begin{surferPage}{Визави}
Сингуларан или регуларан – пријатељ или непријатељ\\
\smallskip
\[x^2	- x^3+ y^2+ y^4+ z^3- z^4	=  0\]

\vspace{0.3cm}
Сингуларне тачке или сингуларитети су видљиви јер површ није глатка или мекана, као на пример шиљак или превој.\\
\vspace{0.3cm}
Шиљак лево од визави површи представља сингуларитет; међутим, глатко испупчење с десне стране је регуларна тачка. Сингуларитети су занимљиви јер мале промене у једначини доводе до зачуђујућих промена у изгледу. \\

\vspace{0.3cm}
Да ли знате да су неки људи посвећени проучавању ових тачака? Црне рупе и Велики Прасак представљају сингуларитете једначина космичких модела. А сада погледајте врхове ваших прстију, сингуларитети наших врхова прстију нас одређују!
\end{surferPage}
