\documentclass[sr]{./../../common/SurferDesc}%%%%%%%%%%%%%%%%%%%%%%%%%%%%%%%%%%%%%%%%%%%%%%%%%%%%%%%%%%%%%%%%%%%%%%%
%
% The document starts here:
%
\begin{document}
\footnotesize
% Einfache Singularitäten 

\begin{surferPage}
  \begin{surferTitle}Динг Донг\end{surferTitle}  \\
Промените фигуру мењајући једначину\\

\smallskip
\[x^2	+ y^2	+ z^3	= z^2\]

\singlespacing
Једначина и облик Динг Донг-а су једноставни. Фигура се добија окретањем грчког слова $\alpha$ око његове осе. Ако је погледате одоздо, Динг Донг изгледа као кап воде. Можемо да посматрамо како кап пада.
\newline
Ако једначини додамо мали параметар $a$ и непрекидно га мењамо, можемо да направимо серију слика које приказују како се кап ствара, како се приближава својој крајњој позицији и како се на крају одваја. То је као успорени снимак филма:

\[x^2	+ y^2	+ z^3	-z^2+0.1\cdot a=0.\]

У сваком тренутку кап је у равнотежи, у којој гравитација потире површински напон. Али равнотежа капи није стабилна и она се откида пре пада. Теорија катастрофе математичара Ренеа Тома проучава како мале промене вредности параметра утичу на непосредне промене равнотеже. 



  \begin{surferText}
     \end{surferText}
\end{surferPage}
%%%%%%%%%%%%%%%%%%%%%%%%%%%%%


%%%%%%%%%%%%%%%%%%%%%%%%%%%%


\end{document}
%
% end of the document.
%
%%%%%%%%%%%%%%%%%%%%%%%%%%%%%%%%%%%%%%%%%%%%%%%%%%%%%%%%%%%%%%%%%%%%%%%
