\documentclass[sr]{./../../common/SurferDesc}%%%%%%%%%%%%%%%%%%%%%%%%%%%%%%%%%%%%%%%%%%%%%%%%%%%%%%%%%%%%%%%%%%%%%%%
%
% The document starts here:
%
\begin{document}
\footnotesize
% Einfache Singularitäten 

%%% 1.Tafel
%%%%%%%%%%%%%%%%%%%%%%%%%%%%%

\begin{surferPage}
  \begin{surferTitle}Бризгалица\end{surferTitle}   \\
Бесконачно много слова у речи\\
\smallskip
\[y z (x^2	+ y - z)	= 0\]

\vspace{0.3cm}
Импресионисти су сликали куће и ливаде помоћу хиљада шарених тачкица. Слично томе, математичке површи се састоје од хиљада тачака, али тачке немају ширину ни масу него су решења једначине! \\
\vspace{0.3cm}
Један од начина да се замисли бесконачност је да се броји: $1, 2, 3 $ \dots\\
Увек постоји већи број и никада нећемо моћи да избројимо до краја.\\
\vspace{0.3cm}
Али не садрже само површи бесконачан број тачака. Само између $0$ и $1$ их има бесконачно много. Изгледа вам немогуће? Само се сетите да су тачке бескрајно мале. Морали бисте да их нацртате много да бисте попунили део праве између $0$ и $1$, у ствари бесконачно много.




  \begin{surferText}
     \end{surferText}
\end{surferPage}


\end{document}
%
% end of the document.
%
%%%%%%%%%%%%%%%%%%%%%%%%%%%%%%%%%%%%%%%%%%%%%%%%%%%%%%%%%%%%%%%%%%%%%%%
