\begin{surferPage}{Хеликс}
Тање од сапунице\\
  \smallskip
\[6x^2	= 2x^4	+ y^2	z^2\]

\singlespacing
Мехури од сапунице су осетљиви; има се утисак да ће пући и када се само гледају. Њихове површи имају две стране. Споља је сапун а унутра вода. Ако слој сапуна постане сувише танак – ово се дешава када мехур постаје већи – вода проузрокује да мехур пукне.\\
\vspace{0,3cm}
Алгебарске површи су много тање од сапунице, састоје се само од слојева тачака. И пошто користимо машту да створимо ове тачке, без масе или густине, површи не пуцају, чак и када имају испупчења и наборе као хеликс.\\
\vspace{0,3cm}
Али, ако желимо да направимо тродимензиони модел површи хеликс, морамо да направимо скулптуру дебљу од праве хеликс површи. Ово се може постићи ојачавањем површи са једне стране.
\end{surferPage}
