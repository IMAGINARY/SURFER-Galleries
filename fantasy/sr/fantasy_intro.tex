\begin{surferIntroPage}{Површи маште}{fantasy_kolibri}{Површи маште}
Често чујемо како је математика компликована, али је чињеница да нам она помаже да схватимо сложеност света у коме живимо. На пример, кроз препознавање основних структура и важних заједничких особина објеката из стварности. Класификација је поступак сакупљања свих објеката истих важних особина у једну класу, у исто време занемарујући мање важне особине. Ово је један од најважнијих начина да се стекне општа слика о бескрајној разноликости објеката и облика нашег света. Због овога је математика фундаментална. Од онога што желимо да разумемо зависи шта је важно или не; на пример, величина или облик објекта.
\\

\vspace{0.4cm}

Описивање и класификација облика је стара људска потреба, али како је спровести, није тако очигледно. Стари Грци су превасходно користили геомeтрију и односе геометријских објеката. Касније, алгебру су суштински развили Арапи (Ал Хорезми, 900 година п.н.е.). У 18. веку Ферма и Декарт су постигли велику ствар увођењем координатног система у коме се описују геометријски односи. Ово је омогућило да се алгебра и геометрија користе заједно.
\\
\vspace{0.4cm}
Програм SURFER представља одличан пример за овакав однос, јер ствара геометрију (слику) из алгебре (формуле).
У овој галерији можете да искусите лепоту математике и да постанете креативни. Изаберите једну од површи са десне стране. Математичка веза формуле и облика је објашњена кроз низ примера на једноставан начин.\\
Машту и интуицију додајете Ви...
\end{surferIntroPage}
