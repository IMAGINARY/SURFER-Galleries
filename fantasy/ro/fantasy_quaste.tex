\begin{surferPage}{Ciucure}
ABC-ul ecua\c tiilor 
  \smallskip
\[8z^9-24x^2z^6-24y^2z^6+36z^8+24x^4z^3-168x^2y^2z^3\]
\[+24y^4z^3-72x^2z^5-72y^2z^5+54z^7-8x^6-24x^4y^2\]
\[-24x^2y^4-8y^6 + 36x^4z^2-252x^2y^2z^2+36y^4z^2\]
\[- 54x^2z^4-108y^2z^4 + 27z^6-108x^2y^2z + 54y^4z\]
\[-54y^2z^3 + 27y^4 = 0\]\\
\vspace{0.3cm}
V-a\c ti uitat mai \^indeaproape la ecua\c tia pentru Ciucure? Arat\u a foarte complicat.
Figura \^ins\u a\c si se poate descrie simplu: marginea superioar\u a are forma literei grece\c sti  $\alpha$, marginea din dreapta are forma 
unei curbe cu un v\^arf (sau col\c t). Un astfel de v\^arf se nume\c ste se nume\c ste {\it punct cuspidal}. Dac\u a tragem punctul cuspidal de-a lungul
curbei \^in form\u a de   $\alpha$, ob\c tinem Ciucurele.
Suprafe\c tele cu aceast\u a proprietate se numesc produse carteziene \^in onoarea matematicianului francez  Ren\'e Descartes.
\\
\vspace{0.3cm}
Monoamele de grad $1$ sunt $x$, $y$, $z$. Monoamele de grad $2$ sunt $x^2, xy, y^2, xz, yz, z^2$. \c Si a\c sa mai departe.
Cu c\^at este mai mare gradul, cu at\^at vom avea mai multe monoame si aceasta ne d\u a posibilitatea s\u a cre\u am forme mai complicate.
Este ca un alfabet: dac\u a avem mai multe litere la dispozi\c tie, putem s\u a scriem cuvinte si expresii mai complexe.

\end{surferPage}