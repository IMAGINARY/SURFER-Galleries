\begin{surferPage}{(Ne)-L\u am\^aia}
Aceasta nu este o l\u am\^aie - \^in\c sel\u atoria imaginilor\\
\smallskip
\[x^2 + z^2 = y^3 (1 - y)^3\]


\singlespacing
\^In mod sigur, atunci c\^and privim aceast\u a imagine pentru pentru prima dat\u a, cu to\c tii g\^andim:
''Uite o l\u am\^aie``. Dar dac\u a aceasta este o l\u am\^aie, de ce nu are gust \c si nici miros? De ce nu are
pori sau pete? Clar, nu poate fi o l\u am\^aie!
\singlespacing
Aceast\u a form\u a nu este o l\u am\^aie, doar un model matematic al ei. \c Si el
ne ajut\u a s\u a \^in\c telegem mai bine propriet\u a\c tile formei unei l\u am\^ai.
\^In acela\c si spirit, un citat din $Alfred\ H.\ S.\ Korzybski$: ``O hart\u a nu este teritoriul pe care \^il
reprezint\u a.''\\
\singlespacing


Ecua\c tiile ne permit s\u a construim modele matematice, modele care ne ajut\u a s\u a studiem mai bine forma lucrurilor.
\singlespacing
Toate acestea fac parte din poezia matematicii: putem genera suprafe\c te frumoase folosind ecua\c tii algebrice
\c si aceste forme ne conduc spre zone neb\u anuite ale intelectului nostru.
\end{surferPage}
