\begin{surferPage}{Pulverizator}
O infinitate de litere intr-un cuv\^ant\\
\smallskip
\[y z (x^2	+ y - z)	= 0\]

\vspace{0.3cm}
Impresioni\c stii au pictat paji\c sti \c si case folosind mii de puncte colorate. In mod analog suprafe\c tele matematice sunt formate din mii de puncte, dar
puncte care nu au nici lungime sau l\u a\c time si nici mas\u a dar rezolv\u a ecua\c tia! \\
\vspace{0.3cm}
Un mod de a ne imagina infinitul este s\u a \^incepem s\u a num\u ar\u am: $1, 2, 3,$ \dots\\
\^Intotdeauna exist\u a un numar mai mare \c si niciodat\u a nu vom termina de num\u arat.\\
\vspace{0.3cm}
Dar nu doar suprafe\c tele con\c tin o infinitate de puncte. Chiar \^intre $0$ \c si $1$ sunt o infinitate de puncte.
Pare imposibil? Imagina\c tiv\u a doar c\u a aceste puncte sunt infinit de mici, c\u a sunt desenate cu un creion cu grosime zero.
Trebuie s\u a desena\c ti multe puncte ca s\u a umple\c ti spa\c tiul dintre $0$ \c si $1$. Infinit de multe.
\end{surferPage}
