\begin{surferPage}{Insecta}
Ecua\c tia, un nume f\u ar\u a ambiguit\u a\c ti\\
\smallskip
\[x^2 + y^2	= z^3	(1 - z) \]


\singlespacing
Toate figurile din aceast\u a galerie au nume. Voi cum le-a\c ti fi numit? Dar altcineva, cum le-ar fi numit?\\
\vspace{0.3cm}
Putem g\u asi un mijloc de a da nume formelor care s\u a nu duc\u a la confuzii? Matematica a g\u asit
o solu\c tie: numele este chiar ecua\c tia lor. Ea determin\u a toate punctele formei, toate curbele, cutele 
\c si v\^arfurile sale. Trebuie doar s\u a \c stim cum s\u a g\u asim toate acestea \^in ecua\c tie \c si 
cum s\u a le desen\u am.\\
\vspace{0.3cm}
Ecua\c tiile sunt scrise \c si interpretate la fel \^in \^intreaga lume deoarece limbajul matematicii este
universal, la fel ca partiturile muzicale.
\end{surferPage}