\begin{surferPage}{Colibri}
Ecua\c tia determin\u a punctele\\
  
  \smallskip
\[z^3+ y^2	z^2	= x^2\]

\singlespacing
\^In termeni algebrici, Pas\u area Colibri este dat\u a de toate punctele $(x, y, z)$ care verific\u a ecua\c tia 
\smallskip

\[ x^2= y^2z^2+z^3.\]
\smallskip

De exemplu, $(0,0,0),$ $(1,0,1)$ \c si $(3,-2,-3)$ sunt puncte ale P\u as\u arii, pe c\^and $(0,1,1)$ nu este.\\
\singlespacing
Lumea noastr\u a trei-dimensional\u a este guvernat\u a de trei direc\c tii: \^in fa\c t\u a \c si \^in spate, 
st\^anga \c si dreapta, sus \c si jos. Aceste direc\c tii se identific\u a cu $x$, $y$ \c si $z$. 
Orice punct \^in spa\c tiu poate fi descris folosind c\^ate o valoare pentru fiecare dintre cele trei direc\c tii. 
Aceste valori se numesc coordonatele $(x,y,z)$ ale punctului.\\
\singlespacing
Acum, pentru toate punctele din spa\c tiu, \^inlocuim valorile coordonatelor lor \^in ecua\c tiu \c si 
le color\u am doar pe cele pentru care ecua\c tia este satisf\u acut\u a. Totalitatea punctelor colorate formeaz\u a 
imaginea.

\end{surferPage}

