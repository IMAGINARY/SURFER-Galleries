
\begin{surferIntroPage}{Fantasy Surfaces}{fantasy_kolibri}{Fantezia suprafe\c telor}
 
Auzim adesea despre c\^at de complicat\u a este matematica, dar adev\u arul este c\u a matematica ne ajut\u a
s\u a \^in\c telegem lumea \^in care tr\u aim. 
De exemplu, prin recunoa\c sterea structurilor fundamentale \c si a propriet\u a\c tilor comune importante ale
obiectelor reale.

Punerea laolalt\u a \^in aceea\c si \textit{clas\u a} a obiectelor care au acelea\c si propriet\u a\c ti importante
ignor\^and implicit propriet\u a\c tile mai pu\c tin importante, se nume\c ste clasificare.
Este una dintre cele mai importante metode prin care putem avea o privire globala aupra infinitei diversit\u a\c ti
a obiectelor si formelor din lumea \^inconjur\u atoare. Pentru aceasta, matematica este fundamental\u a.
A decide ce este important \c si ce nu, depinde de ce vrem s\u a \^in\c telegem. Aceasta poate fi, de exemplu, forma
sau m\u arimea unui obiect.
\\

\vspace{0.4cm}
Descrierea \c si clasificarea formelor este o veche nevoie uman\u a, cu toate acestea cum poate fi ea f\u acut\u a,
nu este deloc evident. Vechii greci foloseau cu prec\u adere geometria \c si propor\c tiile obiectelor geometrice.
Ceva mai t\^rziu algebra a fost dezvoltat\u a esen\c tialmente de arabi (Al Khwarizmi, 900 \^i. Hr.).
Introducerea de matematicienii Fermat \c si Descartes a sistemelor de coordante pentru a 
descrie rela\c tii geometrice a fost o mare realizare a secolului 18. Aceasta a f\u acut posibil\u a folosirea 
simultan\u a a algebrei \c si geometriei.
\\

\vspace{0.4cm}
Programul SURFER este un exemplu central pentru aceast\u a rela\c tie deoarece creeaz\u a geometrie (imaginea)
din algebr\u a (formula).
\^In aceast\u a galerie pute\c ti experimenta frumuse\c tea matematicii \c si deveni voi \^in\c siv\u a creativi.
Alege\c ti o suprafa\c t\u a din dreapta. Leg\u atura matematic\u a dintre formul\u a \c si form\u a este 
explicat\u a \^in mod simplu printr-o serie de exemple.
Imagina\c tia \c si intui\c tia vine de la voi  \dots

\end{surferIntroPage}