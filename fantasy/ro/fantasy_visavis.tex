\begin{surferPage}{Vis-\`a-Vis}
Singular sau regulat - prieten sau du\c sman\\
\smallskip
\[x^2	- x^3+ y^2+ y^4+ z^3- z^4	=  0\]

\vspace{0.3cm}
Punctele singulare sau singularit\u a\c tile, se pot identifica vizual pentru c\u a \^in aceste puncte suprafa\c ta nu este neted\u a. Poate avea, de exemplu, 
un v\^arf sau
o cut\u a.
\\
\vspace{0.3cm}
Punctul cuspidal din st\^anga suprafe\c tei Vis-\`a-Vis este o singularitate pe c\^and dealul neted din dreapta este un punct regulat. 
Singularit\u a\c tile sunt interesante pentru c\u a schimb\u ari mici ale ecua\c tiei pot schimba forma suprafe\c tei \^in mod surprinz\u ator.
\\

\vspace{0.3cm}
\c Stia\c ti c\u a sunt oameni care se dedic\u a \^in mod special studierii acestor puncte? G\u aurile negre \c si Big Bang-ul sunt singularit\u a\c ti 
ale ecua\c tiilor modelului cosmologic. \c Si acum uita\c ti-v\u a la degetele voastre, singularit\u a\c tile amprentelor noastre ne identific\u a!
\end{surferPage}

