\begin{surferPage}{Nepali}
O lume f\u ar\u a sf\^ar\c sit\\

\smallskip
\[(x y - z^3 -1)^2= (1 - x^2	- y^2)^3\]

\singlespacing
Poate g\u asi\c ti c\u a o suprafa\c t\u a este pur \c si simplu frumoas\u a \c si vre\c ti s\u a o pune\c ti \^intr-o sfer\u a de cristal cu z\u apad\u a
\c si s\u a v\u a juca\c ti cu ea. Dar s\u a nu crede\c ti c\u a pute\c ti alege orice suprafa\c t\u a ca s\u a o pune\c ti \^in sufragerie!
\\
\singlespacing
Sunt suprafe\c te care se extind la infinit \c si, chiar dac\u a sunt foarte frumoase, nu le vom putea niciodat\u a  pune \^intr-o sfer\u a de cristal,
indiferent de m\u arimea sa. \^In acest caz numim suprafa\c ta {\it nem\u arginit\u a}.
Pentru a desena o astfel de suprafa\c t\u a trebuie s\u a omitem p\u ar\c ti ale ei.
\\
\singlespacing
Proprietatea de a fi m\u arginit\u a nu poate recunoscut\u a u\c sor, nici chiar cu ajutorul lui SURFER. Este ca \c si cum am \^incerca s\u a afl\u am
dac\u a universul este m\u arginit: cum nu \^ii \c stim marginile, s-ar putea ca aceste margini s\u a existe sau nu.
\end{surferPage}

