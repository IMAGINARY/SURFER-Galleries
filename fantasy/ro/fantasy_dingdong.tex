\begin{surferPage}{Bing-Bang}
Schimb\^and ecua\c tia, schimb\u am figura \\

\smallskip
\[x^2	+ y^2	+ z^3	= z^2\]

\singlespacing
Ecua\c tia \c si forma Bing-Bang-ului sunt simple. Figura se ob\c tine rotind litera greceasc\ a Alfa \^in jurul axei sale. Dac\u a o privim
r\u asturnat\u a, figura arat\u a ca o pic\u atur\u a de ap\u a. Putem urm\u ari pic\u atura c\u az\^and.
\newline
Dac\u a ad\u aug\u am ecua\c tiei un parametru mic $a$ \c si apoi \^il schimb\u am  \^in mod continuu, putem crea o serie de imagini care arat\u a
cum se na\c ste pic\u atura, cum se apropie de pozi\c tia final\u a \c si apoi se separ\u a. Este ca un \c sir de imagini fixe dintr-un film: 

\[x^2	+ y^2	+ z^3	-z^2+0.1\cdot a=0.\]

\^In orice moment pic\u atura este \^intro pozi\c tie de echilibru \^in care gravitatea compenseaz\u a tensiunea superficial\u a. Dar echilibrul pic\u aturii
este instabil \c si pic\u atura \^incepe s\u a tremure \^inainte de a c\u adea. Teoria catastrofelor a matematicianului Ren\'e Thom studiaz\u a cum
mici schimb\u ari ale parametrilor pot cauza schimb\u ari imediate \^in echilibru.

\end{surferPage}
