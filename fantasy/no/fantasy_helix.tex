\begin{surferPage}{Heliks}
Tynnere enn en såpefilm\\
  \smallskip
\[6x^2	= 2x^4	+ y^2	z^2\]

\singlespacing
Såpebobler er skjøre, og de ser ut til å sprekke bare vi ser på dem. Overflaten deres har to sider. På utsiden er det såpe og på innsiden vann. Hvis boblen vokser seg større, blir såpelaget tynnere, og vannet får boblen til å sprekke. \\
\vspace{0,3cm}
Algebraiske flater er mye tynnere enn såpefilmer, de består bare av et lag av punkter. Og siden det er fantasien vår som skaper disse punktene, uten masse eller tykkelse, sprekker de ikke, selv om de har spisser og folder som heliksen.\\
\vspace{0,3cm}
Men, hvis vi vil lage en reell, tredimensjonal modell av heliksen, må vi bygge opp en skulptur som er tykkere enn den virkelige heliksflaten. Det kan vi få til ved å forsterke flaten på den ene siden.
\end{surferPage}
