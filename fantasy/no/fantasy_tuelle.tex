\begin{surferPage}{Tut}
Uendelig mange bokstaver i ett ord\\
\smallskip
\[y z (x^2	+ y - z)	= 0\]

\vspace{0.3cm}
Impresjonister malte hus og enger med tusenvis av små, fargede flekker. På samme måte er matematiske flater sammensatt av tusenvis av punkter. Faktisk av uendelig mange, nemlig av alle løsningene til ligningen! \\
\vspace{0.3cm}
Du kan forestille deg uendelighet ved å begynne å telle:  $1, 2, 3,\dotsc$\\
Det vil alltid finnes et større tall, og du vil aldri klare å telle til enden.\\
\vspace{0.3cm}
Det er ikke bare flater som består av uendelig mange punkter. Også mellom tallene $0$ og $1$ er det uendelig mange av dem. Er det virkelig mulig? Se for deg at punktene er uendelig små. De er malt med en pensel med tykkelse lik null. Du må male mange av dem for å fylle tallinjen mellom $0$ og $1$, nemlig uendelig mange.
\end{surferPage}
