\begin{surferIntroPage}{Fantasiflater}{fantasy_kolibri}{Fantasiflatene}
Vi hører ofte hvor komplisert matematikken er, men faktisk hjelper den oss til å forstå kompleksiteten av verden omkring oss. For eksempel gjennom gjenkjennelse av fundamentale strukturer og viktige felles egenskaper hos virkelige objekter. Om vi samler alle objekter med felles viktige egenskaper i en klasse, samtidig som vi ser bort fra mindre viktige egenskaper, har vi klassifisert objektene. Det er et av de aller viktigste virkemidlene vi har for å få en oversikt over de uendelig mange objektene og formene verden består av. Her er matematikken fundamental. Hva som er viktige egenskaper eller ikke, avhenger av hva du vil vite mer om.  Det kan for eksempel være størrelsen eller formen til et objekt.  
\\

\vspace{0.4cm}

Vi mennesker har lenge hatt behov for å beskrive og klassifisere former, og hvordan det gjøres er ikke opplagt. De gamle grekerne brukte mest geometri og proporsjoner til geometriske objekter. Senere utviklet araberne algebra (Al Khwarizmi, 900 B.C.). På 1800-tallet ivret matematikerne Fermat og Descartes for å introdusere koordinatsystemet for å beskrive geometriske forhold. Da ble det mulig å bruke algebra og geometri sammen.  
\\
\vspace{0.4cm}
Programmet SURFER er et godt eksempel på denne relasjonen, siden det skaper geometri (bildet) ut fra algebra (formelen). I dette galleriet kan du oppleve matematikkens skjønnhet og selv bli kreativ. Velg en av flatene på høyre side. Den matematiske sammenhengen mellom formel og form er forklart på en enkel måte gjennom mange eksempler. Fantasien og intuisjonen kommer fra deg.
\end{surferIntroPage}
