\begin{surferPage}{Søt}
\smallskip
\[(x^2+ 9/4y^2	+ z^2- 1)^3- x^2z^3	- 9/80y^2z^3	= 0\]

\singlespacing
Kjærlighetsbrev
\singlespacing
Slik kan vi ikke holde på!\\
Det som er, må forbli. \\
Når vi møtes igjen nå, \\
må noe nytt oppstå.\\
Sammen på toppen er vi.\\
Hva gjør vi da på topp?\\
Visst ikke sløseri.\\
For det som på topp gjør en stopp,\\
blir ikke så fort brukt opp.\\
På en benk i Grunewald\\
sitter i regnet to kropper.\\
Fatt mot, jente! Skriv til meg overalt!\\
Din Fritz! (Husk topper).\\
{\it Dikt av Joachim Ringelnatz}
\singlespacing 
Kjærlighetens lidenskap forbindes ofte med den følelsesmessige kraften i en ''singularitet''. Denne forbindelsen dukker opp i mange kunstarter.
\singlespacing 
Prøv å endre den siste eksponenten i ligningen fra 3 til 2 og se hva som skjer.
\end{surferPage}
