\begin{surferPage}{Quaste}
Ligningenes ABC
  \smallskip
\[8z^9-24x^2z^6-24y^2z^6+36z^8+24x^4z^3-168x^2y^2z^3\]
\[+24y^4z^3-72x^2z^5-72y^2z^5+54z^7-8x^6-24x^4y^2\]
\[-24x^2y^4-8y^6 + 36x^4z^2-252x^2y^2z^2+36y^4z^2\]
\[- 54x^2z^4-108y^2z^4 + 27z^6-108x^2y^2z + 54y^4z\]
\[-54y^2z^3 + 27y^4 = 0\]\\
\vspace{0.3cm}
Har du sett nærmere på ligningen Quaste? Den ser veldig komplisert ut. Figuren er enkel å beskrive: Den øvre kanten har form som den greske bokstaven $\alpha$, og den høyre kanten er kurveformet med en topp. Denne toppen kalles en {\it cusp}. Hvis vi drar en slik {\it cusp} langs alfakurven, får du Quaste. Flater med denne egenskapen kalles kartesianske produkter, til ære for den franske matematikeren Ren\'e Descartes.\\
\vspace{0.3cm}
Monomer (ledd i ligningen) av første grad er $x$, $y$, $z$. Monomer av andre grad er $x^2, xy, y^2, xz, yz, z^2$. Monomer av høyere grad gir oss flere muligheter til å skape mer kompliserte former. Det er som et alfabet: Jo flere bokstaver vi har, jo mer kompliserte ord og uttrykk kan vi skrive.
\end{surferPage}
