\documentclass[no]{./../../common/SurferDesc}%%%%%%%%%%%%%%%%%%%%%%%%%%%%%%%%%%%%%%%%%%%%%%%%%%%%%%%%%%%%%%%%%%%%%%%
%
% The document starts here:
%
\begin{document}
\footnotesize
% Einfache Singularitäten 

\begin{surferPage}
  \begin{surferTitle}Nepali\end{surferTitle}  \\
En uendelig verden \\

\smallskip
\[(x y - z^3 -1)^2= (1 - x^2	- y^2)^3\]

\singlespacing
Kanskje du finner en flate som er så vakker at du vil sette den inn i en krystallkule med snø for å riste på og leke med den. Men tro ikke at du kan velge en hvilken som helst flate og ha den i stua di!
\\
\singlespacing
Det finnes flater som strekker seg i det uendelige og som, selv om de er svært vakre, ikke vil få plass i en krystallkule, uansett hvor stor kula er. Vi sier at flaten er \textit{åpen}. For å tegne en slik flate, må vi skjule deler av den.
\\
\singlespacing
Å bestemme om en flate er lukket, er ikke lett, selv ikke ved hjelp av SURFER. Det er som om vi vil finne ut om universet er lukket: Siden vi ikke kjenner dets grenser, vet vi ikke om det har noen. 
  \begin{surferText}
     \end{surferText}
\end{surferPage}
%%%%%%%%%%%%%%%%%%%%%%%%%%%%%
%%%%%%%%%%%%%%%%%%%%%%%%%%%%%


\end{document}
%
% end of the document.
%
%%%%%%%%%%%%%%%%%%%%%%%%%%%%%%%%%%%%%%%%%%%%%%%%%%%%%%%%%%%%%%%%%%%%%%%
