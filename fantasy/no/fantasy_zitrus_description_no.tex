\documentclass[no]{./../../common/SurferDesc}%%%%%%%%%%%%%%%%%%%%%%%%%%%%%%%%%%%%%%%%%%%%%%%%%%%%%%%%%%%%%%%%%%%%%%%
%
% The document starts here:
%
\begin{document}
\footnotesize
% Einfache Singularitäten 

%%% 1.Tafel
\begin{surferPage}
  \begin{surferTitle}Sitrus\end{surferTitle}  \\ %%% Zitrus
Dette er ingen sitron – bildet bedrar\\
\smallskip
\[x^2 + z^2 = y^3 (1 - y)^3\] 


\singlespacing
Ved første øyekast tenker vi nok alle: ``Det er en sitron``. Men hvis det er en sitron, hvorfor har den verken lukt eller smak? Hvor er porene og flekkene? Så klart det ikke kan være en sitron! 
\singlespacing
Denne formen er ikke en sitron, men en matematisk modell av den. Den hjelper oss til å forstå mer av egenskapene til formen som sitronen har. I geografi finner vi et passende sitat av $Alfred\ H.\ S.\ Korzybski$: ''Kartet er ikke landet.'' \\
\singlespacing
Ligninger gjør det mulig å bygge matematiske modeller som hjelper oss til å lære mer om formen til tingene omkring oss. 
\singlespacing
Dette er noe av matematikkens poesi: Med algebraiske ligninger skaper vi vakre figurer som inspirerer oss til å tenke på nye måter.


  \begin{surferText}
     \end{surferText}
\end{surferPage}

\end{document}
%
% end of the document.
%
%%%%%%%%%%%%%%%%%%%%%%%%%%%%%%%%%%%%%%%%%%%%%%%%%%%%%%%%%%%%%%%%%%%%%%%
