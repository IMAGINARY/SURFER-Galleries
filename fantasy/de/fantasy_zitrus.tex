\begin{surferPage}{Zitrus}
Das ist keine Zitrone - Wie Bilder täuschen können\\
\smallskip
\[x^2 + z^2 = y^3 (1 - y)^3\] 


\singlespacing
Beim Betrachten dieses Bildes denken wir sicher alle: Das ist eine Zitrone. Aber wenn es eine Zitrone wäre, warum hat sie dann keinen Geruch und keinen Geschmack? Warum keine Poren oder Flecken? Es ist klar, dass es keine Zitrone sein kann!\\
\singlespacing
Diese Figur ist keine Zitrone, sondern ein mathematisches Modell einer Zitrone. Es hilft uns, die Eigenschaften der Form zu verstehen, die eine Zitrone ausmacht. In der Geographie gibt es hierzu ein passendes Zitat von \emph{Alfred\ H.\ S.\ Korzybski}: ''Die Landkarte ist nicht das Land.'' \\
\singlespacing
Gleichungen erlauben uns, mathematische Modelle zu konstruieren, die Objekten gleichen. Diese Modelle zu studieren hilft uns dann wiederum, die Form der Objekte zu begreifen.\\
\singlespacing
All das ist Teil der Poesie der Mathematik. Ausgehend von algebraischen Gleichungen können wir wunderschöne Figuren generieren, die unsere Gedanken in neue Gebiete unseres Geistes vordringen lassen.
\end{surferPage}
