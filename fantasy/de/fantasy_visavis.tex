\begin{surferPage}{Vis à Vis}
Spitz oder glatt - Freunde oder Feinde\\
\smallskip
\[x^2	- x^3+ y^2+ y^4+ z^3- z^4	=  0\]

\vspace{0.3cm}
Die spitzen Punkte, die so genannten {\it Singularitäten}, erkennt man häufig an ihrer Form. Es handelt sich um Punkte, an denen die Fläche nicht glatt und weich ist, sondern zum Beispiel eine Spitze oder eine Falte hat.\\
\vspace{0.3cm}
Die Fläche Vis à Vis zeigt sehr gut, was eine Singularität ist: die Spitze auf der linken Seite. Und sie zeigt auch, was sie nicht ist: der ebene Hügel auf der rechten Seite. Singularitäten sind unter anderem deshalb interessant, weil sie -
im Gegensatz zu stabilen glatten Punkten - ihr Aussehen schon bei kleinen Änderungen in der Gleichung überraschend ändern können.\\
\vspace{0.3cm}
Wissen Sie, dass es Menschen gibt, die sich speziell dem Studium dieser Punkte widmen? Die schwarzen Löcher und der Beginn des Universums, der Big Bang, sind Singularitäten in den Gleichungen der kosmologischen Modelle. Und betrachten Sie Ihre Fingerspitzen: die Singularitäten unserer Fingerabdrücke  identifizieren uns.
\end{surferPage}
