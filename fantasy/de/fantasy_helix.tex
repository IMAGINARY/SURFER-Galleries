\begin{surferPage}{Helix}
Dünner als eine Seifenblase\\
  \smallskip
\[6x^2	= 2x^4	+ y^2	z^2\]

\singlespacing
Seifenblasen sind so sensibel, dass sie durch bloßes hinsehen zerplatzen können. Ihre Oberflächen haben jedoch zwei Seiten. Außen ist Seife und innen ist Wasser. Wird die Seifenschicht zu dünn, wenn die Blase größer wird, dann zerreißt das Wasser die Seifenhaut.\\
\vspace{0,3cm}
Algebraische Flächen sind noch viel dünner als Seifenhäute, sie sind nur aus einer Punktschicht gemacht. Und da wir nur unsere Vorstellungskraft verwenden, um die Punkte zu erzeugen, ohne Masse oder Dicke, zerbrechen die Flächen nicht, auch wenn sie Spitzen und Falten haben, die so stark geformt sind wie bei Helix.\\
\vspace{0,3cm}
Aber, wenn wir ein reales, dreidimensionales Modell der Fläche von Helix erstellen wollen, müssen wir eine Skulptur erzeugen, die dicker ist als die echte Helix-Fläche, indem wir die Fläche auf einer Seite verstärken. 
\end{surferPage}
