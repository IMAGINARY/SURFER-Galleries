\begin{surferPage}{Kolibri}
Die Gleichung bestimmt die Punkte\\
  
  \smallskip
\[z^3+ y^2	z^2	= x^2\]

\singlespacing
Algebraisch gesehen ist der Kolibri gegeben durch alle Punkte  $(x, y, z)$, die die Gleichung
\smallskip
\[ x^2= y^2z^2+z^3\]
\smallskip
 erfüllen. Zum Beispiel, $(0,0,0),$ $(1,0,1)$ und $(3,-2,-3)$ sind Punkte des Kolibris, während $(0,1,1)$ nicht zum Kolibri gehört.\\
 \singlespacing
Unsere dreidimensionale Welt wird durch drei Richtungen bestimmt: vor und zurück, links und rechts und oben und unten. Diese Richtungen werden mit $x$, $y$ und $z$ bezeichnet. Jeder Punkt im Raum kann nun durch einen Wert für die jeweilige Richtung bestimmt werden. Das nennt man die Koordinaten $(x,y,z)$ dieses Punktes.\\
\singlespacing
Man setzt nun alle Punkte des Raumes in die Gleichung ein und färbt diejenigen, die die Gleichung erfüllen, ein. Alle eingefärbten Punkte zusammen ergeben dann das Bild.
\end{surferPage}