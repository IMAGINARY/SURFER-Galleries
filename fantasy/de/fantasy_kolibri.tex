\documentclass[de]{./../../common/SurferDesc}%%%%%%%%%%%%%%%%%%%%%%%%%%%%%%%%%%%%%%%%%%%%%%%%%%%%%%%%%%%%%%%%%%%%%%%
%
% The document starts here:
%
\begin{document}
\footnotesize
% Einfache Singularitäten 

%%%%%%%%%%%%%%%%%%%%%%%%%%%%%
\begin{surferPage}
  \begin{surferTitle}Kolibri\end{surferTitle}   \\
Die Gleichung bestimmt die Punkte\\
  
  \smallskip
\[z^3+ y^2	z^2	= x^2\]

\singlespacing
Algebraisch gesehen ist der Kolibri gegeben durch alle Punkte  $(x, y, z)$, die die Gleichung
\smallskip
\[ x^2= y^2z^2+z^3\]
\smallskip
 erf"ullen. Zum Beispiel, $(0,0,0),$ $(1,0,1)$ und $(3,-2,-3)$ sind Punkte des Kolibris, w"ahrend $(0,1,1)$ nicht zum Kolibri geh"ort.\\
 \singlespacing
Unsere dreidimensionale Welt wird durch drei Richtungen bestimmt: vor und zur�ck, links und rechts und oben und unten. Diese Richtungen werden mit $x$, $y$ und $z$ bezeichnet. Jeder Punkt im Raum kann nun durch einen Wert f�r die jeweilige Richtung bestimmt werden. Das nennt man die Koordinaten $(x,y,z)$ dieses Punktes.\\
\singlespacing
Man setzt nun alle Punkte des Raumes in die Gleichung ein und f�rbt diejenigen, die die Gleichung erf�llen, ein. Alle eingef�rbten Punkte zusammen ergeben dann das Bild. \\
\singlespacing


  \begin{surferText}
     \end{surferText}
\end{surferPage}


\end{document}
%
% end of the document.
%
%%%%%%%%%%%%%%%%%%%%%%%%%%%%%%%%%%%%%%%%%%%%%%%%%%%%%%%%%%%%%%%%%%%%%%%
