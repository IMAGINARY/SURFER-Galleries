\begin{surferPage}{Zeck}
Die Gleichung, ein unverwechselbarer Name \\
\smallskip
\[x^2 + y^2	= z^3	(1 - z) \]


\singlespacing
Alle Figuren in dieser Galerie haben Namen. Welche Namen hätten Sie gewählt? Welche Namen hätte sich wohl eine andere Person ausgesucht?\\
\vspace{0.3cm}
Kann man einen Weg finden, Formen so zu benennen, dass es zu keiner Verwechslung kommt? \\
In der Mathematik löst man dieses Problem mit dem Benennen durch eine Gleichung. Eine einzige Gleichung bestimmt die gesamte Figur: alle ihre Punkte, alle ihre Kurven, auch alle ihre Löcher, ihre Falten und Spitzen. Man muss nur wissen wie man diese Formen in der Gleichung findet und dann zeichnet.\\

\vspace{0.3cm}
Ausserdem schreibt und interpretiert man die Gleichungen überall auf dieselbe Weise, weil die Sprache der Mathematik universell ist, so wie die Noten in der Musik.
\end{surferPage}
