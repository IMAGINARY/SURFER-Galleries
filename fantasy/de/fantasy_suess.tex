\begin{surferPage}{Süss}
\[(x^2+ 9/4y^2	+ z^2- 1)^3- x^2z^3	- 9/80y^2z^3	= 0\]

\singlespacing
Liebesbrief
\singlespacing
So kann es nun nicht weitergehn! \\
Das, was besteht, muß bleiben. \\
Wenn wir uns wieder wiedersehn, \\
Muß irgendetwas geschehn. \\
Was wir dann auf die Spitze treiben.\\ 
Was - was auf einer Spitze tut? \\
Gewiß nicht Plattitüden. \\
Denn was auf einer Spitze ruht, \\
Wird nicht so leicht ermüden. \\
Auf einer Bank im Grunewald \\
Zu zweit im Regen sitzen, \\
Ist blöd. Mut, Mädchen! Schreibe bald! \\
Dein Fritz! (Remember Spitzen). \\
 {\it Gedicht von Joachim Ringelnatz}
\singlespacing 
Die Leidenschaft einer Liebe verbindet man gewöhnlich mit der emotionellen Kraft einer ''Singularität''. Diese Verbindung taucht auch immer wieder in allen Arten der Kunst auf.
\singlespacing 
Ändern Sie den letzten Exponenten der Gleichung von einer 3 zu einer 2 und beobachten Sie, was passiert.
\end{surferPage}
