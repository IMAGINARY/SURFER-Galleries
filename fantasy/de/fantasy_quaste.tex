\begin{surferPage}{Quaste}
Das ABC der Gleichungen

  \smallskip
\[8z^9-24x^2z^6-24y^2z^6+36z^8+24x^4z^3-168x^2y^2z^3\]
\[+24y^4z^3-72x^2z^5-72y^2z^5+54z^7-8x^6-24x^4y^2\]
\[-24x^2y^4-8y^6 + 36x^4z^2-252x^2y^2z^2+36y^4z^2\]
\[- 54x^2z^4-108y^2z^4 + 27z^6-108x^2y^2z + 54y^4z\]
\[-54y^2z^3 + 27y^4 = 0\]
\singlespacing
Haben Sie sich die Gleichung von {\it Quaste} angesehen? Sie sieht ziemlich kompliziert aus. Im Gegensatz dazu kann man die Figur aber einfach beschreiben: der obere Rand hat die Form des griechischen Buchstabens Alpha $\alpha$, der rechte Rand hat die Form einer Kurve mit einer Spitze. Eine solche Spitze nennt man {\it Kuspe}. Wenn man eine solche Kuspe nun entlang der Alpha-Kurve zieht, entsteht die Quaste. Die Flächen, die diese Eigenschaft haben, nennt man, zu Ehren des französischen Philosophen René Descartes, kartesische Produkte.

Wenn man $x,y, z$ kombiniert, kann man Monome vom Grad $1$ konstruieren, die einfach die Variablen $x,y, z $ sind; vom Grad 2,  $x^2, xy, y^2, xz, yz, z^2$; und so weiter. Je höher der Grad, umso mehr Monome haben wir und das gibt uns mehr Spielraum, um kompliziertere Formeln zu erzeugen. Es ist wie das ABC: je mehr Buchstaben zur Verfügung stehen, umso kompliziertere Wörter können wir schreiben.
\end{surferPage}
