\begin{surferIntroPage}{Phantasieflächen}{fantasy_kolibri}{Die phantasievollen Flächen}
Oft spricht man darüber, wie kompliziert Mathematik ist. Richtig ist aber auch, dass sie uns hilft die Komplexität unserer Welt zu verstehen. Zum Beispiel dadurch, dass mit Hilfe von Mathematik grundlegende Strukturen und wichtige gemeinsame Eigenschaften realer Objekte erkannt werden. Fasst man alle Objekte mit denselben wichtigen Eigenschaften in einer \textit{Klasse} zusammen, während man weniger wichtige Eigenschaften weglässt, so nennt man dies Klassifikation. Sie ist eines der wichtigsten Mittel, um eine \"Ubersicht über die schier unendliche Vielfalt der Objekte und Formen unserer Welt zu bekommen. Und dafür ist die Mathematik unerlässlich. Was wichtig oder unwichtig ist, hängt davon ab, was man verstehen m\"ochte. Das kann zum Beispiel die Gr\"o\ss e und Form eines Objekts sein.
\\
Formen zu beschreiben und zu klassifizieren ist ein altes menschliches Bed\"urfnis, das  \textit{wie} ist aber nicht offensichtlich. Die alten Griechen benutzten dazu Geometrie und die Verhältnisse geometrischer Gr\"o\ss en. Später wurde die Algebra im Wesentlichen von den Arabern (Al Khwarizmi 9. Jhd. n. Chr.) entwickelt. Es war dann im 18. Jahrhundert eine gro\ss e Leistung der Mathematiker Fermat und Descartes, das Koordinatensystem zur Beschreibung geometrischer Verhältnisse einzuführen. Erst dadurch wurde es m\"oglich Algebra und Geometrie gemeinsam zu verwenden. 
\\

Der SURFER ist dafür ein Paradebeispiel, indem er aus Algebra (die Formel) Geometrie (das Bild) erzeugt. 
In dieser Galerie können Sie nun die Schönheit der Mathematik selbst interaktiv erleben und gestalten. Tippen Sie auf eine der Fl\"achen rechts. Der mathematische Zusammenhang zwischen den Formen und den Formeln wird an Beispielen in einfachen Worten beschrieben. \\ Die Phantasie und die Intuition kommen von Ihnen \dots
\end{surferIntroPage}
