\begin{surferPage}{Tülle}
Unendlich viele Buchstaben in einem Wort\\
\smallskip
\[y z (x^2	+ y - z)	= 0\]


\singlespacing
Auf die gleiche Weise wie die Impressionisten Häuser und Wiesen mit tausenden Farbpunkten gemalt haben, sind die Flächen aus tausenden Punkten zusammengesetzt. Wenn man genau ist, aus unendlich vielen, nämlich aus allen Lösungen der Gleichung!\\
\singlespacing
Eine Art und Weise sich die Unendlichkeit vorzustellen, ist es einfach zu zählen: $1, 2, 3,$ \dots\\
Es gibt immer eine größere Zahl und wir werden es nie schaffen, alle natürlichen Zahlen aufzuzählen.\\
\singlespacing
Doch nicht nur die Fläche besteht aus unendlich vielen Punkten. Auf dem Zahlenstrahl liegen zwischen den Zahlen $0$ und $1$ allein unendlich viele Punkte. Das scheint nicht möglich? Bedenken Sie dass die Punkte unendlich klein sind. Man zeichnet sie sozusagen mit einem Stift der Dicke Null. Man muss ganz schön viele Punkte zeichnen, um die Linie zwischen $0$ und $1$ auszufüllen, nämlich unendlich viele. 
\end{surferPage}
