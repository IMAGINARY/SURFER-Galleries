\begin{surferPage}{Непальская шляпа}
Мир без конца и края \\

\smallskip
\[(x y - z^3 -1)^2= (1 - x^2	- y^2)^3.\]

\singlespacing
Возможно, поверхность покажется Вам просто красивой, и Вам захочется поместить её в стеклянный шар со снежными хлопьями для того, чтобы её можно было встряхнуть и играть с ней. Но, пожалуйста, не подумайте, что Вы можете выбрать любую поверхность, для того чтобы разместить её у себя в комнате!
\\
\singlespacing
Существуют поверхности, распространяющиеся в бесконечность. Их, какими бы прекрасными они ни были, Вы никогда не сможете поместить в стеклянный шар, каким бы большим он не был. В этом случае говорят, что поверхность неограничена, а для того, чтобы её изобразить необходимо отбросить часть поверхности.
\\
\singlespacing
Свойство поверхности быть ограниченной нельзя распознать просто эмпирически, здесь не поможет даже SURFER. Дело здесь обстоит таким образом, как если бы нам нужно было определить, бесконечна ли Вселенная или нет? От того, что мы не знаем границ и краев, мы не можем сделать вывод об их наличии или отсутствии.
\end{surferPage}
