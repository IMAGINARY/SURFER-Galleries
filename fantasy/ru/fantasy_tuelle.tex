\documentclass[ru]{./../../common/SurferDesc}%%%%%%%%%%%%%%%%%%%%%%%%%%%%%%%%%%%%%%%%%%%%%%%%%%%%%%%%%%%%%%%%%%%%%%%
%
% The document starts here:
%
\begin{document}
\footnotesize
% Einfache Singularitäten 

%%% 1.Tafel
%%%%%%%%%%%%%%%%%%%%%%%%%%%%%

\begin{surferPage}
  \begin{surferTitle}Носик\end{surferTitle}   \\
Бесконечно много букв в одном слове\\
\smallskip
\[y z (x^2	+ y - z)	= 0\]

\vspace{0.3cm}
Почти также как импрессионисты изображали дома и луга при помощи тысяч цветовых пятен, и поверхности состоят из тысяч точек. А точнее говоря, из бесчисленного множества точек – всех решений данного уравнения. \\
\vspace{0.3cm}
Чтобы представить себе бесконечность, просто продолжайте считать: $1, 2, 3,$ \dots\\
Всегда имеется большее число и нам не удастся перечислить все натуральные числа.\\
\vspace{0.3cm}
Но всё же не только поверхности состоят из бесчисленного множества точек. На числовом луче на одном лишь промежутке между $0$ и $1$ имеется бесконечное количество точек. Кажется, что это невозможно? Примите во внимание то, что точки бесконечно малы. Т.е. их рисуют карандашом, диаметр которого равен $0$. Нужно изобразить очень много точек (а точнее говоря – бесконечно много), чтобы заполнить линию от $0$ до $1$.

  \begin{surferText}
     \end{surferText}
\end{surferPage}


\end{document}
%
% end of the document.
%
%%%%%%%%%%%%%%%%%%%%%%%%%%%%%%%%%%%%%%%%%%%%%%%%%%%%%%%%%%%%%%%%%%%%%%%
