\begin{surferPage}{Колибри}
Уравнение определяет точки поверхности:\\
  
  \smallskip
\[z^3+ y^2	z^2	= x^2.\]

\singlespacing
С алгебраической точки зрения, поверхность «Колибри» задана всеми точками $(x, y, z)$, удовлетворяющими уравнению
\smallskip
\[ x^2= y^2z^2+z^3.\]
\smallskip
Например, $(0,0,0),$ $(1,0,1)$ и $(3,-2,-3)$  - точки поверхности «Колибри», в то время как точка $(0,1,1)$ не принадлежит ей.\\
 \singlespacing
Наш трёхмерный мир определяется тремя направлениями: вперёд – назад, влево – вправо и вверх – вниз. Эти направления обозначают при помощи $x$, $y$ и $z$. Всякая точка в пространстве может быть задана определенным значением соответствующего направления. Это называют координатами $(x,y,z)$ данной точки. \\
\singlespacing
Подставим в уравнение все точки нашего пространства и окрасим те из них, которые удовлетворяют уравнению. Все окрашенные таким образом точки и дают нам изображение поверхности «Колибри».
\end{surferPage}
