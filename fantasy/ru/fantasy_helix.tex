\begin{surferPage}{Геликс}
Тоньше, чем мыльная плёнка\\
  \smallskip
\[6x^2	= 2x^4	+ y^2	z^2.\]

\singlespacing
Мыльные пузыри так нежны, что могут лопнуть лишь только потому, что на них взглянули. Но поверхность мыльного пузыря всё же имеет $2$ стороны. Снаружи расположено мыло, а внутри – вода. Если мыльный слой становится слишком тонким при увеличении пузыря, то вода разрывает «мыльную кожу» пузыря. \\
\vspace{0,3cm}
Алгебраические поверхности намного тоньше, чем мыльный слой, ведь они состоят лишь из слоя точек. Ну, а если уж мы используем силу нашей фантазии, чтобы представить точки без массы и величины, то и поверхности не лопаются, даже если у них есть острые вершины и складки, например геликс.\\
\vspace{0,3cm}
Но, если мы захотим создать реальную трёхмерную модель поверхности геликса, то нам нужно будет создать скульптуру, более «толстую», чем поверхность самого геликса, укрепляя поверхность с одной стороны.
\end{surferPage}
