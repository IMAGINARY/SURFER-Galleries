\begin{surferPage}{Sweet}
\smallskip
\[(x^2+ 9/4y^2	+ z^2- 1)^3- x^2z^3	- 9/80y^2z^3	= 0\]

\singlespacing
Liebesbrief
\singlespacing
So kann es nun nicht weitergehn! \\
Das, was besteht, mu"s bleiben. \\
Wenn wir uns wieder wiedersehn, \\
Mu"s irgendetwas geschehn. \\
Was wir dann auf die Spitze treiben.\\ 
Was - was auf einer Spitze tut? \\
Gewi"s nicht Plattit"uden. \\
Denn was auf einer Spitze ruht, \\
Wird nicht so leicht erm"uden. \\
Auf einer Bank im Grunewald \\
Zu zweit im Regen sitzen, \\
Ist bl"od. Mut, M"adchen! Schreibe bald! \\
Dein Fritz! (Remember Spitzen). \\
 {\it Gedicht von Joachim Ringelnatz}\\
 {\it Стихотворение Йоахима Рингельнатца}\\
\singlespacing 
Страсть, присущую любви, обычно связывают с эмоциональной силой «сингулярности». Эта связь проявляется снова и снова в различных видах искусства.
\singlespacing 
Измените последние показатели степени в уравнении с $3$ на $2$ и понаблюдайте, что произойдет...
\end{surferPage}
