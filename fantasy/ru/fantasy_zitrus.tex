\begin{surferPage}{Цитрус}
Это не лимон – изображения могут быть обманчивы\\
\smallskip
\[x^2 + z^2 = y^3 (1 - y)^3\] 


\singlespacing
Когда мы смотрим на это изображение, то, наверное, все думают: это же лимон. Но если бы это был лимон, то почему же у него нет запаха и вкуса? Почему нет пор и пятен? Понятно, что это не может быть лимоном!
\singlespacing
Эта фигура – не лимон, а математическая модель лимона. Она помогает нам понять свойства формы, которую имеет лимон. В географии для этого есть подходящая цитата Альфреда Коржубского: «Карта – это не страна».\\
\singlespacing

Уравнения позволяют нам конструировать математические модели, похожие на объекты. Изучение этих моделей позволяет нам, в свою очередь, понимать форму объекта.
\singlespacing
Всё это – часть поэзии математики. Исходя из алгебраических уравнений мы можем создавать прекрасные фигуры, которые позволяют нашим мыслям проникнуть в новые сферы нашего сознания.
\end{surferPage}
