\begin{surferIntroPage}{Фантастич. пов-сти}{fantasy_kolibri}{Фантастические поверхности}
Часто говорится о том, как сложна математика. Но справедливо и то, что она помогает понять наш мир в его многообразии. Например, благодаря математике можно выделить основополагающие структуры и важнейшие общие свойства реальных объектов. Если объединить все объекты с одними и теми же важнейшими свойствами в один \emph{класс}, отбросив при этом менее важные качества, то получим классификацию объектов. Она является одним из основных средств для того, чтобы увидеть бесконечное многообразие объектов и форм нашего мира. Для этого использование математики неизбежно. То, что важно, а что нет, зависит от того, что мы хотим понять. Например, это могут быть величина и форма объекта.

Описание и классификация форм — древняя человеческая потребность, но как это сделать — не очевидно. Древние греки использовали для этого геометрию и соотношение геометрических величин. Позднее, преимущественно арабами, активно развивалась алгебра (Мухаммед ибн Муса Хорезми, 9 в. н.э.). Большим достижением математиков Ферма и Декарта в 17 веке стало введение координатной системы для описания геометрических отношений. Лишь благодаря этому стало возможным совместное использование алгебры и геометрии. 

Программа SURFER прекрасный тому пример, в ней из алгебры (формулы) создаётся геометрия (образы). В этой интерактивной галерее Вы можете увидеть красоту математики и создавать её самостоятельно. Кликните на одну из поверхностей с правой стороны. Математические взаимосвязи форм и формул описываются на примерах простым языком. А от Вас требуются лишь фантазия и вдохновение ...
\end{surferIntroPage}
