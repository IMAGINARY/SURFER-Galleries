\begin{surferPage}{«Калейдоскоп»}
Мы создаем новые формы \\
\smallskip
\[x^2	- y^2z^2	= 0.\]

\singlespacing
Для того, чтобы создавать новые формы, следует понимать уравнения. Самые простые  части – это так называемые мономы (одночлены), т.е. алгебраические выражения из букв и цифр.
\singlespacing
Одночлен содержит следующие элементы: знак, коэффициент, переменные и степень.\\
\singlespacing
Например:
\smallskip
\[2xy^2z = +2x^1y^2z^1.\]
\\
\smallskip
Степень одночлена – это сумма показателей степеней его переменных: $степень = 1+2+1=4$.  \\
\singlespacing
Для того, чтобы составлять уравнения используют математические операции: сложение, вычитание и умножение. Это те операции, с которыми мы знакомимся в начальной школе. С их помощью можно изобразить любые алгебраические поверхности.\singlespacing
А Вы можете создавать формы с острыми вершинами и отверстиями лишь путем сложения и умножения?
\end{surferPage}
