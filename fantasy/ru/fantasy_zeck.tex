\begin{surferPage}{Клещ}
Уравнение – имя, которое нельзя перепутать\\
\smallskip
\[x^2 + y^2	= z^3	(1 - z) \]


\singlespacing
У всех фигур в этой галерее есть свои имена. А какие названия подобрали бы Вы? Какие бы выбрали Ваши друзья? Можно ли найти подход, чтобы назвать формы, избегая путаницы?\\
\vspace{0.3cm}
В математике эту проблему решают посредством использования уравнений в качестве обозначений. Одно единственное уравнение определяет всю фигуру: все ее точки, все ее кривые, а также все ее отверстия, складки и вершины. Нужно лишь знать, как найти эти формы в уравнении и как их потом изобразить.\\
\vspace{0.3cm}
Кроме того, уравнения записывают и интерпретируют повсюду одинаково, т.к. язык математики универсален, также, как ноты в музыке.
\end{surferPage}
