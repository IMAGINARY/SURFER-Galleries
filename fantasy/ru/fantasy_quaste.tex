\begin{surferPage}{Кисть}%
\singlespacing
Азбука уравнений

{\scriptsize\begin{gather*}
8z^9-24x^2z^6-24y^2z^6+36z^8+24x^4z^3-168x^2y^2z^3+24y^4z^3\\
-72x^2z^5-72y^2z^5+54z^7-8x^6-24x^4y^2-24x^2y^4-8y^6\\
+ 36x^4z^2-252x^2y^2z^2+36y^4z^2- 54x^2z^4-108y^2z^4\\
 + 27z^6-108x^2y^2z + 54y^4z-54y^2z^3 + 27y^4 = 0
\end{gather*}}

Посмотрели на уравнение «Кисти»? Оно выглядит довольно сложным. Но поверхность всё же можно легко описать: Верхний край этой поверхности – петля в форме греческой буквы $\alpha$ («альфа»). Край с правой стороны, напротив, состоит из двух параллельных, т.н. куспидальных (остроконечных) кривых с одной вершиной на каждой из них. Если двигать одну из таких заострённых кривых вдоль $\alpha$ -петли, то получим поверхность «Кисти». Поверхности с таким свойством называют прямым или декартовым произведением двух множеств (по имени французского математика Рене Декарта).
%
Если будем комбинировать  $x$, $y$, $z$, то можно конструировать одночлены первой степени, которые и являются просто переменными x, y, z; второй степени - $x^2, xy, y^2, xz, yz, z^2$ и т.д. Чем выше степень, тем больше одночленов, а это даёт нам больше возможностей для того, чтобы создавать сложные формулы. Это своеобразная азбука, алфавит: чем больше букв у нас в распоряжении, тем более сложные слова мы можем создавать.
\end{surferPage}
