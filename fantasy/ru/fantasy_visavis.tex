\begin{surferPage}{Визави}
Заостренный или гладкий – друг или враг\\
\smallskip
\[x^2	- x^3+ y^2+ y^4+ z^3- z^4	=  0\]

\vspace{0.3cm}
Особые точки, т.н. сингулярности, часто распознают по их форме. Здесь речь идёт о точках, в которых поверхность не является гладкой и гибкой, а, например, имеет остриё или складку.\\
\vspace{0.3cm}
Поверхность «Визави» прекрасно демонстрирует, что же такое сингулярность: острие на левой стороне. Кроме того, она показывает, чем сингулярность не является: пологий холмик справа. Сингулярности интересны ещё и потому, что они, в отличие от обычных гладких точек, при малейших изменениях в уравнении удивительным образом изменяют свой «внешний вид». \\

\vspace{0.3cm}
Вы знаете, что есть люди, посвятившие свою жизнь изучению таких точек? Черные дыры и центр Вселенной, «Большой взрыв» - это тоже сингулярности в уравнениях космологических моделей. А посмотрите на кончики своих пальцев: сингулярности наших отпечатков пальцев позволяют нас идентифицировать.
\end{surferPage}
