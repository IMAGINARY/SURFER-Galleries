\begin{surferPage}{Динь-дон}
Изменение фигуры через изменение уравнения:\\

\smallskip
\[x^2	+ y^2	+ z^3	= z^2\]

\singlespacing
Уравнение и форма поверхности «Динь-дон» очень просты. Фигуру можно получить, если вращать греческую букву $\alpha$ («альфа») вокруг своей оси. Если посмотреть с другой стороны, то поверхность «Динь-дон» похожа на каплю воды. Можно даже сказать, что мы наблюдаем как падает капля.
\newline
Если ввести в уравнение параметр $a$, который непрерывно изменять, то можем получить серию изображений, которая показывает образование капли, как она расширяется у нижнего основания и в конце концов отрывается и падает вниз. 

Это своеобразная серия кадров из фильма:
\smallskip\smallskip\smallskip\smallskip
\[x^2	+ y^2	+ z^3	-z^2+0.1\cdot a=0.\]

В каждый момент времени капля находится в состоянии равновесия, при которой сила тяжести уравновешивает поверхностное натяжение. Но очевидно равновесное состояние капли нестабильно, и капля «дрожит» вплоть до своего падения. Теория катастроф математика Рене Тома занимается тем, как малые изменения параметров могут вызывать мгновенные изменения равновесных состояний.
\end{surferPage}
