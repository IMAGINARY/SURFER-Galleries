\begin{surferPage}{דינג-דונג}
שנו את הצורה על-ידי שינוי המשוואה\\

\smallskip
\[x^2	+ y^2	+ z^3	= z^2\]

\singlespacing
זוהי משוואה פשוטה וכך גם צורת הדינג-דונג. הצורה מתקבלת מסיבוב האות היוונית אלפא סביב צירה. אם תביטו בה במהופך, תבחינו בכך שדינג-דונג נראית כמו טיפת מים. תוכלו לצפות בטיפה הנופלת.
\newline
אם תוסיפו פרמטר קטן $a$ למשוואה ותשנו אותו באופן רציף, תוכלו ליצור סדרת תמונות הממחישות את ההיווצרות ההדרגתית של הטיפה, את התקרבותה למצב הסופי ולבסוף, את היפרדות הטיפה הבודדת. סדרת התמונות דומה למסגרות המופיעות בסליל של סרט קולנוע:

\[x^2	+ y^2	+ z^3	-z^2+0.1\cdot a=0.\]

בכל רגע הטיפה נמצאת בשיווי משקל כך שכוח המשיכה מפצה על מתח הפנים שלה. אך שיווי המשקל של הטיפה אינו יציב, ולכן הטיפה רוטטת קלות לפני שהיא נופלת. תורת הקטסטרופות שהגה המתמטיקאי רנה תום עוסקת בדרכים שבהן שינויים קטנים בפרמטרים עשויים לגרום לשינויים מידיים בשיווי המשקל.
\end{surferPage}
