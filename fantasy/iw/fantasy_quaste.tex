\begin{surferPage}[גדיל]{גדיל}
האלף-בית של המשוואות
  \smallskip
\[8z^9-24x^2z^6-24y^2z^6+36z^8+24x^4z^3-168x^2y^2z^3\]
\[+24y^4z^3-72x^2z^5-72y^2z^5+54z^7-8x^6-24x^4y^2\]
\[-24x^2y^4-8y^6 + 36x^4z^2-252x^2y^2z^2+36y^4z^2\]
\[- 54x^2z^4-108y^2z^4 + 27z^6-108x^2y^2z + 54y^4z\]
\[-54y^2z^3 + 27y^4 = 0\]\\
\vspace{0,3cm}
הביטו מקרוב על במשוואה של הצורה הקרויה גדיל. היא נראית מורכבת מאד.
את הצורה עצמה ניתן לתאר במילים פשוטות: השפה העליונה היא בעלת צורת האות היוונית $\alpha$, ואילו לשפה הימנית יש צורת עקומה עם פסגה. פסגה מאין זו נקראת {\it חוד}. אם גוררים חוד מעין זה לאורך עקומה בצורת האות אלפא מקבלים את הצורה המכונה גדיל. משטחים בעלי מאפיין מעין זה קרויים מכפות קרטזיות, לכבוד המתמטיקאי הצרפתי רנה דקארט.\\
\vspace{0,3cm}
חד-איברים ממעלה $1$ הם $x$, $y$, $z$. חד-איברים ממעלה $2$ הם $x^2, xy, y^2, xz, yz, z^2$. וכן הלאה. ככל שעולה המעלה, כך מתרבים החד-איברים, והדבר מגדיל את מספר האפשרויות ליצירת צורת בעלות מורכבות הולכת וגדלה. הדבר דומה לאלף-בית: ככל שיעמוד לרשותנו מספר אותיות גדול יותר, כך נוכל ליצור מילים ומשפטים מורכבים יותר.
\end{surferPage}
