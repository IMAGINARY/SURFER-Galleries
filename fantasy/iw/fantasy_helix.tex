\begin{surferPage}{חֶלזוֹנִית}
דק יותר מדופן של בועת סבון\\
  \smallskip
\[6x^2	= 2x^4	+ y^2	z^2\]

\singlespacing
בועות סבון הן מבנים רגישים; נדמה שדי במבט עיניים כדי לגרום להן להתנפץ. למשטח הבועה שני צדדים. בצד החיצוני נמצא הסבון, ואילו בצד הפנימי נמצאים המים. אם שכבת הסבון נעשית דקה מדי – וזה בדיוק מה שקורה כאשר הבועה גדלה – המים שבצד הפנימי יגרמו לבועה להתנפץ.\\
\vspace{0,3cm}
משטחים אלגבריים דקים הרבה יותר מדופן של בועת סבון. הם עשויים משכבות של נקודות. היות שנקודות אלה הן רק פרי הדמיון שלנו, ולפיכך אין להן מסה או צפיפות ממשית, הרי שהמשטחים אינם מתנפצים, גם אם נוצרים בהם פסגות וקמטים כמו בהֶליקס.\\
\vspace{0,3cm}
אך אם נרצה ליצור מודל תלת-ממדי של משטח ההֶליקס, יהיה עלינו לבנות "פסל" שעוביו גדול מזה של משטח ההליקס האמיתי. נוכל לעשות זאת על-ידי חיזוק המשטח בצד אחד.
\end{surferPage}
