\begin{surferIntroPage}{משטחים פנטסטיים}{fantasy_kolibri}{משטחים פנטסטיים}
לעתים קרובות אנו שומעים אמירות על המורכבות הרבה של המתמטיקה. זה אולי נכון, אך אין ספק שהמתמטיקה גם מסייעת לנו רבות בהבנה של עולמנו. לדוגמה, באמצעות זיהוי של מבנים בסיסיים ומאפיינים משותפים חשובים של עצמים ממשיים. פעולת הקיבוץ של כל העצמים בעלי אותם מאפיינים חשובים ב-\textit{מחלקה} אחת, תוך התעלמות מהמאפיינים הפחות חשובים נקראת סיווג או קלסיפיקציה. זוהי אחת הדרכים החשובות ביותר לקבלת מבט כולל על המגוון האינסופי של עצמים וצורות בעולמנו. המתמטיקה ממלאת תפקיד מכריע בדרך להשגת מטרה זו. כדי שנוכל להחליט מה חשוב ומה לא עלינו לדעת מה ברצוננו להבין. ה"מה" עשוי להיות למשל גודלו וצורתו של עצם מסוים.
\\

\vspace{0,4cm}

הצורך לתאר ולסווג צורות הוא צורך אנושי קדום, אך הדרך לעשות זאת אינה מובנת מאליה. היוונים הקדמונים השתמשו בעיקר בגאומטריה וביחסי ממדים – פרופורציות – של עצמים גאומטריים. מאוחר יותר, היו אלה הערבים שפיתחו את ענף האלגברה (אל-ח'ואריזמי, 900 לפנה"ס). המאה ה-18 מאופיינת בהישג חשוב של המתמטיקאים פֶרמָה (Fermat) ודֶקארט (Descartes), אשר החלו להשתמש במערכת קואורדינטות כדי לתאר יחסים גאומטריים. הדבר אפשר לעשות שימוש משותף באלגברה ובגאומטריה.
\\
\vspace{0,4cm}
תוכנת SURFER היא דוגמה מצוינת לקשר זה בין האלגברה לגאומטריה, היות שהיא יוצרת המחשות גאומטריות (התמונות) מתוך האלגברה (הנוסחאות).
בגלריית תמונות זו תוכלו לחוות בעצמכם את יופייה של המתמטיקה ולברוא יצירות חדשות בעצמכם. בחרו באחד המשטחים בצד ימין. הזיקה המתמטית בין הנוסחה לבין הצורה תוסבר בפשטות, באמצעות סדרת דוגמאות.\\
החלק של הדמיון והאינטואיציה הוא באחריותכם \dots
\end{surferIntroPage}
