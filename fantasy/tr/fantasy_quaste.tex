\begin{surferPage}{Kucak}
Denklemlerin ABC'si...
  \smallskip
\[8z^9-24x^2z^6-24y^2z^6+36z^8+24x^4z^3-168x^2y^2z^3\]
\[+24y^4z^3-72x^2z^5-72y^2z^5+54z^7-8x^6-24x^4y^2\]
\[-24x^2y^4-8y^6 + 36x^4z^2-252x^2y^2z^2+36y^4z^2\]
\[- 54x^2z^4-108y^2z^4 + 27z^6-108x^2y^2z + 54y^4z\]
\[-54y^2z^3 + 27y^4 = 0\]\\
\vspace{0.3cm}
Kucak yüzeyinin denklemine yakından baktınız mı? Çok karışık görünüyor, değil mi?
Oysa şekil basitçe tarif edilebiliyor: üst sınır Yunan alfabesindeki  $\alpha$ harfine benziyor; yan sınır bir sivrilik içeren bir eğri biçiminde. Bu tür bir sivriliğe {\it gaga (cusp)} diyoruz. Gagayı $\alpha$ eğrisi boyunca sürüklerseniz ortaya Kucak çıkar. Bu yöntemle elde edilebilen yüzeylere (Fransız matematikçi Ren\'e Descartes'a ithafen) Kartezyen çarpım denir.\\
\vspace{0.3cm}
Derecesi $1$ olan monomialler $x$, $y$ ve  $z$'dir. Derecesi  $2$ olanlarsa $x^2, xy, y^2, xz, yz, z^2$'dir vs. Derece arttıkça monomial sayısı da artar ve böylece daha karmaşık şekilleri yaratmak için daha fazla olanak doğar. Aslında bir alfabe gibi: eğer emrinizde daha çok harf varsa daha karmaşık sözcükler ve cümleler yazabilirsiniz.
\end{surferPage}
