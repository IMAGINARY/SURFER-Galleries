\begin{surferPage}{Fiyonk Makarna}
Denklem noktaları belirler...\\
  
  \smallskip
\[z^3+ y^2	z^2	= x^2\]

\singlespacing
Cebirsel ifadeyle, Fiyonk Makarna yüzeyi
\smallskip
\[ x^2= y^2z^2+z^3\]
\smallskip
denklemini sağlayan tüm  $(x, y, z)$ noktalarıyla verilir. Örneğin, $(0,0,0),$ $(1,0,1)$ ve $(3,-2,-3)$ noktaları Fiyonk Makarna üzerindedir;  $(0,1,1)$ ise üzerinde değildir.\\
 \singlespacing
Üç boyutlu dünyamız üç yönle tarif edilir: ileri/geri, sol/sağ ve yukarı/aşağı. Bu yönler  $x$, $y$ ve $z$ ile ifade edilir. Her nokta bu yönlerin her birinden bir değerle betimlenir. Bu değerlere noktanın koordinatları denir ve nokta $(x,y,z)$ diye yazılır.\\
\singlespacing
Şimdi uzayın tüm noktalarını bu değerleriyle denklemin içine sokuyoruz ve sadece denklemi sağlayanları boyuyoruz. Boyanan tüm noktalar hep beraber şekli oluşturuyor.
\end{surferPage}
