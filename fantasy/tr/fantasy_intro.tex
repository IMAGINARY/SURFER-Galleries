\begin{surferIntroPage}{Albenili Yüzeyler}{fantasy_kolibri}{Albenili Yüzeyler}
Matematiğin ne kadar karmaşık olduğu söylenir durur. Öte yandan dünyamızın karmaşıklığını anlamamıza yardımcı olduğu da bir gerçektir; örneğin, temel yapıları ve fiziksel nesnelerin önemli ortak özelliklerini tanımamızı sağlayarak bunu yapar.
Bir nesne \textit{sınıfında}, diğer tüm özellikleri göz ardı ederek  aynı önemli özelliğe sahip tüm nesneleri bir araya toplamaya sınıflandırma denir.
Sınıflandırma, dünyamızdaki nesne ve formların sonsuz çeşitliliğine dair bir genel bakış edinmemiz için önemli bir araçtır. Bu açıdan matematik esastır. Neyin önemli olup neyin olmadığına karar vermek, neyi  anlamak istediğinize bağlıdır. Anlamak istediğiniz şey örneğin bir nesnenin formu ya da büyüklüğü olabilir.
\\

\vspace{0.4cm}

Formu tarif etmek ve sınıflandırmak eski bir insan ihtiyacıdır; bunun nasıl yapılacağıysa bilinmezdir.
Eski Yunan, geometriyi ve geometrik nesnelerin oranlarını kullandı.
Daha sonra Araplar (Harezmi, İ.S. 900) cebri inşa ettiler. 17. yüzyılda Fermat ve Descartes'in büyük başarısıyla, geometrik ilişkileri tarif etmek için koordinat sistemleri keşfedildi. Böylece cebir ve geometri birlikte kullanılabilecekti.
\\
\vspace{0.4cm}
SURFER yazılımı işte bu ilişkinin birincil örneklerinden biridir; cebir (formül) aracılığıyla geometriyi (şekil, form) yaratır.
Bu galeride matematiğin güzelliğini bir deneyim olarak yaşayabilir ve kendiniz de yaratıcı olabilirsiniz. Sağ taraftaki yüzeylerden birini seçin. Formül ve form arasındaki bağlantı basit bir biçimde bir dizi örnekle açıklanmaktadır. \\
Hayal gücü ve sezgi sizden çıkıp gelecektir\dots
\end{surferIntroPage}
