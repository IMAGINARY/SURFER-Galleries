\begin{surferPage}{Maymuncuk}
Bir sabun köpüğünden bile ince...\\
  \smallskip
\[6x^2	= 2x^4	+ y^2	z^2\]

\singlespacing
Sabun köpükleri hassastır; bakarak bile onları patlatırmışız gibi gelir. Yüzeylerinin iki tarafı vardır. Dış taraf sabun, iç taraf sudur. Eğer sabun katmanı çok incelirse - yani balon büyüdükçe - su, balonun patlamasına yol açar.\\
\vspace{0,3cm}
Cebirsel yüzeyler sabun köpüklerinden çok daha incedir; noktalardan oluşmuş bir katmandırlar. Unutmayın, bu noktaları yaratmak için hayal gücümüzü kullanıyoruz. Noktaların ne kütleleri ne yoğunlukları var. Dolayısıyla, şu Maymuncuk gibi sivrilikleri ve kırışıklıkları olsa da,   bizim yüzeyler patlamaz.\\
\vspace{0,3cm}
Maymuncuk yüzeyinin üç boyutlu bir modelini inşa etmek istersek, asıl Maymuncuk yüzeyinden daha kalın bir heykel yapmak zorunda kalırız. Bunun için de yüzeyi bir tarafına doğru kalınlaştırıp sağlamlaştırmak gerekir.
\end{surferPage}
