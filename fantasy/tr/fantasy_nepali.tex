\begin{surferPage}{Nepalli}
Sonu gelmeyen dünya... \\

\smallskip
\[(x y - z^3 -1)^2= (1 - x^2	- y^2)^3\]

\singlespacing
Diyelim bir yüzeyi çok güzel buldunuz ve bir cam topun içine bir tutam karla birlikte yerleştirip oynamak istediniz. Sakın herhangi bir yüzeyi oturma odanıza götürebileceğinizi düşünmeyin!
\\
\singlespacing
Öyle yüzeyler vardır ki sonsuza kadar uzar giderler. Çok güzel olsalar da ne boyutta bir cam top alırsanız alın topun içine sığmazlar.
Böyle yüzeylere \textit{sınırlanmamış} diyoruz. Bunları resmedebilmek için bazı bölümlerini saklamamız gerekir elbette.
\\
\singlespacing
Sınırlanmışlık özelliği, SURFER yardımıyla bile, kolayca saptanabilecek bir özellik değildir.
Bu tıpkı Evren'in sınırlanmış olup olmadığını bulmaya benzer: sınırlarının var olup olmadığını bilmediğimizden olabilir de olmayabilir de.
\end{surferPage}
