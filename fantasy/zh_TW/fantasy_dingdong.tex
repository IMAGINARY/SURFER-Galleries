\begin{surferPage}{叮咚}

通過改變方程來改變圖像\\

\smallskip
\[x^2	+ y^2	+ z^3	= z^2\]

\singlespacing
叮咚的方程和形狀都很簡單。它的圖像可由希臘字母$\alpha$繞它的對稱軸旋轉而得到。如果你倒過來看,叮咚就像一滴水。我們可以看到這滴水正在下落。
\newline
如果在這個方程中加入一個小參數$a$並讓它連續的變化,我們可以創建一系列的圖像,這些圖像展示了這滴水出現的過程,它是如何接近它的結束位置並最終分離的。它就像一部電影的靜止圖像:
\smallskip

\[x^2	+ y^2	+ z^3	-z^2+0.1\cdot a=0.\]

\singlespacing
水滴在每一個時刻都處在引力和曲面張力相等的平衡狀態。但這種平衡是不穩定的,它在落下之前會抖動。由數學家勒內·托姆發展的突變理論就是研究參數在怎樣小的變化下會導致平衡狀態的突然改變。
\end{surferPage}
