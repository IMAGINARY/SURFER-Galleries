\begin{surferPage}[Würfel]{Ein Würfel}
Wenn man die Exponenten in der Kugelgleichung erhöht, passiert etwas Überraschendes. Die Kugel formt sich zu einem Würfel mit abgerundeten Kanten.\\
\vspace{0.3cm}

Doch das gilt nicht für alle Exponenten.
Die Gleichung für einen Würfel lautet:
\[x^n+y^n+z^n=b\]
wobei $n$ immer eine gerade Zahl sein muss.\\
\vspace{0.3cm}
Was passiert, wenn man für $n$ eine ungerade Zahl einsetzt?
Wie verändert sich der Würfel, wenn man für $n$ immer größere, gerade Zahlen einsetzt?
\end{surferPage}
