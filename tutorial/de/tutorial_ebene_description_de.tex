\documentclass[de]{./../../common/SurferDesc}%%%%%%%%%%%%%%%%%%%%%%%%%%%%%%%%%%%%%%%%%%%%%%%%%%%%%%%%%%%%%%%%%%%%%%%
%
% The document starts here:
%
\begin{document}
\footnotesize
% Einfache Singularit�ten 
%

\begin{surferPage}
  \begin{surferTitle}Die Ebene\end{surferTitle}
   \begin{surferText}

Die Gleichung f�r die Ebene lautet \[x=0.\] Sie ent�lt keine Informationen �ber $y$ und $z$. Das bedeutet, dass $y$ und $z$ nicht eingeschr�nkt sind, sie k�nnen also beliebige Werte annehmen.
Die Gleichung $x=0$ ist also f�r alle Punkte erf�llt, deren $x$-Werte Null und deren Werte f�r $y$ und $z$ beliebig sind.  Das ergibt die $yz$-Ebene.
\newline \newline
Wieso wieso sieht eine unendliche gro�e Fl�che aus wie ein gef\"ullter Kreis? Die Antwort liegt im Programm versteckt. Es legt immer eine unsichtbare Kugel um die Fl�che. Wir k�nnen nur das Innere der Kugel sehen. Sonst w�rde ja gar nicht alles auf den Bildschirm passen. 
     \end{surferText}
\end{surferPage}

%%%%%end
%%% Local Variables: 
%%% mode: latex
%%% TeX-master: "jDM08_expl"
%%% End: 



\end{document}
%
% end of the document.
%
%%%%%%%%%%%%%%%%%%%%%%%%%%%%%%%%%%%%%%%%%%%%%%%%%%%%%%%%%%%%%%%%%%%%%%%
