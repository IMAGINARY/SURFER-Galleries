\documentclass[de]{./../../common/SurferDesc}%%%%%%%%%%%%%%%%%%%%%%%%%%%%%%%%%%%%%%%%%%%%%%%%%%%%%%%%%%%%%%%%%%%%%%%
%
% The document starts here:
%
\begin{document}
\footnotesize
% Einfache Singularit�ten 
%
\begin{surferPage}
  \begin{surferTitle}Die Parabel\end{surferTitle}
   \begin{surferText}
   
Die Normalparabel wird mit der Gleichung \[y=x^2\] beschrieben.
Damit wir diese Gleichung in das Formelfeld eingeben k�nnen, m�ssen wir sie wieder nach Null umstellen:
\[x^2-y=0\]
Das kleine Dach im Formelfeld symbolisiert den Exponenten. Also 
\[ x  \,\hat{\ } \, 2 =x^2.\]
Wir f�gen noch zwei Parameter hinzu:
\[a \cdot x^2-y+b=0.\]
Der Parameter $a$ ist hierbei f�r Stauchung und Streckung der Parabel und der Parameter $b$ f�r den Abstand zum Ursprung verantwortlich.
\newline
F\"uge die Parameter $a$ und $b$ in die SURFER-Formel ein und ver\"andere sie. Wie muss man die Formel ver�ndern, damit die Parabel nach unten ge�ffnet ist?      
\end{surferText}
\end{surferPage}


\end{document}
%
% end of the document.
%
%%%%%%%%%%%%%%%%%%%%%%%%%%%%%%%%%%%%%%%%%%%%%%%%%%%%%%%%%%%%%%%%%%%%%%%
