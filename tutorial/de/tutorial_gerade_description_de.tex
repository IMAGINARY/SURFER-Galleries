\documentclass[de]{./../../common/SurferDesc}%%%%%%%%%%%%%%%%%%%%%%%%%%%%%%%%%%%%%%%%%%%%%%%%%%%%%%%%%%%%%%%%%%%%%%%
%
% The document starts here:
%
\begin{document}
\footnotesize
% Einfache Singularit�ten 
%

\begin{surferPage}
  \begin{surferTitle}Die Gerade\end{surferTitle}
   \begin{surferText}
   
Die Fl�che zeigt ein bekanntes Bild aus dem Schulunterricht: das Koordinatensystem, das durch $xy=0$ erzeugt wird und eine Gerade. Ihre Formel lautet:
\[y=a\cdot x + b\]
\[ \Rightarrow \quad a\cdot x +b -y=0.\]
Der Parameter $a$ ist hierbei der Anstieg der Geraden und der Parameter $b$ der Abstand zwischen der Geraden und dem Koordinatenursprung.
\newline \newline
Die Werte f�r $a$ und $b$ sind nicht fest. Rechts sind zwei Leisten aufgetaucht, an denen Sie die Werte f�r diese Parameter ver�ndern k�nnen. Durch Verschieben der Regler erkennt man, wie sich der Anstieg und der Abstand zum Ursprung f�r unterschiedliche Werte verhalten.

     \end{surferText}
\end{surferPage}
%%%End:

\end{document}
%
% end of the document.
%
%%%%%%%%%%%%%%%%%%%%%%%%%%%%%%%%%%%%%%%%%%%%%%%%%%%%%%%%%%%%%%%%%%%%%%%
