\begin{surferPage}{Цитрус}
В нижней строке Вы можете самостоятельно ввести формулу. Программа SURFER мгновенно рассчитает соответствующую поверхность и выведет её на экран. При этом справа от знака равенства в уравнении всегда стоит нуль. 
Уравнение может состоять из следующих элементов:
\newline
переменные:
\vspace{-1ex}\[x, y, z, \]\vspace{-1ex}
коэффициенты, заданные числами или параметрически:
\vspace{-1ex}\[1, 2, 3, \dots a, b, c, \dots, \]\vspace{-1ex}
математические операции:
\vspace{-1ex}\[+,  - , \cdot \quad \textnormal{и} \]\vspace{-1ex}
показатели степени:
\vspace{-1ex}\[ ^2, ^3, ^n .\]\vspace{-1ex}
Каждая точка координатной системы определяется значением трёх переменных $x$, $y$ и $z$. Если при подставновке координат точки в уравнение, с правой его части получается нуль, то окрасим такую точку в некоторый цвет. Программа определяет такие точки координатной системы в ходе так называемой «трассировки лучей». Эти точки называют нулями функции. Все окрашенные точки создают искомую поверхность.
\end{surferPage}
