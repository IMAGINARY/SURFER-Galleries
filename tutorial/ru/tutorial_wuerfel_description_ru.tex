\documentclass[ru]{./../../common/SurferDesc}%%%%%%%%%%%%%%%%%%%%%%%%%%%%%%%%%%%%%%%%%%%%%%%%%%%%%%%%%%%%%%%%%%%%%%%
%
% The document starts here:
%
\begin{document}
\footnotesize
% Einfache Singularitäten 
%

\begin{surferPage}
  \begin{surferTitle}Кубик\end{surferTitle}
   \begin{surferText}
   
Если увеличить показатель степени в уравнении сферы, то произойдёт что-то удивительное... Сфера приобретёт форму кубика с «округленными углами».\\
\vspace{0.3cm}
 Но это справедливо не для любого показателя степени. Уравнение кубика следующее:
\[x^n+y^n+z^n=b,\]
где $n$ – чётное число.\\
\vspace{0.3cm}
Что же произойдет, если $n$ – нечётное число? Что будет происходить с кубиком, если $n$ будет принимать всё большее, чётное значение?
     \end{surferText}
\end{surferPage}


\end{document}
%
% end of the document.
%
%%%%%%%%%%%%%%%%%%%%%%%%%%%%%%%%%%%%%%%%%%%%%%%%%%%%%%%%%%%%%%%%%%%%%%%
