\begin{surferPage}{Плоскость}
Уравнение плоскости: \[x=0.\] Оно не содержит информации об $y$ и $z$. Это значит, что $y$ и $z$ не ограничены, т.е. могут принимать любые значения. Таким образом, уравнение $x=0$ выполняется для всех точек, чья координата $х$ равна нулю, а координаты $y$ и $z$ – произвольные. В результате получаем плоскость $yz$. 
\newline \newline
Так почему же бесконечная поверхность выглядит как круг? Ответ скрывается в программе. Она всегда накладывает невидимую сферу на поверхность. И мы можем наблюдать лишь то, что находится внутри этого шара. В противном случае ничего не поместилось бы на монитор.
\end{surferPage}
