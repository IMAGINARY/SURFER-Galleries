\begin{surferPage}{Парабола}
Обычная парабола описывается уравнением \[y=x^2.\]
Для того, чтобы ввести это уравнение в поле для формул, мы должны его преобразовать, приравняв к нулю:
\[x^2-y=0\]
Маленькая «крышечка» в поле для введения формул – обозначения для показателя степени, т.е.
\[ x  \,\hat{\ } \, 2 =x^2.\]
Добавим ещё два параметра:
\[a \cdot x^2-y+b=0.\]
Параметр $a$ «отвечает» за сжатие и растяжение параболы, а параметр $b$ – за расстояние от центра координат.
\newline
Введите параметры $a$ и $b$ в формулу программы SURFER, меняйте их. Каким образом нужно изменить формулу, чтобы парабола «перевернулась»?
\end{surferPage}
