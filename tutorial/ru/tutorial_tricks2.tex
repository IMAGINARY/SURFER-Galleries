\begin{surferPage}[Трюки профи I]{II. Профессиональные трюки}
Кривая пересечения двух цилиндров в предыдущем примере – бесконечно тонкая линия, т.е. линия, «нарисованная» карандашом нулевой толщины. При помощи формулы
\[ f^2+g^2-a=0\]
можно рассчитать кривую пересечения двух поверхностей $f$ и $g$. Т.к. сумма двух квадратов равна нулю лишь, если оба слагаемых $f^2$ и $g^2$ одновременно равны $0$, то при $a = 0$ получаем кривую пересечения уравнений $f=0$ и $g=0$. Параметр $a$ влияет на уравнение: он задает толщину «карандаша», которым чертится линия. Если Вы измените $a$, то кривую пересечения можно сделать видимой.
\newline \newline
Поздравляем, теперь Вы – настоящий профессиональный SURFER! Удачи в конструировании новых формул!
\end{surferPage}
