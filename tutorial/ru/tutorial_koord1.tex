\begin{surferPage}[I. Коорд. система]{I. Координатная система}
Здесь Вы видите координатную систему, состоящую из трубочек, окружающих сами координатные оси. На самом деле, оси бесконечно тонкие, поэтому для их визуализации используются такие трубки.

Координатная система описывает трёхмерное пространство. Передвигаясь влево/вправо, мы перемещаемся по оси $x$; вверх/вниз – по оси $y$; вперед/назад – по оси $z$. Но направления при этом не зафиксированы жестко, так как координатную систему можно поворачивать.

Кликните пальцем по монитору и «перетащите» координатную систему. И вот она уже начинает поворачиваться. Для визуализации вращения внизу всегда показывается малая координатная система.
\end{surferPage}
