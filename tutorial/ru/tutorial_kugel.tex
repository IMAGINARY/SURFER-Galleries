\documentclass[ru]{./../../common/SurferDesc}%%%%%%%%%%%%%%%%%%%%%%%%%%%%%%%%%%%%%%%%%%%%%%%%%%%%%%%%%%%%%%%%%%%%%%%
%
% The document starts here:
%
\begin{document}
\footnotesize
% Einfache Singularitäten 
%
\begin{surferPage}
  \begin{surferTitle}Сфера\end{surferTitle}
   \begin{surferText}
   
В уравнении окружности присутствуют лишь переменные $x$ и $y$. Т.е. мы всё ещё находимся в двухмерном пространстве. Уравнение окружности: 
\[x^2+y^2=r^2.\]
Поверните поверхность в сторону так, чтобы увидеть одну из трубок. Эта трубка появляется, потому что вдоль оси $z$ нет ограничений. Если же заменить переменную $x$ на $z$, то снова появится трубка.\\
Прибавьте квадрат отсутствующей переменной (например, $z^2$ в случае к уравнению окружности), в результате получим сферу: 
\[x^2+y^2+z^2=r^2,\]
написание для программы SURFER:
\[0=x^2+y^2+z^2-a^2.\]
А что произойдет, если вращать сферу?

     \end{surferText}
\end{surferPage}


\end{document}
%
% end of the document.
%
%%%%%%%%%%%%%%%%%%%%%%%%%%%%%%%%%%%%%%%%%%%%%%%%%%%%%%%%%%%%%%%%%%%%%%%
