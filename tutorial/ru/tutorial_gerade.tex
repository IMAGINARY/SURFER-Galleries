\begin{surferPage}{Прямая}
Поверхность представляет собой известное изображение из школьного курса: координатная система и прямая. \\Формула прямой такова:
\[y=a\cdot x + b\]
\[ \Rightarrow \quad a\cdot x +b -y=0.\]
Параметр $а$ – это угловой коэффициент прямой, а параметр $b$ – расстояние по вертикали от координатного центра до прямой.
\newline \newline
Значения $а$ и $b$ изменяются. Справа даны $2$ поля, с помощью которых Вы можете менять параметры. Передвигая ползунок, можно понять, как при различных величинах коэффициентов ведут себя угол наклона прямой и расстояние от нее до начала координат.
\end{surferPage}
