\begin{surferPage}[I. Коорд. система]{II. Координатная система}
   Система координат, заданная формулой \[xy=0,\] состоит из двух пересекающихся плоскостей. А мы смотрим на них сверху. Поэтому третье измерение, т.е. ось $z$ нам не видна.\\
\vspace{0.3cm}
При $x=0$ получаем плоскость $yz$, при $y=0$ -- плоскость $xz$. В правой части обоих уравнений стоит ноль. Если их перемножить, то они будут изображены независимо друг от друга – в произведении множитель, если он равен нулю, приведет к тому, что всё уравнение равно нулю – т.е. поверхности будут объединены, в этом случае говорят, что их сложили.

Таким образом теоретически можно перемножить любое количество поверхностей. Лишь для программы SURFER все сложнее будет проводить соответствующие вычисления.
\end{surferPage}
