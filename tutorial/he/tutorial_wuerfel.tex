\begin{surferPage}[Cube]{קובייה}
כאשר מגדילים את ערך החזקות במשוואת הספרה, קורה משהו מפתיע. הספרה הופכת לקובייה עם פינות מעוגלות.\\
\vspace{0,3cm}
אך תוצאה זו אינה נכונה עבור כל החזקות. משוואת הקובייה היא:
\[x^n+y^n+z^n=b,\]
כאשר המספר $n$ חייב להיות זוגי.\\
\vspace{0,3cm}
מה קורה כאשר מציבים מספרים לא-זוגיים? כיצד משתנה הקובייה כאשר ממשיכים להגדיל את החזקות הזוגיות?
\end{surferPage}
