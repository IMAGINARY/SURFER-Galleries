\begin{surferPage}{לימון}
בשדה שלהלן תוכלו להזין נוסחה. תוכנית SURFER מחשבת את המשטח הנגזר מהמשוואה ומציגה אותו. על-מנת שהמחשב יוכל להבין את המשוואה, חלקה הימני חייב להיות שווה לאפס. 
\\
המשוואה עשויה לכלול את האלמנטים הבאים:
\newline
משתנים:
\[x, y, z, \]
מקדמים בצורת מספרים או פרמטרים:
\[1, 2, 3, \dots a, b, c, \dots, \]
פעולות אריתמטיות:
\[+,  - , \cdot \quad \textnormal{וכן} \]
חזקות:
\[ ^2, ^3, ^n .\]
כל נקודה במערכת הצירים מיוצגת על-ידי ערך עבור כל אחד משלושת המשתנים  $x$, $y$ ו- $z$. אם התנאי המחליף את המשתנים בערכים של נקודה זו שווה לאפס, הנקודה תוצג על המסך. באמצעות טכניקה הנקראת
 \textit{מעקב קרניים}
 
 \textit{ \textenglish{(ray tracing)}} התוכנית מאתרת את כל הנקודות האלה; הן נקראות נקודות האפס של המשוואה. סך כל הנקודות המוצגות מניב בעצם את המשטח המוצג על המסך.
\end{surferPage}
