\begin{surferPage}{직선}
다음은 우리가 교과서에서도 본 익숙한 그림입니다: $xy=0$ 으로 만들어지는 좌표계, 그리고 직선. \\
직선의 식은 다음과 같습니다:
\[y=a\cdot x + b\]
\[ \Rightarrow \quad a\cdot x +b -y=0.\]
매개변수 $a$는  직선의 기울기 그리고 매개변수 $b$는 $y$-절편 나타냅니다.
\newline \newline
$a$와 $b$의 값은 고정되지 않았습니다. 화면 오른쪽에 있는 세개의 막대기 중 왼쪽에 있는 두개를 이용하여 매변수의 크기를 변화시킬 수 있습니다. 여러분이 직접 값을 변화시키면서 기울기와 직선과 원점 사이의 거리가 어떻게 변하는지 볼 수 있습니다.
\end{surferPage}
