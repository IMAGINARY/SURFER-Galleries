\begin{surferPage}{Limun}
U polje ispod mo\v zete unijeti jednad\v zbu. SURFER tada odredi pripadnu plohu i prika\v ze ju. Desna strana jednad\v zbe mora biti jednaka nuli da bi ra\v cunalo moglo razumjeti jednad\v zbu.
\newline \newline
Jednad\v zba mo\v ze sadr\v zavati bilo koji od sljede\' cih elemenata:
\newline \newline
varijable:
\[x, y, z, \]
koeficijente u obliku brojeva ili parametara:
\[1, 2, 3, \dots a, b, c, \dots, \]
aritmeti\v cke operacije:
\[+,  - , \cdot \quad \textnormal{i} \]
eksponente:
\[ ^2, ^3, ^n .\]
Svaka to\v cka koordinatnog sustava je odre\dj ena s vrijednostima varijabli $x$, $y$ i $z$. Ukoliko u zadanom izrazu zamijenimo varijable s koordinatama neke to\v cke i dobijemo nulu, tada \' ce ta to\v cka biti prikazana. Aplikacija koristi \textit{Ray Tracing} tehniku kako bi prona\v sla i prikazala sve to\v cke koje zadovoljavaju zadanu jednad\v zbu. Takve to\v cke zovemo korijeni jednad\v zbe, a sve te to\v cke zajedno tada tvore plohu.
\end{surferPage}
