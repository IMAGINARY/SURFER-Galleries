\begin{surferPage}[Parabola]{Parabola}
Parabola je odre\dj ena jednad\v zbom \[y=x^2.\]
Kako bismo unijeli njenu jednad\v zbu u pripadno polje, moramo je prvo zapisati na na\v cin da s desne strane znaka jednakosti bude nula:
\[x^2-y=0\]
Eksponent unosimo pomo\' cu malog znaka u obliku krova. Dakle, imamo
\[ x  \,\hat{\ } \, 2 =x^2.\]
Dodajemo jo\v s dva parametra:
\[a \cdot x^2-y+b=0.\]
Parametar $a$ se koristi za skupljanje i rastezanje parabole, a parametar $b$ za promjenu udaljenosti parabole od ishodi\v sta.\\
\newline
Unesite parametre $a$ i $b$ u polje za formulu i poku\v sajte ih mijenjati. Na koji na\v cin morate promijeniti formulu da biste dobili parabolu koja je okrenuta naopako?
\end{surferPage}
