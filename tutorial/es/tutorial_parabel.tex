\begin{surferPage}{La Parábola}
La par{\'a}bola est{\'a} dada por la ecuaci{\'o}n  \[y=x^2.\]
Para ingresar esta f{\'o}rmula y que la entienda el SURFER, hay que reescribirla de manera tal que quede un cero a la derecha, es decir:
\[x^2-y=0\]
El {\it peque{\~n}o sombrerito} que se escribe despu{\'e}s de la $x$ simboliza que luego viene un exponente. Esto es: 
\[ x  \,\hat{\ } \, 2 =x^2.\]
Podemos agregarle a la f{\'o}rmula dos par{\'a}metros: $a$ y $b$.\newline
Y quedar{\'a} entonces:
\[a \cdot x^2-y+b=0.\]
El par{\'a}metro $a$ modificar{\'a} la compresi{\'o}n o dilataci{\'o}n de la par{\'a}bola, el par{\'a}metro $b$ indicar{\'a} la distancia de la par{\'a}bola al origen de coordenadas.
\newline
Insert{\'a} los par{\'a}metros $a$ y $b$ en la f{\'o}rmula SURFER y jug{\'a} con ellos. ?`C{\'o}mo hay que  modificar la f{\'o}rmula para que la par{\'a}bola est{\'e} al rev{\'e}s?
\end{surferPage}
