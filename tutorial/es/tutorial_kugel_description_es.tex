\documentclass[es]{SurferDesc}%%%%%%%%%%%%%%%%%%%%%%%%%%%%%%%%%%%%%%%%%%%%%%%%%%%%%%%%%%%%%%%%%%%%%%%
%
% The document starts here:
%
\begin{document}
\footnotesize
% Einfache Singularitten 
%
\begin{surferPage}
  \begin{surferTitle}La Esfera\end{surferTitle}
   \begin{surferText}
   
La ecuaci{\'o}n de la circunferencia s{\'o}lo contiene las variables $x$ e $y$. Estamos todav{\'i}a en el espacio de dos dimensiones:
\[x^2+y^2=r^2.\]
Si lo pon{\'e}s como una superficie, vas a ver un tubo. Esto es porque no hay ninguna restricci{\'o}n para el eje $z$. Si reemplaz{\'a}s la variable $x$ por $z$, todav{\'i}a tenemos un tubo.\\
Agreg{\'a} ahora el t{\'e}rmino cuadr{\'a}tico que le falta (por ejemplo $z^2$ a la ecuaci{\'o}n de la circunferencia). 
Y As{\'i} obtenemos la ecuaci{\'o}n de una esfera:
\[x^2+y^2+z^2=r^2,\]
En formato SURFER:
\[x^2+y^2+z^2-a^2=0.\]
?`Qu{\'e} pasa si giramos la esfera?

     \end{surferText}
\end{surferPage}


\end{document}
%
% end of the document.
%
%%%%%%%%%%%%%%%%%%%%%%%%%%%%%%%%%%%%%%%%%%%%%%%%%%%%%%%%%%%%%%%%%%%%%%%
