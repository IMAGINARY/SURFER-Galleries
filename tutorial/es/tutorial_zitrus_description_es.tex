\documentclass[es]{SurferDesc}%%%%%%%%%%%%%%%%%%%%%%%%%%%%%%%%%%%%%%%%%%%%%%%%%%%%%%%%%%%%%%%%%%%%%%%
%
% The document starts here:
%

\begin{document}
\footnotesize
% Einfache Singularit�ten 
%
\begin{surferPage}
  \begin{surferTitle}Lim{\'o}n
  \end{surferTitle}   %%% Zitrus

En la l{\'i}nea de abajo pod{\'e}s ingresar una f{\'o}rmula, una ecuaci{\'o}n siempre con un cero del lado derecho. Luego SURFER calcular{\'a} la superficie asociada y te la mostrar{\'a}. 
\\
Dicha ecuaci{\'o}n puede contener los siguientes elementos:
\newline
Variables:
\[x, y, z, \]
Coeficientes que pueden ser n{\'u}meros o par{\'a}metros:
\[1, 2, 3, \dots a, b, c, \dots, \]
Operaciones aritm{\'e}ticas:
\[+,  - , \cdot \quad \textnormal{y} \]
Exponentes:
\[ ^2, ^3, ^n .\]
Cada punto del sistema de coordenadas representa el reemplazo las variables $x$, $y$, $z$ por alg{\'u}n valor. Si los valores reemplazados en la ecuaci{\'o}n cumplen con que sea igual a cero, este punto se mostrar{\'a} en la pantalla. A trav{\'e}s de un m{\'e}todo llamado \textit{ray tracing} (trazado de rayos) el programa encuentra todos esos puntos; se llaman ceros de la ecuaci{\'o}n. Todos esos puntos juntos conforman la superficie que ves.

  
  \begin{surferText}
     \end{surferText}
\end{surferPage}

%%%%%%%%%%%%%%%%%%%%%%%%%%%%%%%%

%%% Local Variables: 
%%% mode: latex
%%% TeX-master: "jDM08_expl"
%%% End: 



\end{document}
%
% end of the document.
%
%%%%%%%%%%%%%%%%%%%%%%%%%%%%%%%%%%%%%%%%%%%%%%%%%%%%%%%%%%%%%%%%%%%%%%%
