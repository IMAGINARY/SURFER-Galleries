\documentclass[es]{SurferDesc}%%%%%%%%%%%%%%%%%%%%%%%%%%%%%%%%%%%%%%%%%%%%%%%%%%%%%%%%%%%%%%%%%%%%%%%
%
% The document starts here:
%
\begin{document}
\footnotesize
% Einfache Singularit�ten 
%

%%%%%%%%%%%%%%%%%%%%%%%%%%%%%%%%

\begin{surferPage}
  \begin{surferTitle}El Sistema de Coordenadas I \end{surferTitle}
   \begin{surferText}
   
Ac{\'a} pod{\'e}s ver un sistema de coordenadas hecho de tubos; en realidad, est{\'a}n colocados alrededor de los ejes reales. Los propios ejes son infinitamente delgados, as{\'i} que usamos estos tubos para poder verlos.\\
Este sistema de coordenadas describe nuestro espacio tridimensional. Si te mov{\'e}s a la izquierda o a la derecha te est{\'a}s moviendo a lo largo del eje $x$, si te mov{\'e}s hacia arriba o hacia abajo, lo hac{\'e}s a lo largo del eje $y$ y si te mov{\'e}s para adelante o para atr{\'a}s, a lo largo del eje $z$. Las direcciones no son fijas ya que pod{\'e}s cambiar el sistema de coordenadas.

\vspace{0.3cm}
Us{\'a} tu dedo para arrastrar el sistema de coordenadas. Mientras tanto, para visualizar la rotaci{\'o}n, vas a ver una peque{\~n}a representaci{\'o}n del sistema de coordendas.
     \end{surferText}
\end{surferPage}


\end{document}
%
% end of the document.
%
%%%%%%%%%%%%%%%%%%%%%%%%%%%%%%%%%%%%%%%%%%%%%%%%%%%%%%%%%%%%%%%%%%%%%%%
