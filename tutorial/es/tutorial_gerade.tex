\begin{surferPage}{La línea recta}
Esta superficie muestra una imagen que conocemos del colegio: un sistema de coordenadas dado por $xy=0$  y una l\'inea.\\
Su f\'ormula es:
\[y=a\cdot x + b\]
\[ \Rightarrow \quad a\cdot x +b -y=0.\]
El par\'ametro $a$ es la {\it pendiente} de la l\'inea y el par\'ametro $b$ es la {\it ordenada al origen}, es decir, el valor que toma la variable $y$ cuando la variable $x$ toma el valor cero.
\newline \newline
Los valores de $a$ y $b$ no est\'an fijos. A la derecha aparecieron dos controles para cambiar los valores de estos par\'ametros. Si modific\'as estos valores vas a ver c\'omo cambian la pendiente y la ordenada al origen.
\end{surferPage}
