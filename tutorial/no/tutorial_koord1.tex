\begin{surferPage}{Koordinatsystemet I}
Her ser du et koordinatsystem laget av rør som er lagt rundt de egentlige aksene. Selve aksene er uendelig tynne, så vi bruker disse rørene for å se dem.\\
Koordinatsystemet beskriver det tredimensjonale rommet. Hvis du beveger deg til venstre eller høyre, flytter du deg langs $x$-aksen. Beveger du deg opp eller ned, flytter du deg langs $y$-aksen, 
og er bevegelsen rettet framover eller bakover, er det z-aksen du flytter deg langs. Retningene er ikke faste, siden du kan dreie koordinatsystemet rundt. \\

\vspace{0.3cm}
For å dreie på koordinatsystemet, bruker du fingeren til å dra det et stykke bortover. Underveis vises en liten variant av koordinatsystemet for å visualisere rotasjonen.
\end{surferPage}
