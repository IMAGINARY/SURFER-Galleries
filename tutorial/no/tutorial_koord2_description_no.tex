\documentclass[no]{./../../common/SurferDesc}%%%%%%%%%%%%%%%%%%%%%%%%%%%%%%%%%%%%%%%%%%%%%%%%%%%%%%%%%%%%%%%%%%%%%%%
%
% The document starts here:
%
\begin{document}
\footnotesize
% Einfache Singularit�ten 
%

\begin{surferPage}
  \begin{surferTitle}The Coordinate System II\end{surferTitle}
   \begin{surferText}
   
The coordinate system given by 
\[xy=0\]
is not a real coordinate system; it is created by two intersecting planes. You look at the planes {\it from above}. This way, the third dimension, i.e. the $z$-axis is hidden. \\
\vspace{0.3cm}
From $x=0$ we obtain the $yz$-plane and from  $y=0$ the $xz$-plane.
Both equations have the ''equal to zero'' on their right side. If you multiply them they are displayed together, because a product is zero if one of the factors is zero. This means that the two surfaces are {\it added}. \\
This way we can add an arbitrary number of surfaces. However, it becomes more and more difficult to compute them.
\end{surferText}
\end{surferPage}




\end{document}
%
% end of the document.
%
%%%%%%%%%%%%%%%%%%%%%%%%%%%%%%%%%%%%%%%%%%%%%%%%%%%%%%%%%%%%%%%%%%%%%%%
