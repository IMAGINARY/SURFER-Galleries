\begin{surferPage}{Nappa}
L'ABC delle equazioni
  \smallskip
\[8z^9-24x^2z^6-24y^2z^6+36z^8+24x^4z^3-168x^2y^2z^3\]
\[+24y^4z^3-72x^2z^5-72y^2z^5+54z^7-8x^6-24x^4y^2\]
\[-24x^2y^4-8y^6 + 36x^4z^2-252x^2y^2z^2+36y^4z^2\]
\[- 54x^2z^4-108y^2z^4 + 27z^6-108x^2y^2z + 54y^4z\]
\[-54y^2z^3 + 27y^4 = 0\]\\
\vspace{0.3cm}
Hai guardato bene l'equazione della Nappa? Sembra molto complicata.
Invece si pu\`o spiegare la figura in modo abbastanza semplice: il bordo superiore ha la forma della lettera greca $\alpha$ (si legge alfa), il bordo destro ha la forma di una curva con una punta. Il nome tecnico \`e {\it cuspide}. Trascinando la cuspide lungo la curva a forma di $\alpha$ si ottiene la Nappa. Superfici con questa propriet\`a si chiamano prodotti cartesiano in onore del matematico francese Ren\'e Descartes.\\
\vspace{0.3cm}
Monomi di grado $1$ sono ad esempio $x$, $-y$, $-z$. Monomi di grado $2$ sono $x^2, xy, y^2, xz, yz, z^2$. Eccetera, eccetera. Al crescere del grado, aumentano i monomi a disposizione, cos\`{\i} otteniamo sempre pi\`u modi per creare forme via via pi\`u complicate. Sono come un alfabeto: pi\`u lettere abbiamo, pi\`u parole e frasi riusciamo a scrivere.
\end{surferPage}
