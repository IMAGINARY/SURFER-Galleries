\begin{surferPage}[Sfera]{La sfera}
L'equazione della circonferenza contiene solo le variabili $x$ e $y$. Siamo ancora nello spazio due-dimensionale.
L'equazione della circonferenza \`e:
\[x^2+y^2=r^2.\]
Se ruoti la superficie, puoi vedere un tubo. Questo succede perch\'e non ci sono restrizioni lungo l'asse $z$. Se sostituisci la variabile $x$ con $z$, ottieni ancora un tubo. Queste superfici si chiamano {\it cilindri}.\\
Ora aggiungi la variabile mancante al quadrato (per esempio aggiungi $z^2$ all'equazione della circonferenza).
Cos\`{\i} si ottiene una sfera:
\[x^2+y^2+z^2=r^2,\]
nel formato SURFER:
\[x^2+y^2+z^2-a^2=0.\]
Cosa succede quando ruoti la sfera? 
\end{surferPage}
