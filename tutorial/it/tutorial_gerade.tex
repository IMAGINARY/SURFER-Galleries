\begin{surferPage}[Retta]{La retta}
Questa superficie mostra un'immagine nota dai tempi della scuola: un sistema di coordinate, date da $xy=0$, e una retta. La sua formula usuale \`e
\[y=a\cdot x + b.\]
In SURFER si deve scrivere in modo che ci sia ``uguale a zero'' a destra
\[a\cdot x +b -y=0.\]
Il parametro $a$ \`e il coefficiente angolare della retta e il parametro $b$ \`e la distanza tra l'origine del sistema di coordinate e il punto di intersezione della retta con l'asse $y$.
I valori di $a$ e $b$ non sono fissati. Sul lato destro ci sono due cursori che si possono usare per cambiare i loro valori. Quando cambi i valori puoi vedere come l'inclinazione e la distanza dall'origine aumentano o diminuiscono.
\end{surferPage}


