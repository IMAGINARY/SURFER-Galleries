\begin{surferPage}{Zitrus}
Vous pouvez entrer une formule dans la zone ci-dessous. Ensuite, SURFER calculera la surface associée et l'affichera. La partie droite de la formule est toujours égale à zéro.
\\
Les équations peuvent contenir les éléments suivants :
\newline
Des variables : 
\[x, y, z, \]
Des coefficients, sous forme de nombres ou de paramètres : 
\[1, 2, 3, \dots a, b, c, \dots, \]
Des opérations arithmétiques :
\[+,  - , \cdot \quad \textnormal{and} \]
Et des exposants :
\[ ^2, ^3, ^n .\]
Chaque point du système de coordonnées est représentée par une valeur pour chacune des trois variables $x$, $y$ et $z$. Si l'expression calculée en remplaçant les variables par leurs valeurs en ce point vaut zéro, alors le point sera affiché. Le programme trouve tous ces points grave à une méthode appelée \textit{ray tracing}. Ces points sont appelés les racines ou les zéros de l'équation. Et c'est l'ensemble de tous ces points qui forme la surface.
\end{surferPage}
