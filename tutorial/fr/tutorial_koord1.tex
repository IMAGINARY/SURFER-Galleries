\begin{surferPage}[Coordonées I]{Le Système de Coordonnées I}
Vous voyez ici un système de coordonnées fait de tubes placés autour des vrais axes. Comme les axes eux-mêmes sont infiniment fins, on utilise ces tubes pour pouvoir les voir.\\
Le système de coordonnées décrit notre espace tridimentionnel. Si vous bougez à gauche ou à droite, vous vous déplacez selon l'axes des $x$, si vous bougez à gauche ou à droite, vous vous déplacez selon l'axes des $y$ et si vous bougez à gauche ou à droite, vous vous déplacez selon l'axes des $z$. Les directions ne sont pas fixées, puisque vous pouvez faire tourner les système de coordonnées.\\
\vspace{0.3cm}
Pour le tourner, faites glisser le système de coordonnées avec votre doigt. Pour visualiser la rotation, une petite représentation du système de coordonnées s'affiche quand cela tourne.
\end{surferPage}
