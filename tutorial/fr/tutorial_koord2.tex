\begin{surferPage}[Coordonées II]{Le Système de Coordonnées II}
Le système de coordonnées donné par
\[xy=0\]
n'est pas un vrai système de coordonnées. Il est créé par deux plans qui se croisent. Vous regardez ces plans {\it d'au-dessus}. Ainsi, la troisième dimension, c'est-à-dire l'axe des $z$ est caché. \\
\vspace{0.3cm}
Pour $x=0$, on obtient le plan $yz$ et pour $y=0$ le plan $xz$.
Les deux équations ont ''égal à zéro'' dans leur partie droite. Si vous les multipliez, elles sont affichées ensemble, car un produit est nul si un de ses facteurs est nul. Cela signifie que les deux surfaces {\it se superposent}. \\
De cette manière, on peut ajouter un nombre arbitraire de surfaces. Néanmoins, elle deviennent de plus en plus difficile à calculer.
\end{surferPage}
