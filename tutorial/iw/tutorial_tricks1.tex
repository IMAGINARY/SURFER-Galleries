\begin{surferPage}{טיפים למתקדמים I}
עד כה, למדתם רבות על התוכנית. כעת, הגיע הזמן לקבל כמה טיפים מעניינים:\\
\vspace{0,3cm}
כאשר מכפילים את המשוואות של שני משטחים, הם מצטרפים לזה לזה. אם כעת תחסירו ערך קטן מהמכפלה, העקומה הנוצרת במפגש ביניהם תהפוך להיות חלקה יותר. פירוש הדבר ששני המשטחים ניתכים זה לזה.
\vspace{0,3cm}
בנוסחה, אנו מחסירים את הפרמטר $a$. ערכו הראשוני הוא 0, אך אם תגדילו את ערכו, תוכלו לראות כיצד המשטחים ניתכים זה לזה (או נמסים זה לתוך זה).
\end{surferPage}
