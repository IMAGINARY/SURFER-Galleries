\begin{surferPage}{הקו}
משטח זה מציג תמונה המוכרת לנו מבית הספר: מערכת צירים (קואורדינטות) המתוארת על-ידי המשוואה $xy=0$ וקו. \\הנוסחה שלו היא:
\[y=a\cdot x + b\]
\[ \Rightarrow \quad a\cdot x +b -y=0.\]
הפרמטר $a$ הוא שיפוע הקו והפרמטר $b$ הוא המרחק שבין הקו לבין ראשית הצירים.
\newline \newline
הערכים עבור $a$ ו-$b$ אינם קבועים. המחוונים בצד ימין מאפשרים לשנות את ערכי הפרמטרים. כאשר משנים את הערכים ניתן לראות כיצד השיפוע והמרחק מראשית הצירים גדלים או קטנים.
\end{surferPage}
