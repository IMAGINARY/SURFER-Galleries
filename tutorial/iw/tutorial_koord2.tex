\begin{surferPage}{מערכת הצירים II}
מערכת הצירים המתוארת על-ידי
\[xy=0\]
אינה מערכת צירים אמתית; היא נוצרת על-ידי שני מישורים החוצים זה את זה. אתם מביטים במישורים {\it מלמעלה}. מסיבה זו, הממד השלישי, כלומר ציר $z$ נסתר מהעין. \\
\vspace{0,3cm}
מ-$x=0$ אנו מקבלים את מישור $yz$, ואילו מ-$y=0$ אנו מקבלים את מישור $xz$.
שתי המשוואות  "שוות לאפס" בגלל ספרת האפס בצד הימני של המשוואה. אם מכפילים את המשוואות, הן מוצגות יחד, היות שהמכפלה היא אפס כאשר אחד הגורמים הוא אפס. פירוש הדבר ששני המשטחים {\it נוספים זה לזה}. \\
בדרך זה נוכל להוסיף מספר כלשהו של משטחים. עם זאת, ככל שמוסיפים משטחים, ביצוע החישוב הופך יותר ויותר מורכב.
\end{surferPage}
