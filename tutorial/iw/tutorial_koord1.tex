\begin{surferPage}{מערכת הצירים I}
לפניכם מערכת צירים העשויה מצינורות המונחות מסביב לצירים עצמם. הצירים עצמם דקים לאין שיעור, ואנו משתמשים בצינורות כדי שתוכלו לראותם.\\
מערכת הצירים מתארת את המרחב התלת-ממדי שלנו. אם תנועו שמאלה או ימינה, תעשו זאת לאורך ציר ה-$x$; אם תנועו למעלה או למטה תעשו זאת לאורך ציר ה-$y$; ואם תנועו קדימה או אחורה התנועה תהיה לאורך ציר ה-$z$. הכיוונים אינם קבועים היות שאתם יכולים לסובב את מערכת הצירים.\\
\vspace{0,3cm}
כדי לסובב, השתמשו באצבע שלכם וגררו את מערכת הצירים לכיוון הרצוי. בזמן הסיבוב, מופיע ייצוג קטן של מערכת הצירים.
\end{surferPage}
