\begin{surferPage}{טיפים למתקדמים II}
העקומה הנוצרת במפגש בין שני הגלילים בדוגמה הקודמת, היא קו דק עד מאד. הוא דק כל כך שכדי לצייר אותו נזדקק לעיפרון שעוביו 0. באמצעות הנוסחה
\[ f^2+g^2-a=0\]
נוכל לחשב את עקומות המפגש של שני המשטחים $f$ ו-$g$. היות שהסכום $f^2+g^2$ של שני ריבועים הוא רק $0$ כאשר שני המחוברים שווים ל-$0$, מקבלים עבור $a=0$ את עקומת המפגש המתוארת על-ידי $f=0$ ו-$g=0$.
 הפרמטר $a$ מעוות את המשוואה: הוא מוסיף עובי לעיפרון שאתו מציירים את העקומה. אם תשנו את ערכו של $a$ תוכלו לעבות את עקומת המפגש.
\newline \newline
כל הכבוד! הצטרפתם לצוות המומחים של SURFER. אתם מוזמנים להמשיך וליהנות ביצירת צורות ומשטחים חדשים!
\end{surferPage}
