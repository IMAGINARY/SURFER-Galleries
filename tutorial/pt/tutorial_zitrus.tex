\begin{surferPage}{Citrus}
No campo em baixo  pode inserir uma f\'ormula. A seguir, o SURFER calcula a superf\'icie associada e exibe-a. O segundo membro da equa\c c\~ao  \'e sempre igual a zero.
\\
A equa\c c\~ao pode conter os seguintes elementos:
\newline
Vari\'aveis:
\[x, y, z, \]
Coeficientes em forma de n\'umeros ou par\^ametros:
\[1, 2, 3, \dots a, b, c, \dots, \]
Opera\c c\~oes aritm\'eticas:
\[+,  - , \cdot \quad \textnormal{e} \]
Expoentes:
\[ ^2, ^3, ^n .\]
Cada ponto do sistema de coordenadas \'e representado por um valor para cada uma das tr\^es vari\'aveis $x$, $y$ e $z$. Se o termo que substitui as vari\'aveis com os valores deste ponto for igual a zero, ent\~ao o ponto ir\'a ser exibido. Atrav\'es de um m\'etodo chamado \textit{tra\c cado de raios}, o programa encontra todos esses pontos; esses pontos dizem-se os {\it  zeros} da equa\c c\~ao. Todos esses pontos juntos formam, ent\~ao, a superf\'icie.
\end{surferPage}
