\begin{surferPage}[Dicas II]{Dicas de Especialista II}
A curva de interse\c c\~ao dos dois cilindros do exemplo anterior \'e uma linha infinitamente fina. Uma linha infinitamente fina \'e uma linha pintada com um pincel de espessura zero. Utilizando a f\'ormula
\[ f^2+g^2-a=0\]
podemos calcular as curvas de interse\c c\~ao de duas superf\'icies $f$ e $g$. Como a soma $f^2+g^2$ de dois quadrados \'e $0$ apenas se ambas as parcelas forem iguais a $0$, conclu\'imos que para $a=0$ obtemos a curva de interse\c c\~ao dada por $f=0$ e $g=0$.
O par\^ametro $a$ distorce a equa\c c\~ao: adiciona espessura ao pincel que utilizamos para pintar a curva de interse\c c\~ao. Se  modificarmos $a$,  podemos tornar a curva de interse\c c\~ao mais vis\'ivel.
\newline \newline
Parab\'ens, agora \'e um especialista no SURFER. Divirta-se a criar e a inventar novas formas!
\end{surferPage}
