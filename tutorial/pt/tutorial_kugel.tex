\begin{surferPage}[A Esfera]{A esfera}
A equa\c c\~ao da circunfer\^encia cont\'em apenas as vari\'aveis $x$ e $y$. Nesse caso, estamos ainda no espa\c co bidimensional.
A equa\c c\~ao da c\'ircunfer\^encia \'e:
\[x^2+y^2=r^2.\]
Se   rodarmos a superf\'icie,  podemos ver um tubo. Isso ocorre porque n\~ao h\'a nenhuma restri\c c\~ao ao longo do eixo dos $zz$. Se  substitu\'irmos a vari\'avel $x$ por $z$,   continuamos a ter um tubo.\\
Agora adicione o termo que falta com um quadrado (por exemplo $z^2$ para a equa\c c\~ao da circunfer\^encia). 
Esta altera\c c\~ao conduz  a uma superf\'icie esf\'erica:
\[x^2+y^2+z^2=r^2,\]
ou, no formato do SURFER
\[0=x^2+y^2+z^2-a^2.\]
O que acontece se  rodarmos a superf\'icie esf\'erica?
\end{surferPage}
