\begin{surferPage}{Um Cubo}
Se aumentar duas unidades nos expoentes da equa\c c\~ao da superf\'icie esf\'erica, algo de surpreendente ocorre. A superf\'icie esf\'erica transforma-se num cubo com os v\'ertices arredondados.\\
\vspace{0.3cm}
Mas tal n\~ao ocorre para todos os expoentes. A equa\c c\~ao para um cubo \'e:
\[x^n+y^n+z^n=b\]
onde o n\'umero $n$ tem de ser par.\\
\vspace{0.3cm}
O que acontece quando  escolhe n\'umeros \'impares? Como ocorrem as mudan\c cas no cubo quando se aumentam os expoentes pares?
\end{surferPage}
