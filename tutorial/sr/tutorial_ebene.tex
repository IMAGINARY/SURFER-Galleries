\documentclass[sr]{./../../common/SurferDesc}%%%%%%%%%%%%%%%%%%%%%%%%%%%%%%%%%%%%%%%%%%%%%%%%%%%%%%%%%%%%%%%%%%%%%%%
%
% The document starts here:
%
\begin{document}
\footnotesize
% Einfache Singularitäten 
%

\begin{surferPage}
  \begin{surferTitle}Раван\end{surferTitle}
   \begin{surferText}

Једначина равни је \[x=0.\] Не садржи информације о $y$ и $z$. To значи да $y$ и $z$ нису ограничени, већ могу да узимају произвољне вредности.
Једначина $x=0$ важи за све тачке за које је $x$ једнако нула а $y$ и $z$ имају произвољне вредности. Ово је раван $yz$.
\newline \newline
Али како изгледа бесконачна раван попуњена круговима? Одговор се крије у програму. Помоћу њега се површ увек приказује унутар невидљиве сфере и ми само можемо да видимо оно што је унутар сфере. У противном не би стало на екран.

     \end{surferText}
\end{surferPage}

%%%%%end
%%% Local Variables: 
%%% mode: latex
%%% TeX-master: "jDM08_expl"
%%% End: 



\end{document}
%
% end of the document.
%
%%%%%%%%%%%%%%%%%%%%%%%%%%%%%%%%%%%%%%%%%%%%%%%%%%%%%%%%%%%%%%%%%%%%%%%
