\begin{surferPage}{Парабола}
Парабола је задата једначином  \[y=x^2.\]
Да бисмо унели ову једначину у део за формуле, потребно је да је напишемо у еквивалентном облику где је једна страна једнакости нула:
\[x^2-y=0\]
Симбол који изгледа као мали кров означава експонент, у овом случају  
\[ x  \,\hat{\ } \, 2 =x^2.\]
Додајемо два параметра:
\[a \cdot x^2-y+b=0.\]
Параметар $a$ контролише растезање или сабијање параболе а параметар  $b$ растојање параболе од координатног почетка.
\newline
Унесите вредности параметара  $a$ и $b$ у SURFER програм и мењајте их. Како треба да промените формулу да би парабола била окренута надоле?
\end{surferPage}
