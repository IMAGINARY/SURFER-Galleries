\begin{surferPage}{Лимун}
У доње поље можете да унесете формулу. SURFER  након тога израчунава одговарајућу површ и приказује је. Формула са десне стране мора увек да буде једнака нули.
\\
Једначина може да садржи следеће елементе:
\newline
променљиве:
\[x, y, z, \]
коефицијенте у облику бројева или параметара:
\[1, 2, 3, \dots a, b, c, \dots, \]
аритметичке операције:
\[+,  - , \cdot \quad \textnormal{and} \]
експоненте:
\[ ^2, ^3, ^n .\]
Свака тачка координатног система је представљена помоћу вредности три променљиве $x$, $y$ и $z$. Ако је у одређеној тачки вредност једначине нула, та тачка се приказује. Помоћу методе зване \textit{ray tracing} програм проналази све такве тачке; оне се називају нуле једначине и све заједно образују површ.
\end{surferPage}
