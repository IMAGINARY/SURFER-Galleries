\begin{surferPage}[Sfera]{Sfera}
Ecua\c tia unui cerc folose\c ste numai variabilele $x$ \c si $y$. Suntem \^\i nc\u a \^\i n spa\c tiul de dimensiune 2. Ecua\c tia cercului este:
\[x^2+y^2=r^2.\]
Dac\u a roti\c ti suprafa\c ta, pute\c ti vedea un tub. Aceasta deoarece nu avem nicio restric\c tie referitoare la axa $z$. Dac\u a \^\i nlocui\c ti variabila $x$ cu $z$, ve\c ti g\u asi tot un tub.\\
Acum ad\u auga\c ti termenul lips\u a pentru ecua\c tia unei sfere (adic\u a $z^2$ la ecua\c tia cercului). Aceasta conduce la o sfer\u a:
\[x^2+y^2+z^2=r^2,\]
\^\i n format SURFER
\[0=x^2+y^2+z^2-a^2.\]
Ce se \^\i nt\^ampl\u a dac\u a roti\c ti sfera?
\end{surferPage}