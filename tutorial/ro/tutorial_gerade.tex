\begin{surferPage}[Dreapta]{Dreapta}
Aceast\u a suprafa\c t\u a reprezint\u a o imagine pe care o cunoa\c stem din \c scoal\u a: un sistem de coordonate, dat de ecua\c tia $xy=0$ \c si o dreapt\u a. \\ Formula sa este:
\[y=a\cdot x + b\]
\[ \Rightarrow \quad a\cdot x +b -y=0.\]
Parametrul $a$ este panta dreptei, iar parametrul $b$ controleaz\u a dep\u artarea dintre dreapt\u a \c si originea sistemului de coordonate.
\newline \newline
Valorile pentru  $a$ \c si $b$ nu sunt fixate. \^In partea dreapt\u a au ap\u arut dou\u a cursoare ce permit schimbarea valorilor acestor parametri. C\^and schimba\c ti valorile pute\c ti vedea cum se schimb\u a panta \c si dep\u artarea dreptei fa\c t\u a de origine.
\end{surferPage}