\begin{surferPage}[Coordonate I]{Sistemul de coordonate I}
Aici vede\c ti un sistem de coordonate construit din tuburi care sunt plasate \^\i n jurul axelor
autentice. Axele \^\i n sine sunt infinit de sub\c tiri, de aceea utiliz\u am tuburi pentru 
a le putea vizualiza.\\
Sistemul de coordonate descrie spa\c tiul nostru tridimensional. Dac\u a v\u a mi\c sca\c ti spre st\^anga sau dreapta, v\u a mi\c sca\c ti de-a lungul axei $x$-ilor. Dac\u a mi\c sca\c ti \^\i n sus sau \^\i n jos, v\u a mi\c sca\c ti de-a lungul axei $y$-ilor, iar dac\u a v\u a mi\c sca\c ti \^\i nainte sau \^\i napoi atunci v\u a mi\c sca\c ti de-a lungul axei $z$.  Direc\c tiile nu sunt fixate, deoarece pute\c ti roti sistemul de coordonate.
\\
\vspace{0.3cm}
Pentru a-l roti, folosi\c ti degetul \c si trage\c ti de sistemul de coordonate. Pentru a vizualiza rota\c tia, o reprezentare mic\u a a sistemului de coordonate este afi\c sat\u a \^\i n timpul rotirii.
\end{surferPage}