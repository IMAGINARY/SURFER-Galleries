\begin{surferPage}[Coordonate II]{Sistemul de coordonate II}
Sistemul de coordonate definit de 
\[xy=0\]
nu este un sistem de coordonate adev\u arat, este creat prin intersec\c tia a dou\u a plane. Privi\c ti aceste plane {\em de sus}. \^In acest mod, a treia dimensiune, adic\u a axa $z$ este ascuns\u a.
\\
\vspace{0.3cm}
Din $x=0$ ob\c tinem planul $yz$, iar din $y=0$ ob\c tinem planul $xz$. Ambele ecua\c tii au c\^ate un "egal cu zero" \^\i n partea dreapt\u a. Dac\u a le \^\i nmul\c ti\c ti, ele sunt afi\c sate \^\i mpreun\u a, deoarece un produs este zero dac\u a \c si numai dac\u a unul dintre factori este zero. Aceasta \^\i nseamn\u a ca suprafe\c tele se {\em adun\u a}.
\\
\^In acest mod putem ad\u auga un num\u ar arbitrar de suprafe\c te. Totu\c si calculul lor devine din ce \^\i n ce mai dificil.
\end{surferPage}