\begin{surferPage}{Citrica}
\^In c\^ampul de mai jos pute\c ti introduce o ecua\c tie. SURFER va calcula suprafa\c ta asociat\u a \c si o va reprezenta. Membrul drept al ecua\c tiei trebuie s\u a fie nul, pentru a permite calculatorului s\u a \^{\i}n\c teleag\u a ecua\c tia.\\
Ecua\c tia poate con\c tine oricare dintre elementele de mai jos:
\newline
Variabile:
\[x, y, z, \]
Coeficien\c ti, fie numerici, fie parametrici:
\[1, 2, 3, \dots a, b, c, \dots, \]
Opera\c tii aritmetice:
\[+,  - , \cdot \quad \textnormal{and} \]
Exponen\c ti:
\[ ^2, ^3, ^n .\]
Orice punct din sistemul de coordonate este reprezentat de valoarea fiec\u areia dintre cele trei variabile $x,$ $y$ \c si $z.$ Dac\u a termenul ob\c tinut \^{\i}nlocuind variabilele cu valorile fiec\u arui punct este zero, atunci punctul va fi reprezentat. Printr-o metod\u a numit\u a \textit{trasarea pe raze} programul g\u ase\c ste toate aceste puncte; ele se numesc \c si zerouri ale ecua\c tiei. Toate aceste puncte puse \^{\i}mpreun\u a formeaz\u a suprafa\c ta dat\u a.
\end{surferPage}