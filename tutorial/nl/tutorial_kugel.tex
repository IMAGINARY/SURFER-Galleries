\begin{surferPage}[Boloppervlak]{Het boloppervlak}
De vergelijking van de cirkel die je hieronder ziet bevat enkel de variabelen $x$ en $y$. We bevinden ons dus nog steeds in de tweedimensionale ruimte.
\[x^2+y^2=r^2\]
Als je het oppervlak draait, zie je een cilinder. Dit komt omdat er geen restrictie op de $z$-as wordt opgelegd. Als je de veranderlijke $x$ door $z$ vervangt, krijg je nog steeds een cilinder.\\
Voeg nu de ontbrekende term als kwadraat (bijvoorbeeld $z^2$) toe aan de cirkelvergelijking. 
Dit geeft een boloppervlak:
\[x^2+y^2+z^2=r^2,\]
of in SURFER-stijl
\[x^2+y^2+z^2-a^2 = 0.\]
Wat gebeurt er als je bol draait?
\end{surferPage}
