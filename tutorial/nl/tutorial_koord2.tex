\begin{surferPage}[Co\"ordinaten II]{Het co\"ordinatenstelsel II}
Het co\"ordinatenstelsel gegeven door
\[xy=0\]
is geen echt co\"ordinatenstelsel; het wordt immers gevormd door twee snijdende vlakken. Hier bekijk je de vlakken {\it van bovenaf}. Op die manier blijft de derde dimensie, dus de $z$-as, verborgen. \\
\vspace{0.3cm}
Uit $x=0$ bekomen we het $yz$-vlak en uit  $y = 0$ het $xz$-vlak.
Beide vergelijkingen worden in het rechterlid gelijk gesteld aan nul. Als je deze vergelijkingen met elkaar vermenigvuldigt worden de oppervlakken samen weergegeven, omdat een product van twee factoren gelijk is aan nul als en slechts als \'e\'en van die factoren gelijk is aan nul. Dit betekent dat de afzonderlijke oppervlakken worden {\it samengevoegd}. \\
Op die manier kunnen we een willekeurig aantal oppervlakken samenvoegen. Hoe meer er zijn, hoe moeilijker het wordt om ze te berekenen. 
\end{surferPage}
