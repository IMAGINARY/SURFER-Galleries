\begin{surferPage}[Coordinates II]{النظام الإحداثي 2}
ليس النظام الإحداثي المحدد بالمعادلة 
\[xy=0\]
نظاماً إحداثياً حقيقياً؛ فهو ناتج من تقاطع مستويين. تنظرون إلى المستويين {\it من فوق}. وهكذا، يكون البعد الثالث، أي محور $z$، مستوراً. \\
\vspace{0.3cm}
من $x=0$ نحصل على مستوي $yz$ ومن $y=0$ نحصل على مستوي $xz$. 
يساوي الطرف اليمين في كلتي المعادلتين صفراً. عند ضربهما، يتم عرضهما سوياً لأن حاصل الضرب يساوي صفراً إذا كان احد العوامل يساوي صفراً. هذا يعني أنه يتم {\it جمع} السطحين. \\
بهذه الطريقة، يمكننا جمع قدر ما نريده من السطوح. ولكن يصبح حسبها أكثر وأكثر صعوبة.
\end{surferPage}
