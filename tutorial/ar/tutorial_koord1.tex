\begin{surferPage}[إحداثيات 1]{النظام الإحداثي 1}
نرى هنا نظاماً إحداثياً مصنوعاً من أنابيب تم وضعها حول محاور حقيقية. المحاور بذاتها رفيعة جداً فنستخدم الأنابيب لنتمكن من رؤيتها.\\
يصف النظام الإحداثي فضاءنا الثلاثي الأبعاد. التحرك نحو الشمال أو نحو اليمين يوازي التحرك على طول محور $x$، التحرك نحو الأعلى أو نحو الأسفل يوازي التحرك على طول محور $y$، التحرك نحو الأمام أو نحو الوراء يوازي التحرك على طول محور $z$. الإتجاهات ليست ثابتة إذ بالإمكان تدوير النظام الإحداثي.\\ 
\vspace{0.3cm}
من أجل تدويره، يجب إستعمال الإصبع لسحب النظام الإحداثي. من أجل رؤية الدوران، يتم عرض تمثيل مصغر للنظام الإحداثي أثناء دورانه.
\end{surferPage}