\begin{surferPage}{Zitrus}
في الحقل أدناه، يمكن إدخال معادلة. فيحسب برنامج \textenglish{SURFER} السطح التابع لهذه المعادلة ويعرضه. لكي يتمكن الحاسوب من فهم المعادلة، بجب أن يكون الطرف الأيمن للمعادلة مساوياً لصفر.
\\
يمكن أن تحتوي المعادلة على أي من العناصر التالية:
\newline
متغيرات:
\[x, y, z, \]
معاملات على شكل أرقام أو بارمترات:
\[1, 2, 3, \dots a, b, c, \dots, \]
عمليات حسابية:
\[+,  - , \cdot \quad \textnormal{and} \]
أس أو قوة العدد:
\[ ^2, ^3, ^n .\]
بإعطاء قيمة رقمية لكل متغير من المتغيرات الثلاث $x$ و$y$ و$z$، يتم تمثيل كل نقطة من نظام الإحداثيات. إذا كانت الصيغة الناتجة عن إستبدال المتغيرات بقيم رقمية تساوي صفر، فسيتم عرض النقطة. بفضل تقنية راي تريسينغ
\textenglish{(\textit{ray tracing})},
يجد البرنامج كافة هذه النقاط المسماة جذور المعادلة. ويتشكل السطح عندها من مجموعة هذه النقاط.
\end{surferPage}
